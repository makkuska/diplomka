\section{Návrh replikačního řešení}
  Po provedení rešerše a zohlednění všech podmínek, požadavků a možností
  katedry, byl sestaven návrh kompletního databázového řešení založeného na
  procesu replikace. Z databázových serverů byl vybrán server PostgreSQL hned
  z~několika důvodů. Jedná se o plnohodnotný databázový systém dostupný zda\-rma
  se všemi nástroji, je široce používaný v oblasti geoinformačních technologií,
  je multiplatfomní a od verze ArcGIS 9.3 plně podporováný produkty ArcGIS.
  Návrh počítá s použitím ArcSDE pro propojení databáze s ArcGIS produkty. Při
  výběru verzí je nutné zajistit kompatibilitu verzí jednotlivých software,
  a~poté ArcSDE nainstalovat společně s PostrgreSQL.

  Byl navržen replikační cluster s nejméně třemi servery z důvodů, které již
  byly diskutovány v kapitole \ref{kReplikace}. Celý cluster poběží na stejné
  platformě a proto bude možno použít streaming replikaci se všemi výhodami
  zmíněnými v kapitole \ref{kReplikace}. Byla zvolena jednosměrná master-slave
  replikace, cluster tedy bude obsahovat jeden master a dva (popř. více) slave
  serverů. Aby nedošlo ke ztrátě dat v~případě, že by master server spadl dřív,
  než se data zkopírují na slave server, pro první slave (\texttt{slave1}) byla
  zvolena varianta synchronní replikace. Je vhodné, aby servery běžely
  v~lokální síti z důvodu rychlosti a spolehlivosti spojení mezi master a slave
  serverem. 


  Druhý server (\texttt{slave2}) bude replikovat asynchronně a zároveň, aby
  ne\-do\-chá\-ze\-lo k~přetížení master serveru, bude replikace probíhat ze
  \texttt{slave1} na \texttt{slave2}, tedy kaskádově. Tím bude řešení zároveň
  přípraveno na výpadek master serveru, protože v~případě, že master vypadne,
  \texttt{slave1} bude povýšen na master a \texttt{slave2} bude ihned moci
  replikovat. Ze \texttt{slave2} lze dále tvořit pravidelnou, například denní
  nebo týdenní, zálohu pomocí ulitily \texttt{pg\_dump}. Záloha pomocí
  \texttt{pg\_dump} tak nebude zatěžovat master server a sama o sobě bude
  probíhat rychleji, než by tomu bylo na master serveru, který je již tak velmi
  vytížen dalšími procesy.

  \begin{center}
    \includegraphics[width=\textwidth]{obr/schema_navrhKatedra.png}
       Srovnání multimaster a master-slave replikace
  \end{center}

Uživatelé se budou připojovat skrze nástroj pgpool, který se bude tvářit jako jediný databázový server, ke kterému se klienti přihlásí bez ohledu na typ jejich dotazu a on sám pak rozhodne, ke kterému ze serverů klienta přihlásí. Tím bude mít zároveň možnost rozložit zátěž na dostupné uzly v clusteru. Pro ještě větší efektivitu provozu databáze bude pgpool uchovávat databázová spojení a při novém dotazu využije stávajícího spojení, místo aby vytvářel spojení nové. 

Vzhledem k tomu, že klienti budou k databázovému serveru přistupovat skrze pgpool, není potřeba aby jednotlivé uzly v clusteru měly veřejnou IP adresu. Plně dostačuje, že servery poběží na lokální síti a pouze pgpool bude na serveru s veřejnou IP, čímž se zajistí, že data budou přístupná z internetu. 

Návrh počítá také s externími pracovišti, která budou často číst z databáze a~budou mít zájem o zrychlení přístupu k datům tím, že se slave server přesune na jejich pracoviště. Typ replikace se zvolí podle jejich operačního systému a jeho architektury. Pokud se bude jednat o shodný systém, jaký bude použit ve výše popsaném clusteru, pak bude možno použít asynchronní streaming replikaci, naopak pokud se bude jednat o systém jiný, bude použita Slony-I replikace. 

