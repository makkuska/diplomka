\section{Úvod}
Dnešní trend je ukládat a ponechávat stále více dat pouze v digitální podobě. Mnoho dokumentů už se vůbec netiskne do papírové podoby, což podporuje i trend e\-le\-ktro\-nic\-kých schránek a podpisů. S přibývajícím množstvím dat je však třeba řešit komplikace, které informace uložené pouze v elektronické podobě přinášejí. Počítačoví experti řeší například otázky, kam ukládat tak velké množství dat, jak data efektivně aktualizovat, jak zabránit poškození dat ať už způsobených lidským faktorem či chybou hardware. V případě, že se poškodí disk, můžeme často během okamžiku přijít o~všechna data, někdy však pro ztrátu dat stačí pouze stisknout tlačítko na klávesnici.

Vhodným způsobem uchovávání dat je ukládaní do databáze s následnou replikací. Replikací je myšlena pokročilá funkcionalita, která zajišťuje kopii dat na více serverů. Nabízí ji většina dnešních databázových serverů, zajišťuje větší robustnost databáze a vysokou dostupnost dat. Replikaci lze využít ve všech odvětvích, která pracují s daty. Výjimkou není ani geoinformatika, která často pracuje s~velkými objemy dat, které nesou informaci o geografické poloze. Právě reprezentace geografické polohy, skrze textový zápis souřadnic daných bodů, může způsobit razantní zvýšení objemu dat. 

U webových map se musí řešit velký počet dotazů do databáze, protože například každé posunutí výřezu či přiblížení, resp. oddálení výřezu mapy, je samostatným dotazem, který musí kapacita serveru zvládat. Například pokud bude uživatel procházet plánovanou 100km trasu posouváním výřezu mapy po 10~km, může to serveru způsobit velkou zátěž. Replikaci ocení uživatelé pracující na společném projektu, distribuovaná pra\-co\-viš\-tě i společnosti s velkým množstvím důležitých dat, jejichž dostupnost je rozhodující pro jejich fungování. 


