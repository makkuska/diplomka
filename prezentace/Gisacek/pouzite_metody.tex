\section{Použité metody a programové komponenty}
Konfigurace replikace zahrnovala studium návodů jednotlivých nástrojů pro
replikaci, výběr vhodných programových komponent a jejich následné praktické
nastavení. To bylo testováno průběžně na několika počítačích. 

Jako databázový server byl zvolen {\it PostgreSQL} s plnou podporou pro správu
prostorových dat, která je zajištěna nádstavbou {\it PostGIS}. Pro replikaci byla
zvolena nativní {\it PostgreSQL streaming replikace} a externí nástroj {\it Slony-I}. Pro
efektivní využívání databáze byl dále vybrán externí nástroj pgpool, který
zajišťuje snížení zátěže jednotlivých serverů rovnoměrným rozkládáním dotazů od
klientů mezi jednotlivé databáze. 

Nástroj pro replikaci Slony-I byl testován na operačním systému Ubuntu
GNU/Li\-nux 12.4 a zároveň na operačním systému Windows XP. 

Nativní PostgreSQL streaming replikace byla testována pouze na operačním
systému Linux. Server geohydro.upol.cz byl poskytnut jako testovaní server pro
tuto práci. Na server byl nainstalován 32bitový operační systém Debian
GNU/Li\-nux 7.3, který byl vybrán kvůli jeho stabilitě a jevil se tedy pro server
jako vhodný. Tato verze ovšem umožnila instalaci pouze programů verzí
PostgreSQL 9.1, PostGIS 1.5 a pgpool 3.1. Vzhledem k tomu, že se nejedná o
nejnovější verze zmíněných produktů, byla replikace testována také na osobním
počítači ve verzích PostgreSQL 9.3, PostGIS 2.1 a pgpool 3.3. To umožnilo
nastudování dalších možností, které nové verze přináší a které byly zohledněny
v návrhu replikačního řešení. 

Pro testování byla používána ukázková prostorová data vytvořená pro účel této
práce a dále byla na server uložena datová sada ArcČR ve verzi 3.0.


