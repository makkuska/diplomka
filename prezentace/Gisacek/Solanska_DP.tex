\documentclass{llncs}
\usepackage[czech]{babel}
\usepackage[utf8]{inputenc}
\usepackage{graphicx}
\usepackage{amssymb}
\input{unity}

\title{Synchronizace a replikace geodat v prostředí Esri platformy}
\author{Markéta Solanská}
\institute{Katedra geoinformatiky, Přírodovědecká fakulta, Univerzita Palackého v Olomouci, 17. listopadu 50, 779 00 Olomouc, Česká republika, \\
\email{marketa.solanska@gmail.com}} 

\begin{document}
\maketitle

\begin{abstract}
  Tato práce hodnotí možnosti dostupných replikačních řešení a na základě toho
  navrhuje databázové řešení s ohledem na možnosti a požadavky katedry. V
  rešerší části byly vymezeny pojmy synchronizace, replikace a související
  pojem verzování a popsána replikace včetně variant synchronní, asynchronní,
  jednosměrné, obousměrné, kaskádové, logické i fyzické. Byly rozebrány
  požadavky na databázové ukládání dat jednotlivých produktů ArcGIS a byla
  podrobně popsána technologie ArcSDE, která se v ArcGIS produktech používá pro
  připojení k databázi. Na základě rešerše byl vybrán databázový systém
  PostgreSQL, který je možno použít v kombinaci s produkty ArcGIS, což bylo
  jedním z hlavních požadavků pro výběr databázového systému. Byl sestaven
  návrh databázového řešení, který zohledňuje všechny požadavky katedry a
  možnosti daných technologií. Bylo vytvořeno testovací prostředí na serveru
  poskytnutém katedrou, na němž byly dané procesy otestovány. Na základě toho
  byl pak sepsán podrobný popis toho, jak nastavit replikaci ve variantě
  streaming a Slony-I. Návrh zahrnuje také možnost použití nástroje pgpool pro
  rozložení zátěže mezi servery v databázovém clusteru.
\end{abstract}


\begin{keywords}
  replikace, synchronizace, verzování, databázový systém, Po\-stgreSQL, ArcSDE, ArcGIS
\end{keywords}

\begin{abstractEnglish}
  The main goal of this thesis is to evaluate options of replication solutions
  which are available and based on this research design a~database solution
  which considers possibilities and requirements of the Department of
  Geoinformatics. In the theoretical part terms replication, synchronization
  and versioning are defined including description of synchronous,
  asynchronous, master-slave, multimaster, cascade, logical and physical
  replication. The requirements of ArcGIS products for storage of data in
  database were considered and ArcSDE Technology which is used by ArcGIS
  products for database storage of spatial data was described. Based on the
  research database management system PostgreSQL was chosen because it is
  supported by ArcGIS products. The design of the database solution was created
  based on all requirements and the main processes were tested. Based on that a
  manual of the proposed replication solution setup was written. Two
  replication options were tested - PostgreSQL native streaming replication and
  replication using PostgreSQL extension Slony-I. The design includes a
  description of usage of pgpool utility used for load-balancing. 
\end{abstractEnglish}

\begin{keywordsEnglish}
  replication, synchronization, versioning, database management system, PostgreSQL, ArcSDE, ArcGIS
\end{keywordsEnglish}

Dnešní trend je ukládat a ponechávat stále více dat pouze v digitální podobě. Mnoho dokumentů už se vůbec netiskne do papírové podoby, tím spíš pokud dnes existují elektronické podpisy, díky kterým je tištěná verze naprosto zbytečná. S přibývajícím počtem dat je však třeba řešit komplikace, které počítačová data přinášejí. Počítačoví experti řeší například otázky, kam ukládat tak velké množství dat, jak data efektivně aktualizovat, jak zabránit poškození dat ať už způsobených lidským faktorem či fyzickým poškozením hardware. V připadě, že se poškodí disk, můžeme často během okamžiku přijít o všechna data, někdy však pro ztrátu dat stačí pouze stisknout tlačítko na klávesnici. Určitě už se Vám nejednou stalo, že jste se nemohli přihlásit do svého účtu na internetu z důvodu přetížení serveru. I to je problém, který velké množství dat a velký počet uživatelů přináší. Jak tedy pracovat s těmito objemy, jak zabránit komplikacím, které mohou poškodit či zcela zničit celou dosavadní práci, a jak zrychlit celý proces práce s daty? 

Řešením velkého počtu výše uvedených problémů může být ukládaní dat do databáze a jejich následná replikace. Replikací je myšlena pokročilá funkce, která zajišťuje kopii dat na více serverů. Nabízí ji většina dnešních databázových serverů, zajišťuje větší robustnost databáze a vysokou dostupnost dat. Replikaci lze využít ve všech odvětvích, které pracují s daty. Výjimkou tedy není ani geoinformatika, která pracuje s velkými objemy dat, které navíc nesou informaci o geografické poloze. Právě reprezentace geografické polohy, skrze textový zápis souřadnice daných bodů, může způsobit razantní zvýšení velikosti dat. Například u webových dat se navím musí řešit častý přístup k databázi, protože každé posunutí výřezu či přiblížiní, resp. oddálení výřezu mapy, je samostatným dotazem, který musí kapacita serveru zvládat. Při představě, že si uživatel bude posouvat výřez mapy po 50m, může to způsobit velkou zátěž pro server. V tomto případě je potřeba řešit replikaci z důvodu rozložení zátěže. 

Z mého pohledu data středně velkého až velkého projektu je vhodnější ukládata do databáze než jiných formátů typu shapefile, Microsoft Access nebo obyčejného tabulkového procesoru. Nabízí nám to sofistikované uložení dat, snadné propojení jednotlivých vrstev, snadnou přenostitelnost dat, možnost relačního propojení dat nebo efektivní vyhledávání. Replikace samotná se poté využívá pro kopii dat a následnou aktualizaci změn, která v databázi nastanou. 

Replikaci ocení uživatelé pracující na společném projektu, distribuovaná pracoviště i společnosti s velkým množstvím důležitých dat, jejichž kopie je rozhodující pro jejich fungování. Dobrým příkladem využitelnosti replikace je také nový trend využívání offline aplikací v mobilních telefonech. Databáze se vždy replikuje do mobilního telefonu, kde může fungovat offline a vždy, když se klient připojit na internetovou síť, aplikace kontroluje zda není na serveru novější verze databáze a pokud je, zkopíruje pouze změny, které proběhly od posledního stahování. (Jako příklad z geoinformatického prostředí bych uvedla diplomovou práci Dalibora Janáka, který řeší replikaci databáze lezeckých cest do mobilní aplikace.) 

Databázové systémy nabízí širokou škálu nastavitelnosti, která umožňuje přizpůsobit replikaci danému řešení.

\section{Použité metody a programové komponenty}
Konfigurace replikace zahrnovala studium návodů jednotlivých nástrojů pro
replikaci, výběr vhodných programových komponent a jejich následné praktické
nastavení. To bylo testováno průběžně na několika počítačích. 

Jako databázový server byl zvolen {\it PostgreSQL} s plnou podporou pro správu
prostorových dat, která je zajištěna nádstavbou {\it PostGIS}. Pro replikaci byla
zvolena nativní {\it PostgreSQL streaming replikace} a externí nástroj {\it Slony-I}. Pro
efektivní využívání databáze byl dále vybrán externí nástroj pgpool, který
zajišťuje snížení zátěže jednotlivých serverů rovnoměrným rozkládáním dotazů od
klientů mezi jednotlivé databáze. 

Nástroj pro replikaci Slony-I byl testován na operačním systému Ubuntu
GNU/Li\-nux 12.4 a zároveň na operačním systému Windows XP. 

Nativní PostgreSQL streaming replikace byla testována pouze na operačním
systému Linux. Server geohydro.upol.cz byl poskytnut jako testovaní server pro
tuto práci. Na server byl nainstalován 32bitový operační systém Debian
GNU/Li\-nux 7.3, který byl vybrán kvůli jeho stabilitě a jevil se tedy pro server
jako vhodný. Tato verze ovšem umožnila instalaci pouze programů verzí
PostgreSQL 9.1, PostGIS 1.5 a pgpool 3.1. Vzhledem k tomu, že se nejedná o
nejnovější verze zmíněných produktů, byla replikace testována také na osobním
počítači ve verzích PostgreSQL 9.3, PostGIS 2.1 a pgpool 3.3. To umožnilo
nastudování dalších možností, které nové verze přináší a které byly zohledněny
v návrhu replikačního řešení. 

Pro testování byla používána ukázková prostorová data vytvořená pro účel této
práce a dále byla na server uložena datová sada ArcČR ve verzi 3.0.



\section{Vymezení pojmů}
\label{kVymezeniPojmu}

Databáze je strukturovaná kolekce dat, která slouží pro efektivní ukládání dat a~jejich zpětně čtení \cite{Oppel2009}. V relační databázi jsou data ukládána ve formě tabulek, tedy entit a atributů, které jsou vzájemně propojeny vazbami mezi entitami \cite{Connolly2005}. Toto logické uložení vazeb mezi tabulkami umožňuje efektivní manipulaci s daty, rychlé vyhledávání i komplexní analýzu \cite{Momjian2001}. 

Obvykle se rozlišují pojem databáze, který odkazuje na obecný koncept, a pojem databázový systém nebo přesněji {\it systém řízení báze dat} \footnote{angl. Database Management System (DBMS)}, což je konkrétním počítačovým program, který zajišťuje fyzické uložení dat. Moderní SŘBD jsou navrženy na principu klient/server, kdy databáze běží jako služba na pozadí a čeká na dotazy od klientů. Server uživatelům umožňuje skrze jazyk SQL přístupovat k databázi, vytvářet a aktualizovat data, stejně jak jako vyhledávat či analyzovat \cite{Connolly2005}.

{\it Prostorová databáze}, někdy také zvaná {\it geodatabáze}, není nic jiného než databáze obohacená o datový typ určený pro ukládání prostorové informace o~prvku, prostorové indexy a sadu funkcí vhodných pro správu prostorových dat. Dnes umožňují ukládat prostorová data například databázové systémy PostgreSQL 9.x, Microsoft SQL Server, Oracle Database, MySQL nebo SQLite. 

Pojmy replikace a synchronizace některé zdroje rozlišují, jiné je naopak po\-va\-žu\-jí za synonyma. Všechny zmíněné pojmy souvisí se zálohováním dat, tedy kopírovaním dat mezi dvěmi a více uložišti, a se liší konkrétním důvodem pro použití daného procesu. 

O synchronizaci souborů či datových složek je možno mluvit v případě, že existují dva datové zdroje, které je potřeba v daný okamžik sjednotit. Jde tedy o proces, který probíhá jednorázově a to většinou z důvodů potřeby porovnání dvou a více datových uložišť, které je potřeba dostat do totožného stavu. To může například přispět snazší spolupráci více uživatelů nad stejnými daty nebo pomoct uživateli, který pracuje na více počítačích. Proces může proběhnout jednou nebo opakovaně, ať už pravidelně či nepravidelně. U souborů se shodným názvem se porovnává čas posledního zápisu, velikost nebo obsah souboru, naopak soubory, u kterých není nalezena shoda, jsou jednoduše zkopírovány. 

Replikace je proces průběžný, který soustavně hlídá, zda ve zdrojových datech nedošlo ke změně, a pokud ano, dané změny zkopíruje na jiné datové uložiště. Často je tento proces používán právě ve spojitosti s databázemi, kdy jsou data kopírována z~důvodu snížení zátěže serveru, či zvýšení ochrany dat. Replikace je tedy často vyžadována z jiných důvodů než synchronizace, začíná s daty existujícími pouze na jednom uložišti a pro zajištění konzistence dat používá jiných technologií. Více se replikací zabývá kapitola \ref{kReplikace}.

Oba procesy je možno použít jednostranně, tedy kopírovat data pouze z~jednoho uložiště na druhé a nikoliv opačně, nebo oboustraně, kdy se data kopírují navzájem mezi sebou.


        \subsection{Replikace}
        \label{kReplikace}
Replikace je proces, u kterého jsou data a databázové objekty kopírované z jednoho databázového serveru na druhý a poté synchronizovány pro zachování souladu obou databází. Synchronizací v tomto případě myslíme kopírováním všech změn, které v databázi nastanou. Použitím databáze je možno data distribuovat na různě vzdálená místa nebo mezi mobilní uživatele v rámci počítačové sítě a internetu \citep{Microsoft2013}.

Mnohé moderní aplikace se musí zabývat velkým počtem přístupů do databáze, což může v některých případech způsobovat problémy. Buď je server přetížen počtem připojení a data tedy přicházejí k uživateli pomalu, nebo dokonce úplně vypadne. 

Mezi časté důvody použití databázové replikace tedy patří zajištění dostupnosti dat\footnote{angl. High Availability}, resp. snížení pravděpodobnosti, že data nebudou dostupná, což může být způsobeno již zmíněným výpadkem serveru nebo například fyzickou ztrátou dat \citep{ObeHsu2012}. Další důvodem je rozložení zátěže přístupů do databáze mezi více serverů, takže nebude docházet ke zpomalení výkonu hlavního serveru ani k situaci, že data nebudou dostupná kvůli jeho výpadku \citep{BellKindahlThalmann2010}. Databáze je často zálohovaná, například skriptem dump a i to může server zpomalit. Vhodným řešením je tedy nejdříve vytvořit kopii dat na jiný datový server a až poté proces zálohování spustit. 

Všechny databáze zapojené do procesu replikace jsou v odborné literatuře nazývané uzly, v angličtině node. Tyto uzly dohromady tvoří replikační cluster\footnote{Volně přeloženo skupina serveru zapojených do replikace}. Při správně nastavené replikaci, by v clusteru nikdy neměly být méně než 3 uzly. Může se totiž stát, že vypadne jeden ze dvou uzlů, čímž dojde, ikdyž jen na krátkou chvíli, k situaci, že data nebudou v daný okamžik zálohovaná. 

Uzly v replikačním clusteru mohou mít jednu ze dvou základních rolí, nejčastěji nazývaných Master a Slave. Master server nebo pouze Master je server, který poskytuje data k replikaci, má práva na čtení i zápis a probíhají tedy na něm veškeré aktualizace. Je možno se setkat také s pojmenováním Primary server, Provider, Sender, Parent nebo Source server. Naprosto jiný pojem zavádí MS SQL Server, který tento zdrojový server nazývá Publisher (česky Vydavatel). Druhý databázový server je nejčastěji nazýván Slave, Standby, Reciever, Child nebo Subsciber (česky Odběratel). Poslední pojem je také používán MS SQL Serverem. Na tento server, který je dostupný vždy jen pro čtení dat, se data a aktualizace kopírují, není však možné na něj změny zapisovat \citep{RiggsKrossing2010}.

        %parametr H říká že to bude přímo na tom místě kde je v textu...více http://en.wikibooks.org/wiki/LaTeX/Floats,_Figures_and_Captions
          \begin{figure}[H]
            \centering
            \includegraphics[scale=1]{../../../grafy/obr/schema_masterMasterSlave_maxiTence.png}
            \caption {Srovnání Master-Master a Master-Slave replikace}
            \label{srovnaniM-M-S}
          \end{figure}

Podle počtu Master a Slave serverů v replikačním clusteru, se rozlišuje zda se jedná o jednosměrnou nebo obousměrnou replikaci. Tzv. Master-Master replikace umožňuje zapisovat do všech uzlů v replikačním clusteru, což může být praktické například při použití databáze offline \odkazObrazek{srovnaniM-M-S}. Změny se tedy synchronizují mezi všemi databázovými uzly. Tento způsob však nese značné komplikace, je potřeba řešit konflikty změn ve stejných datech a je relativně náročný na údržbu. Tato práce se zabývá použitím druhé způsobu, tzv Master-Slave replikace. Tato replikace používá vždy jen jeden Master server v clusteru a dva a více Slave servery. Kopie dat tedy probíhá jednosměrně, vždy z Master na Slave servery. Podle Bella (2010) mají moderní aplikace často více čtenářů než zapisovatelů, proto je zbytečné, aby se všichni čtenáři připojovali na stejnou databázi jako zapisovatelé a zpomalovali tím jejich práci \citep{BellKindahlThalmann2010}. Z toho důvodu je tedy použití Master-Slave replikace více než vhodné.

Při návrhu replikace je potřeba zamyslet se také nad tím, zda bude synchronní či asynchronní. Synchronní replikace neumožní, aby na Master serveru proběhla nová transakce, dokud se poslední transakce úspěšně neprovede na Slave serveru \citep{Boszormenyi2013}. Tento přístup zajistí, že žádná data nebudou v průběhu transakce ztracena. V některých případech tento způsob může zbytečně zpomalit rychlost přístupu do databáze, protože je nutno čekat na každou nedokončenou transakci. Zároveň může způsobit snížení dostupnosti databáze, protože v případě, že se například přeruší spojení mezi servery, nemůže na masteru proběhnout žádný další dotaz. Ale jistě si najde své opodstatění například při bankovních transakcí, kde je potřeba, aby všechny operace proběhly na obou stranách. V tomto případě je užití tohoto způsobu zcela nezbytné. 

Druhým způsobem je asynchronní replikace, při které se nová data mohou zapisovat na Master server, přestože ještě nedošlo k replikaci stávajících dat na Slave server \citep{ObeHsu2012}. To je sice za běžného provozu rychlejší, v některý případech však může způsobit nekonzistenci dat, například když proběhne transakce na Master serveru, který však spadne dřív, než se změna zapíše na Slave. V takovém případě se Slave změní na Master server, ale zároveň se nikdy nedozví o transakci, o které má uživatel informace, že proběhla v pořádku. 

        \begin{figure}[H]
          \centering
          \includegraphics[scale=1]{../../../grafy/obr/schema_asyncSync_maxiTence.png}
          \caption {Rozdíl mezi synchronní a asanchronní replikací}
        \end{figure}

Replikace v PostgreSQL umožňuje plnou kopii dat z databáze i pouze výběr některých tabulek. Více o možnostech a způsobech nastavení replikace v kapitole \odkazKapitola{} a PRAKTICKÁ ČÁST :)
Dále je možno rozlišovat replikaci pole toho, zda je logická nebo fyzická. Výsledek obou typů má naprosto identický výsledek, přesto se mírně liší. 

Fyzická replikace kopíruje data na druhý server v binární podobě. Tím, že se kopírují celé složky dat, je na Slave serverech zajištěna identická replika. Protože se kopírují binární data, která mají jasně danou strukturu, je potřeba mít na obou serveru stejnou platformu a architekturu. Tento způsob je velice spolehlivý a často snazší na konfiguraci. Naopak logická přenáší SQL příkazy tak, jak byly použity na Master serveru a ty poté proběhnou na Slave serverech. Tím se nasimuluje průběh změn dat na hlavním serveru a zajistí se konzistence dat. Tento způsob je více flexibilní, umožňuje výběr jen několika databází nebo tabulek a není závislý na architektuře ani operačním systému \citep{Boszormenyi2013}. 

Posledním diskutovaným pojmem je kaskádová replikace, která umožňuje připojit repliku k jinému Slave serveru místo k hlavnímu Master serveru. Tento způsob může být výhodných předeším z těchto dvou důvodů. Řekněme, že se kaskádová replikace použivá při existenci většího počtu Slave serverů v clusteru, třeba sta. V případě, že by se všechny repliky připojovaly k hlavnímu serveru, došlo by u něj k razantnímu zpomalení jeho výkonu. Kaskádová replikace může být praktická také v okamžiku, kdy se data přenáší na velkou vzdálenost, třeba do Číny. V případě, že mají v Číně dvě repliky, je zcela zbytečné, aby se obě kopie přenášely na tak velkou vzdálenost, když druhá replika se může připojit k první a mít data s mnohem menším zpožděním.

          \begin{figure}[H]
            \centering
            \includegraphics[scale=1]{../../../grafy/obr/schema_kaskadova.png}
            \caption{Ukázka kaskádové replikace}
            \label{kaskadova}
          \end{figure}

Každý databázový server (myšleno SŘDB) si volí terminologii a konkrétní nastavení mírně odlišně. Tato kapitola se snaží popsat chápání replikace co v největší míře obecně s ohledem na použití tohoto pojmu v PostgreSQL. Zcela jinou terminologii, ikdyž založenou na stejných principech, zavádí MS SQL Server, který používá pojmy transakční replikace pro Master-Slave replikace a slučovací replikaci pro Master-Master replikaci. 



Tato kapitola se zabývá hodnocením současného stavu správy dat na katedře geoinformatiky, návrhem databázového řešení dle požadavků a možností katedry a podrobně popisuje vytvoření testovacího prostředí na serverech katedry dle vytvořeného návrhu. Do hloubky popisuje konfiguraci vybraných nástrojů, včetně jejich praktického spuštění. 

\subsection{Aktuální stav správy dat}
\label{kAktualniStav}

Katedra aktuálně provozuje tři servery, konkrétně virtus.upol.cz, gislib.upol.cz a geohydro.upol.cz. Poslední z jmenovaných byl poskytnut jako testovací server pro tuto práci a v budoucnu s ním počítá jako s master serverem pro zde popisované databázové řešení. První dva zmíněné servery jsou aktivně používány, hostují například geoportál publikovaný skrze ArcGIS Server, který je důležitým prostředkem pro prezentaci projektů a dat, která na katedře vznikají. Data ke geoportálu i dalším aplikacím běžícím na těchto serverech jsou ukládána do MS SQL Serveru, každý ze serverů momentálně obsahuje jiné datové sady, které nejsou pravidelně zálohovány, protože aktualizace dat není příliš častá. Aktuální řešení nepoužívá replikaci dat, což může způsobovat nedostupnost dat z důvodu výpadku serveru. 

Databáze aktuálně obsahují data například z projektů BotanGIS\footnote{\url{http://botangis.upol.cz/botangis/mapa}}, Virtuální studovna CHKO Litovelské Pomoraví\footnote{http://virtus.upol.cz/}, dále data metadatového systému Micka\footnote{\url{gislib.upol.cz/metadata}}, data ze senzorové sítě KGI, data ke studentským pracím a také ukázková data určená pro výuku. Je založeno přibližně 10 účtů, které mají přístup pro zápis, a řádově v desítkách účtů s právem čtení. V současné situaci není do databází příliš často zapisováno. 

Současný stav, kdy se přenášejí data přes různá hardwarová zařízení nebo kopírují po síti, není plně vyhovující z několika důvodů. Často jedná o velké objemy dat, jejichž kopie může trvat řádově až desítky minut. Studenti si musejí dělat kopie dat při každém cvičení, což velice zdržuje výuku. Data jsou poté fyzicky uložena na počítačích v učebnách, což mimo jiné dovoluje, aby se k datům například z různých projektů dostal kdokoliv, kdo má přístup na učebnu. Při každé aktualizaci dat je navíc potřeba data opět zkopírovat, což je další časové omezení, k tomu může dojít k nekonzistenci dat různých datových zdrojů. 

\subsection{Požadavky na databázové řešení}
\label{kPozadavky}

Katedra má zájem využít potenciál databázového řešení a plánuje využít tento návrh k uložení dalších datových sad, které má k dispozici a které jsou momentálně dostupné ve formátech shapefile nebo geodatabáze, ale zatím nejsou uložená v databázi, kterou je možno sdílet. Jedná se například o datové sady ArcČR500 verze 2.0 a 3.0, Data200 (ČUZK), CEDA ČR 150, data, která byla uvolněna pro podporu pro Krajinotvorný program MŽP, nebo data dostupná k produktům Esri nebo Idrisi. Data uložená v databázi pak budou mnohem snáze využitelná jak pracovníky, tak i studenty katedry, kteří data využijí nejen ve výuce, ale také v jejich odborných prácích. Při kopírováním dat na různá datová uložiště je navíc těžké udržet licenční podmínky, se kterými jsou data pořizována. 



\subsection{Návrh replikačního řešení}
\label{kNavrh}

Po provedení rešeršní části a zohlednění všech podmínek, požadavků a možností katedry, byl sestaven následující návrh pro kompletní databázové řešení založené na procesu replikaci. Z databázových serverů, diskutovaných v kapitole \odkazKapitola{kPouziteProstredky}, byl vybrán server PostgreSQL hned z několika důvodů. Jedná se o plnohodnotný databázový systém dostupný zdarma se všemi nástroji, je multiplatfomní a od verze ArcGIS 9.3 je plně podporován produkty ArcGIS. 

Byl navržen replikační cluster s nejméně třemi servery z důvodů, které již byly diskutovány v kapitole \odkazKapitola{kReplikace}. Celý cluster poběží na stejné platformě a proto bude možno použít streaming replikaci, která jakožto nativní řešení PostgreSQL, nabízí větší stabilitu a bezpečnost, díky přenosu transakčních logů, než jiná diskutovaná řešení. Byla zvolena jednosměrná master-slave replikace, cluster tedy bude obsahovat jeden master a dva (popř. více) slave serverů. Aby nedošlo ke ztrátě dat v případě, že by master server spadl dřív, než se data zkopírují na slave server, pro první slave (slave1) byla zvolena varianta synchronní replikace. Je vhodné, aby servery běžely v~lokální síti, protože se tím snižuje pravděpodobnost, že by došlo k~výpadku spojení mezi master a slave1 server a nebylo by tak možno zapisovat na master. 

Druhý server (slave2) bude replikovat asynchronně a zároveň, aby nedocházelo k~přetížení master serveru, bude replikace probíhat ze slave1 na slave2, tedy kaskádově. Ze slave2 lze dále tvořit pravidelnou, například denní nebo týdenní, zálohu pomocí ulitily pg\_dump, která je více popsána v kapitole \odkazKapitola{kPriprava}. Záloha přes pg\_dump tak nebude zatěžovat master server a sama o sobě bude probíhat rychleji, než by tomu bylo na master serveru, který je již tak velmi vytížen dalšími procesy.


Vzhledem k tomu, že existují klienti, kteří mají právo číst i zapisovat, budou přístupy do datábáze řešeny nástrojem pgpool. Uživatelům tedy nebude potřeba dávat přihlašovací údaje dvakrát, jednou pro zápis na master a druhý pro čtení na slave. To jim usnadní práci i z toho pohledu, že si nebudou muset hlídat, ke kterému ze serverů se připojit na základě jejich aktuálního dotazu. pgpool se bude tvářit jako jakákoli jiná databáze, ke které se klienti přihlásí bez ohledu na typ jejich dotazu a pgpool pak sám rozhodne, ke kterému ze serverů klienta přihlásí. Tím bude mít možnost také rozložit zátěž na dostupné uzly v clusteru dle počtu konkrétních dotazů. pgpool bude zároveň uchovávat databázová spojení a při novém dotazu využije stávajícího spojení, místo aby vytvářel spojení nové. Tímto se zajistí plynulost a zvýší rychlost provozu databáze.


        \begin{figure}[H]
          \label{oNavrhKatedra}
          \centering
          \includegraphics[scale=1]{../../../grafy/obr/schema_navrhKatedra2.png}
          \caption {Návrh replikačního řešení}
        \end{figure}

        Vzhledem k tomu, že se k databázi bude přistupovat skrze pgpool, není potřeba aby jednotlivé uzly v clusteru měly veřejnou IP adresu. Plně dostačuje, že servery poběží na lokální síti a pouze pgpool na serveru s veřejnou IP, čímž se zajistí, že data budou přístupná i skrze internet. 

Návrh počítá také s externími pracovišti, která budou často přistupovat do databáze s právem čtení, a budou mít zájem o zrychlení přístupu k datům tím, že se slave server přesune na jejich pracoviště, tedy na hardware, který bude připojen do jejich lokální sítě. Typ replikace se zvolí podle jejich operačního systému a jeho architektury. Pokud se bude jednat o shodný systém, jaký bude použit ve výše popsaném clusteru, pak bude možno použít asynchronní streaming replikaci, naopak pokud se bude bude jednat o systém jiný, bude použita Slony-I replikace.


\begin{thebibliography}{99}
    \bibitem{Oppel2009} OPPEL, A. J. \emph{Databases: A Beginner’s Guide}. New York: McGraw-Hill, 2009, 164 s. ISBN 00-716-0846-X.
\bibitem{Connolly2005} CONNOLLY, T. \emph{Database Systems: A Practical Approach to Design, Implementation, and Management}. Vyd. 4. Harlow: Addison-Wesley, 2005, 1374 s. ISBN 03-212-1025-5.
\bibitem{Momjian2001} MOMJIAN, B. \emph{PostgreSQL: Introduction and Concepts}. Boston, MA: Addison-Wesley, 2001, xxviii, 461 s. ISBN 02-017-0331-9.
    \bibitem{Microsoft2013} MICROSOFT. SQL Server - Replication. \emph{Microsoft} [online], 2013 [cit. 2013-08-27]. Dostupné z: http://technet.microsoft.com/en-us/library/ms151198(v=sql.100).aspx.
    \bibitem{ObeHsu2012} OBE, R., HSU, L. \emph{Postgresql: Up and Running}. Sebastopol, CA: O’Reilly, 2012, 164 s. ISBN 978-144-9326-333.
    \bibitem{BellKindahlThalmann2010} BELL, C., KINDAHL, M., THALMANN, L. \emph{MySQL High Availability}. Vyd. 1. Sebastopol, CA: O’Reilly Media, Inc, 2010. ISBN 978-059-6807-306.
    \bibitem{RiggsKrossing2010} RIGGS, S., KROSING, H. \emph{PostgreSQL 9 Administration Cookbook: Solve real-world     PostgreSQL problems with over 100 simple, yet incredibly effective recipes}. Birmingham: Packt Publishing, 2010, 345 s. ISBN 978-1-849510-28-8.
    \bibitem{Boszormenyi2013} BÖSZÖRMENYI, Z., SCHÖNIG, H.-J. \emph{PostgreSQL Replication: Understand basic replication concepts and efficiently replicate PostgreSQL using high-end techniques to protect your data and run your server without interruptions}. Vyd. 1. Birmingham: Packt Publishing, 2013, vii, 230 s. ISBN 978-1-84951-672-3.

\end{thebibliography}

\end{document}
