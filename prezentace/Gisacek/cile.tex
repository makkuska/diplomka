\section{Cíle}
Cílem práce je provést rešerši v oblasti dostupných replikačních řešení a~na jejím základě prakticky otestovat proces synchronizace a replikace geodat s ohledem na možnosti kombinace s produkty ArcGIS. V rešerši budou diskutovány databázové servery SQL Server a PostgreSQL, oba podporované produkty ArcGIS, a na jejím základě pak bude vybrán jeden, na kterém bude proces replikace prakticky testován.

V teoretické části práce budou podrobně definovány pojmy týkající se zálohování dat, především však synchronizace a replikace, dále detailně rozebrána replikace ve všech možných variantách nastavení, tedy jednosměrná, obousměrná, synchronní, asynchronní, kaskádová, logická a fyzická. Dále rozbor zahrne celé portfolio produktů od desktopového řešení, přes možnosti ArcGIS serveru až po cloudový ArcGIS online.

Praktická část se bude zabývat návrhem replikačního řešení zohledňujícího požadavky katedry na databázové řešení a bude brát v úvahu její možnosti a způsob využívání databáze. Bude připraveno testovací prostředí na základě vytvořeného návrhu, které bude následně prakticky vyzkoušeno. Bude pozorováno, zda replikace probíhá bez chyb a jsou přenesena všechna data v relativně krátkém časovém horizontu. 


