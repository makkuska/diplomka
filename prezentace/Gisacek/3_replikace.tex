\subsection{Replikace}
\label{kReplikace}
  Replikace je proces, u kterého jsou data a databázové objekty kopírovány z~jednoho databázového serveru na druhý a poté synchronizovány pro zachování
  identity obou databází. Synchronizací je v tomto případě myšleno kopírování
  všech změn, které v~databázi nastanou. Použitím replikace je možno data
  distribuovat na různě vzdálená místa nebo mezi mobilní uživatele v rámci
  počítačové sítě a~internetu \cite{Microsoft2013}.

  Vývojáři mnohých moderních aplikací se musí zabývat přetížením serveru
  způsobených velkým počtem současných přístupů do databáze. V případě pře\-tí\-že\-ní
  se prodlouží odezva serveru, data tedy přicházejí k uživateli pomalu, nebo
  server dokonce úplně spadne. 

  Mezi časté důvody použití databázové replikace tedy patří zajištění dostupnosti
  dat\footnote{angl. High Availability}, resp. snížení pravděpodobnosti, že data
  nebudou dostupná \cite{ObeHsu2012}. Další důvodem je rozložení přístupů do
  databáze mezi více serverů, takže nebude docházet ke zpomalení výkonu hlavního
  serveru \cite{BellKindahlThalmann2010}. Ke zpomalení serveru dochází také při
  zálohování, což lze řešit replikací dat na jiný server, na kterém je pak proces
  zálohování spuštěn. 

  Všechny databázové servery zapojené do procesu replikace jsou v odborné
  li\-te\-ra\-tu\-ře nazývány {\it uzly}, angl. node. Tyto uzly dohromady tvoří
  {\it replikační cluster}~\footnote{volně přeloženo jako skupina serverů
  zapojených do replikace}. Při správně nastavené replikaci, jejímž cílem je
  zajištění vysoké dostupnosti dat (HA), by v clusteru nikdy neměly být méně než
  tři uzly. Může se totiž stát, že vypadne jeden ze dvou uzlů, čímž dojde k
  situaci, že data v daný okamžik nebudou zálohovaná. 

  Uzly v replikačním clusteru mohou mít jednu ze dvou základních rolí, nej\-čas\-tě\-ji
  nazývaných {\it master} a {\it slave}. Master server nebo pouze master je
  server, který poskytuje data k replikaci, má práva na čtení i zápis a probíhají
  tedy na něm veškeré aktualizace. Je možno se setkat také s pojmenováním primary
  server, provider, sender, parent nebo source server. Naprosto jiný pojem zavádí
  MS SQL Server, který tento zdrojový server nazývá publisher (česky vydavatel).
  Druhý databázový server je nejčastěji nazýván slave, standby, reciever, child
  nebo subsciber (česky odběratel). Poslední pojem je také používán MS SQL
  Serverem. Na tento server, který je dostupný vždy jen pro čtení dat, se data
  kopírují, není však možné na něj změny zapisovat přímo \cite{RiggsKrossing2010}.

  Podle počtu master a slave serverů v replikačním clusteru se rozlišuje, zda
  se jedná o jednosměrnou nebo obousměrnou replikaci. U tzv. {\it multimaster}
  replikace existuje v replikačním clusteru několik master serverů, tedy těch
  na které se změny zapisují přímo. To je praktické například ve chvíli, kdy je
  i samotných zápisů tolik, že jeden server tuto zátěž neunese. Zápisy z~jednotlivých master serverů se tedy
  nereplikují pouze na slave servery, ale také na všechny ostatní mastery.
  Tento způsob s sebou však nese značné komplikace, je potřeba řešit konflikty
  změn v rámci stejných záznamů, a je tudíž relativně náročný na údržbu. Tato
  práce se zabývá použitím druhé způsobu, tzv. {\it master-slave} replikace.
  Tato replikace používá vždy jen jeden master server v clusteru a dva a více
  slave servery. Kopie dat tedy probíhá jednosměrně, vždy z master na slave
  servery. Podle Bella a kol. (2010) mají moderní aplikace často více čtenářů
  než zapisovatelů, proto je zbytečné, aby se všichni čtenáři při\-po\-jo\-va\-li na
  stejnou databázi jako zapisovatelé a zpomalovali tím jejich práci
  \cite{BellKindahlThalmann2010}.

    %parametr H říká že to bude přímo na tom místě kde je v textu...více http://en.wikibooks.org/wiki/LaTeX/Floats,_Figures_and_Captions
   %  \begin{figure}[H]
  \begin{center}
    \includegraphics[width=\textwidth]{obr/schema_masterMasterSlave.png}
       Srovnání multimaster a master-slave replikace
  \end{center}
   %    \label{srovnaniM-M-S}
   %  \end{figure}
  Při návrhu replikace je potřeba se zamyslet také nad tím, zda bude {\it
  synchronní} či {\it asynchronní}. Synchronní replikace neumožní potvrzení
  transakce modifikující data, dokud všechny změny nejsou přeneseny alespoň na
  jeden slave server \cite{Boszormenyi2013}. Tento přístup zajistí, že žádná
  data nebudou v průběhu zápisu ztracena. V~některých případech tento způsob
  může zbytečně zpomalit rychlost zápisu do databáze, protože je nutno čekat na
  dokončení zápisu na slave server. Zároveň může způsobit nemožnost zápisu do
  databáze v případě, že se přeruší spojení se slave serverem nastaveným pro
  synchronní replikaci. Tento způsob je využíván například při bankovních
  transakcích, kde je potřeba zajistit, aby všechny operace proběhly na obou
  stranách. V tomto případě je užití tohoto způsobu zcela nezbytné. 

  Druhým způsobem je asynchronní replikace, při které se nová data mohou
  zapisovat na master server, přestože ještě nedošlo k replikaci stávajících dat
  na slave server \cite{ObeHsu2012}. To je sice za běžného provozu rychlejší, v
  některý případech však může způsobit nekonzistenci dat, například když proběhne
  transakce na master serveru, který však spadne dřív, než se změna zapíše na
  slave. V takovém případě se slave změní na master server, ale zároveň se nikdy
  nedozví o transakci, o které má uživatel informace, že proběhla v pořádku. 

  % \begin{figure}[H]
  \begin{center}
    \includegraphics[width=\textwidth]{obr/schema_asyncSync.png}
     Rozdíl mezi synchronní a asynchronní replikací
  \end{center}
  % \end{figure}

  Dále je možno rozlišovat replikaci pole toho, zda je {\it logická} nebo {\it
  fyzická}. Při fyzické replikaci se kopírují na druhý server bloky binárních
  datových souborů bez znalosti jejich struktury (sloupce, řádky, …). Pro tento
  způsob kopírování dat je potřeba mít na obou serverech stejnou platformu a
  architekturu. Tento způsob je velice efektivní a často snazší na konfiguraci. 

  Naopak při logické replikaci se v přenášených datech přenáší samotný SQL
  příkaz, který se na slave serveru provede stejně jako na master serveru, nebo
  informace o tom, na kterých řádcích změny proběhly a jaké. Tento způsob je více
  flexibilní, umožňuje výběr jen několika databází nebo tabulek a není závislý na
  architektuře ani operačním systému \cite{Boszormenyi2013}. 

  %   \begin{figure}[H]
  \begin{center}
       \includegraphics[scale=1]{obr/schema_kaskadova.png}\\
       Ukázka kaskádové replikace
  \end{center}
  %     \label{kaskadova}
  %   \end{figure}
  Posledním diskutovaným pojmem je {\it kaskádová} replikace, která u\-mož\-ňu\-je
  připojit další slave k jinému slave serveru místo k hlavnímu master serveru.
  Kaskádovou replikaci lze využít v případě, že je třeba replikovat data na větší
  počet slave serverů v~clusteru. V případě, že by se všechny slave servery
  při\-po\-jo\-va\-ly k hlavnímu serveru, došlo by u něj k razantnímu snížení jeho
  výkonu. Kaskádová replikace může být praktická také v okamžiku, kdy se data
  přenáší na velkou vz\-dá\-le\-nost. V případě, kdy je třeba mít několik replik ve
  velké vzdálenosti od master serveru, je zbytečné, aby se obě kopie přenášely na
  tak velkou vz\-dá\-le\-nost, když druhý slave server lze připoji k~prvnímu.
  Kaskádovou replikace lze využít také při pádu master serveru, kdy jeden slave
  povýší na master a druhý je na něj již připojen, aby přijímal repliky.


  Každý  databázový systém (myšleno SŘDB) si volí terminologii a konkrétní
  nas\-ta\-ve\-ní mírně odlišně. Tato kapitola se snaží popsat chápání replikace
  v co největší míře obecně s ohledem na použití tohoto pojmu v PostgreSQL. Zcela
  jinou terminologii, i když založenou na stejných principech, zavádí MS SQL
  Server, který pro export databáze do souboru používá pojem snímková replikace,
  pro master-slave replikaci pojem transakční replikace a pro multimaster
  replikaci slučovací replikace. 



