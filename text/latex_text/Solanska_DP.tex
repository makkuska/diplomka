\documentclass{thesisKGI}

  %------------------- TITULNÍ STRANA ------------------- 

  \title{SYNCHRONIZACE A REPLIKACE GEODAT V PROSTŘEDÍ ESRI PLATFORMY}
  \author{Markéta SOLANSKÁ}
  \thesistype{Magisterská práce}
  \advisor{doc. RNDr. Vilém Pechanec, Ph.D.}

  \bibliographystyle{csplainnat} %styl citací


  \begin{document}
    \sloppy       %lepší hlídání přetékajících řádků
    \maketitle    %vložení titulní strany

    %------------------------------------------------------------------------- ČESTNÉ PROHLÁŠENÍ

    %vložení prohlášení, třída se sama postará o tvorbu nové stránky a vloží pod text řádek s datumem a jménem
    \begin{declaration}
      \textbf{Čestné prohlášení}

      Prohlašuji, že jsem magisterskou práci magisterského studia oboru Geoinformatika vypracovala samostatně pod vedením RNDr. Viléma Pechance, Ph.D.

      Všechny použité materiály a zdroje jsou citovány s ohledem na vědeckou etiku, autorská práva a zákony na ochranu duševního vlastnictví.

      Všechna poskytnutá i vytvořená digitální data nebudu bez souhlasu školy poskytovat.
    \end{declaration}

    %------------------------------------------------------------------------- PODĚKOVÁNÍ
    
    %poděkování, pokud nějaké chcete uvést, jinak lze tuto sekci smazat
    \begin{dedication}

      Děkuji vedoucímu práce doc. RNDr. Vilému Pechancovi, Ph.D. za podněty a připomínky při vypracování práce.

      Děkuji také konzultantu Tomáši Vondrovi za pomoc při pochopení a praktickém použití databázového serveru PostgreSQL, za jeho rady a podněty, stejně tak jako i jeho kolegovi Pavlovi Stěhule.

      Dále děkuji konzultantům Boudewijn van Leeuwen a Zalan Tobak z University of Szeged za připomínky a podněty k této práci. 
      \vspace{4em}
    \end{dedication}

    %------------------------------------------------------------------------- NASTAVENÍ POČÍTADLA, ZKRATKY
    
    \setcounter{page}{5}          %nastavení počítadla stránek na správnou hodnotu
    \makeTableOfContent{3}        %vložení obsahu, standardně se používají 3 úrovně

    \makeGlossary                 %formát v nemž se vkládají zkratky, vlouží se pouze ty, které budou použity v textu. v textu vkládáme zkratku pomocí příkazu \gls{DTM}
    %vložení seznamu zkratek, smazat, pokud není třeba
      \newglossaryentry{CAD}{name=CAD, description={Computer Aided Design}}
      \newglossaryentry{GIT}{name=GIT, description={geoinformační technologie}}
      \newglossaryentry{SQL}{name=SQL, description={Structured Query Language}}

    %------------------------------------------------------------------------- ÚVOD
    %protože je Úvod nečíslovaný je potřeba ho manuálně vložit do obsahu
    \addcontentsline{toc}{section}{ÚVOD}
    \section*{Úvod}
      Dnešní trend je ukládat a ponechávat stále více dat pouze v digitální podobě. Mnoho dokumentů už se vůbec netiskne do papírové podoby, tím spíš pokud dnes existují elektronické podpisy, díky kterým je tištěná verze naprosto zbytečná. S přibývajícím počtem dat je však třeba řešit komplikace, které počítačová data přinášejí. Počítačoví experti řeší například otázky, kam ukládat tak velké množství dat, jak data efektivně aktualizovat, jak zabránit poškození dat ať už způsobených lidským faktorem či fyzickým poškozením hardware. V připadě, že se poškodí disk, můžeme často během okamžiku přijít o všechna data, někdy však pro ztrátu dat stačí pouze stisknout tlačítko na klávesnici. Určitě už se Vám nejednou stalo, že jste se nemohli přihlásit do svého účtu na internetu z důvodu přetížení serveru. I to je problém, který velké množství dat a velký počet uživatelů přináší. Jak tedy pracovat s těmito objemy, jak zabránit komplikacím, které mohou poškodit či zcela zničit celou dosavadní práci, a jak zrychlit celý proces práce s daty? 

Řešením velkého počtu výše uvedených problémů může být ukládaní dat do databáze a jejich následná replikace. Replikací je myšlena pokročilá funkce, která zajišťuje kopii dat na více serverů. Nabízí ji většina dnešních databázových serverů, zajišťuje větší robustnost databáze a vysokou dostupnost dat. Replikaci lze využít ve všech odvětvích, které pracují s daty. Výjimkou tedy není ani geoinformatika, která pracuje s velkými objemy dat, které navíc nesou informaci o geografické poloze. Právě reprezentace geografické polohy, skrze textový zápis souřadnice daných bodů, může způsobit razantní zvýšení velikosti dat. Například u webových dat se navím musí řešit častý přístup k databázi, protože každé posunutí výřezu či přiblížiní, resp. oddálení výřezu mapy, je samostatným dotazem, který musí kapacita serveru zvládat. Při představě, že si uživatel bude posouvat výřez mapy po 50m, může to způsobit velkou zátěž pro server. V tomto případě je potřeba řešit replikaci z důvodu rozložení zátěže. 

Z mého pohledu data středně velkého až velkého projektu je vhodnější ukládata do databáze než jiných formátů typu shapefile, Microsoft Access nebo obyčejného tabulkového procesoru. Nabízí nám to sofistikované uložení dat, snadné propojení jednotlivých vrstev, snadnou přenostitelnost dat, možnost relačního propojení dat nebo efektivní vyhledávání. Replikace samotná se poté využívá pro kopii dat a následnou aktualizaci změn, která v databázi nastanou. 

Replikaci ocení uživatelé pracující na společném projektu, distribuovaná pracoviště i společnosti s velkým množstvím důležitých dat, jejichž kopie je rozhodující pro jejich fungování. Dobrým příkladem využitelnosti replikace je také nový trend využívání offline aplikací v mobilních telefonech. Databáze se vždy replikuje do mobilního telefonu, kde může fungovat offline a vždy, když se klient připojit na internetovou síť, aplikace kontroluje zda není na serveru novější verze databáze a pokud je, zkopíruje pouze změny, které proběhly od posledního stahování. (Jako příklad z geoinformatického prostředí bych uvedla diplomovou práci Dalibora Janáka, který řeší replikaci databáze lezeckých cest do mobilní aplikace.) 

Databázové systémy nabízí širokou škálu nastavitelnosti, která umožňuje přizpůsobit replikaci danému řešení.


    %------------------------------------------------------------------------- CÍLE PRÁCE
    %každou kapitolu je třeba začít na nové stránce
    \newpage
    \section{CÍLE PRÁCE}
      Cílem diplomové práce je provést rešerši v oblasti dostupných replikačních řešení a na jejím základě prakticky otestovat proces synchronizace a replikace geodat, které je možnost v kombinaci s ArcGIS produkty.

V teoretické části práce budou podrobně definovány pojmy týkající se zálohování dat, především však synchronizace a replikace, dále deteilně rozebrána replikace ve všech možných variantách nastavení, tedy jednosměrná, dvousměrná, synchronní, asynchronní, kaskádová, logická a fyzická. Dále rozbor zahrne celé portfólio produktů od desktop řešení, přes možnosti ArcGIS serveru až po cloudový ArcGIS online.

V rešerší části budou diskutovány dva databázové server, SQL Server a PostgreSQL, oba podporované ArcGIS produkty a jejím základě pak vybrát jeden, na kterém pak proces replikace bude prakticky testován.

Praktická část se bude zabývát návrhem replikačního řešení, které zahrne požadavky a možnosti katedry a bude brát v úvahu její způsoby využívání databáze. Na základě rešerše pak bude vybráno replikační řešení, připraveno testovací prostředí na základě všech výše zmíněných kritérií a na konec i praktickému testování výše zmíněných procesů.

Postupnými opakovanými procesy budou sledovány dílčí parametry procesu (rychlost procesu, úplnost, chybovost, podporované formáty). 


    %------------------------------------------------------------------------- POUŽITÉ METODY A POSTUPY PRÁCE
    \newpage
    \section{POUŽITÉ METODY A POSTUPY PRÁCE}
      \section{Použité metody a programové komponenty}
Konfigurace replikace zahrnovala studium návodů jednotlivých nástrojů pro
replikaci, výběr vhodných programových komponent a jejich následné praktické
nastavení. To bylo testováno průběžně na několika počítačích. 

Jako databázový server byl zvolen {\it PostgreSQL} s plnou podporou pro správu
prostorových dat, která je zajištěna nádstavbou {\it PostGIS}. Pro replikaci byla
zvolena nativní {\it PostgreSQL streaming replikace} a externí nástroj {\it Slony-I}. Pro
efektivní využívání databáze byl dále vybrán externí nástroj pgpool, který
zajišťuje snížení zátěže jednotlivých serverů rovnoměrným rozkládáním dotazů od
klientů mezi jednotlivé databáze. 

Nástroj pro replikaci Slony-I byl testován na operačním systému Ubuntu
GNU/Li\-nux 12.4 a zároveň na operačním systému Windows XP. 

Nativní PostgreSQL streaming replikace byla testována pouze na operačním
systému Linux. Server geohydro.upol.cz byl poskytnut jako testovaní server pro
tuto práci. Na server byl nainstalován 32bitový operační systém Debian
GNU/Li\-nux 7.3, který byl vybrán kvůli jeho stabilitě a jevil se tedy pro server
jako vhodný. Tato verze ovšem umožnila instalaci pouze programů verzí
PostgreSQL 9.1, PostGIS 1.5 a pgpool 3.1. Vzhledem k tomu, že se nejedná o
nejnovější verze zmíněných produktů, byla replikace testována také na osobním
počítači ve verzích PostgreSQL 9.3, PostGIS 2.1 a pgpool 3.3. To umožnilo
nastudování dalších možností, které nové verze přináší a které byly zohledněny
v návrhu replikačního řešení. 

Pro testování byla používána ukázková prostorová data vytvořená pro účel této
práce a dále byla na server uložena datová sada ArcČR ve verzi 3.0.



    
    %------------------------------------------------------------------------- TEORETICKÁ VÝCHODISKA
    \newpage
    \section{TEORETICKÁ VÝCHODISKA}
    
    %------------------------------------------------------------------------- VÝSLEDKY
    \newpage
    \section{VÝSLEDKY}

    %------------------------------------------------------------------------- DISKUZE
    \newpage
    \section{DISKUZE}

    %------------------------------------------------------------------------- ZÁVĚR
    \newpage
    \section{ZÁVĚR}

    %------------------------------------------------------------------------- LITERATURA
    \newpage
    \addcontentsline{toc}{section}{LITERATURA}
    %\makeBibliography{literatura}
    \bibliography{literatura}

    %------------------------------------------------------------------------- ILUSTRACE
    \newpage
    \addcontentsline{toc}{section}{SEZNAM ILUSTRACÍ}
    \section*{SEZNAM ILUSTRACÍ/TABULEK}

    %------------------------------------------------------------------------- SUMMARY
    \begin{summary}
      There is summary of all aims, methods and results in this chapter.
      Summary is not only translation of chapter Závěr. There is more
      information from chapters Cíle, Výsledky and Diskuze. Number of
      pages of Summary chapter is two at least. The style is Normalni
      Summary. Language is set to Angličina(Velká Británie) for automatic
      spell check. Do not use language Angličtina(USA). 
    \end{summary}

    %------------------------------------------------------------------------- PŘÍLOHY
    \newpage
    \vspace*{180pt}
    \begin{center}
      {\Large\textbf{PŘÍLOHY}}
    \end{center}
    \vspace*{\fill}

    \newpage
    \addcontentsline{toc}{section}{PŘÍLOHY}
    \section*{SEZNAM PŘÍLOH}
    \textbf{Volné přílohy}

    Příloha 1 CD \newline
    \newline
    \textbf{Popis sktruktury CD}

      Adresáře a soubory:

        - složka se skripty

        - web - webová stránky jako doplněk k diplomové práci

        - Solanska\_dp.pdf - text diplomové práce

    \vspace*{\fill}

  \end{document}
