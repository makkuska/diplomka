\documentclass{thesisKGI}
\newcommand{\itab}[1]{\hspace{0em}\rlap{#1}}
\newcommand{\tab}[1]{\hspace{.2\textwidth}\rlap{#1}}
  %------------------- TITULNÍ STRANA ------------------- 

  \title{SYNCHRONIZACE A REPLIKACE GEODAT V PROSTŘEDÍ ESRI PLATFORMY}
  \author{Markéta SOLANSKÁ}
  \thesistype{Diplomová práce}
  \advisor{doc. RNDr. Vilém Pechanec, Ph.D.}

  \bibliographystyle{csplainnat} %styl citací


  \begin{document}
    \sloppy       %lepší hlídání přetékajících řádků
    \maketitle    %vložení titulní strany

    %------------------------------------------------------------------------- ČESTNÉ PROHLÁŠENÍ

    %vložení prohlášení, třída se sama postará o tvorbu nové stránky a vloží pod text řádek s datumem a jménem
    \begin{declaration}
      \textbf{Čestné prohlášení}

      Prohlašuji, že jsem závěrečnou práci magisterského studia oboru Geoinformatika vypracovala samostatně pod vedením RNDr. Viléma Pechance, Ph.D.

      Všechny použité materiály a zdroje jsou citovány s ohledem na vědeckou etiku, autorská práva a zákony na ochranu duševního vlastnictví.

      Všechna poskytnutá i vytvořená digitální data nebudu bez souhlasu školy poskytovat.
    \end{declaration}

    %------------------------------------------------------------------------- PODĚKOVÁNÍ
    
    %poděkování, pokud nějaké chcete uvést, jinak lze tuto sekci smazat
    \begin{dedication}

      Ráda bych poděkovala doc. RNDr. Vilému Pechancovi, Ph.D. za ochotné vedení této práce, pečlivé korekce a věcné připomínky.

      Děkuji také konzultantu Tomáši Vondrovi, za jeho cenné rady a odborný vhled, který vnesl do této práce, stejně tak jako i jeho kolegovi Pavlovi Stěhule.

      Dále děkuji konzultantům Boudewijn van Leeuwen a Zalan Tobak působích na Universitě v Szegedu v Maďarsku za inspirativní podněty při vypracování této práce. 
      \vspace{4em}
    \end{dedication}

    %------------------------------------------------------------------------- NASTAVENÍ POČÍTADLA, OBSAH

    \newpage
    \begin{center}
    \section*{zadání}
    \end{center}

    %------------------------------------------------------------------------- NASTAVENÍ POČÍTADLA, OBSAH
    
    \setcounter{page}{5}          %nastavení počítadla stránek na správnou hodnotu
    \makeTableOfContent{3}        %vložení obsahu, standardně se používají 3 úrovně

    %------------------------------------------------------------------------- ILUSTRACE
    \newpage
    \listoffigures
    \newpage
    \listoftables

    %------------------------------------------------------------------------- GLOSSARY
    %\makeGlossary               
    %vložení seznamu zkratek, smazat, pokud není třeba
    %formát v nemž se vkládají zkratky, vlouží se pouze ty, které budou použity v textu. 
    % v textu vkládáme zkratku pomocí příkazu \gls{DTM}
    \newglossaryentry{Esri}{name=Esri, description={Environmental Systems Research Institute}}
    \newglossaryentry{SDE}{name=SDE, description={Spatial Database Engine}}
    \newglossaryentry{SQL}{name=SQL, description={Structured Query Language}}

    %------------------------------------------------------------------------- ÚVOD
    %protože je Úvod nečíslovaný je potřeba ho manuálně vložit do obsahu
    \newpage
    \addcontentsline{toc}{section}{ÚVOD}
    \section*{ÚVOD}
      Dnešní trend je ukládat a ponechávat stále více dat pouze v digitální podobě. Mnoho dokumentů už se vůbec netiskne do papírové podoby, tím spíš pokud dnes existují elektronické podpisy, díky kterým je tištěná verze naprosto zbytečná. S přibývajícím počtem dat je však třeba řešit komplikace, které počítačová data přinášejí. Počítačoví experti řeší například otázky, kam ukládat tak velké množství dat, jak data efektivně aktualizovat, jak zabránit poškození dat ať už způsobených lidským faktorem či fyzickým poškozením hardware. V připadě, že se poškodí disk, můžeme často během okamžiku přijít o všechna data, někdy však pro ztrátu dat stačí pouze stisknout tlačítko na klávesnici. Určitě už se Vám nejednou stalo, že jste se nemohli přihlásit do svého účtu na internetu z důvodu přetížení serveru. I to je problém, který velké množství dat a velký počet uživatelů přináší. Jak tedy pracovat s těmito objemy, jak zabránit komplikacím, které mohou poškodit či zcela zničit celou dosavadní práci, a jak zrychlit celý proces práce s daty? 

Řešením velkého počtu výše uvedených problémů může být ukládaní dat do databáze a jejich následná replikace. Replikací je myšlena pokročilá funkce, která zajišťuje kopii dat na více serverů. Nabízí ji většina dnešních databázových serverů, zajišťuje větší robustnost databáze a vysokou dostupnost dat. Replikaci lze využít ve všech odvětvích, které pracují s daty. Výjimkou tedy není ani geoinformatika, která pracuje s velkými objemy dat, které navíc nesou informaci o geografické poloze. Právě reprezentace geografické polohy, skrze textový zápis souřadnice daných bodů, může způsobit razantní zvýšení velikosti dat. Například u webových dat se navím musí řešit častý přístup k databázi, protože každé posunutí výřezu či přiblížiní, resp. oddálení výřezu mapy, je samostatným dotazem, který musí kapacita serveru zvládat. Při představě, že si uživatel bude posouvat výřez mapy po 50m, může to způsobit velkou zátěž pro server. V tomto případě je potřeba řešit replikaci z důvodu rozložení zátěže. 

Z mého pohledu data středně velkého až velkého projektu je vhodnější ukládata do databáze než jiných formátů typu shapefile, Microsoft Access nebo obyčejného tabulkového procesoru. Nabízí nám to sofistikované uložení dat, snadné propojení jednotlivých vrstev, snadnou přenostitelnost dat, možnost relačního propojení dat nebo efektivní vyhledávání. Replikace samotná se poté využívá pro kopii dat a následnou aktualizaci změn, která v databázi nastanou. 

Replikaci ocení uživatelé pracující na společném projektu, distribuovaná pracoviště i společnosti s velkým množstvím důležitých dat, jejichž kopie je rozhodující pro jejich fungování. Dobrým příkladem využitelnosti replikace je také nový trend využívání offline aplikací v mobilních telefonech. Databáze se vždy replikuje do mobilního telefonu, kde může fungovat offline a vždy, když se klient připojit na internetovou síť, aplikace kontroluje zda není na serveru novější verze databáze a pokud je, zkopíruje pouze změny, které proběhly od posledního stahování. (Jako příklad z geoinformatického prostředí bych uvedla diplomovou práci Dalibora Janáka, který řeší replikaci databáze lezeckých cest do mobilní aplikace.) 

Databázové systémy nabízí širokou škálu nastavitelnosti, která umožňuje přizpůsobit replikaci danému řešení.


    %------------------------------------------------------------------------- CÍLE PRÁCE
    %každou kapitolu je třeba začít na nové stránce
    \newpage
    \section{CÍLE PRÁCE}
      Cílem diplomové práce je provést rešerši v oblasti dostupných replikačních řešení a na jejím základě prakticky otestovat proces synchronizace a replikace geodat, které je možnost v kombinaci s ArcGIS produkty.

V teoretické části práce budou podrobně definovány pojmy týkající se zálohování dat, především však synchronizace a replikace, dále deteilně rozebrána replikace ve všech možných variantách nastavení, tedy jednosměrná, dvousměrná, synchronní, asynchronní, kaskádová, logická a fyzická. Dále rozbor zahrne celé portfólio produktů od desktop řešení, přes možnosti ArcGIS serveru až po cloudový ArcGIS online.

V rešerší části budou diskutovány dva databázové server, SQL Server a PostgreSQL, oba podporované ArcGIS produkty a jejím základě pak vybrát jeden, na kterém pak proces replikace bude prakticky testován.

Praktická část se bude zabývát návrhem replikačního řešení, které zahrne požadavky a možnosti katedry a bude brát v úvahu její způsoby využívání databáze. Na základě rešerše pak bude vybráno replikační řešení, připraveno testovací prostředí na základě všech výše zmíněných kritérií a na konec i praktickému testování výše zmíněných procesů.

Postupnými opakovanými procesy budou sledovány dílčí parametry procesu (rychlost procesu, úplnost, chybovost, podporované formáty). 


    %------------------------------------------------------------------------- POUŽITÉ METODY A POSTUPY PRÁCE
    \newpage
    \section{POUŽITÉ METODY A POSTUPY PRÁCE}
      %\section{Použité metody a programové komponenty}
Konfigurace replikace zahrnovala studium návodů jednotlivých nástrojů pro
replikaci, výběr vhodných programových komponent a jejich následné praktické
nastavení. To bylo testováno průběžně na několika počítačích. 

Jako databázový server byl zvolen {\it PostgreSQL} s plnou podporou pro správu
prostorových dat, která je zajištěna nádstavbou {\it PostGIS}. Pro replikaci byla
zvolena nativní {\it PostgreSQL streaming replikace} a externí nástroj {\it Slony-I}. Pro
efektivní využívání databáze byl dále vybrán externí nástroj pgpool, který
zajišťuje snížení zátěže jednotlivých serverů rovnoměrným rozkládáním dotazů od
klientů mezi jednotlivé databáze. 

Nástroj pro replikaci Slony-I byl testován na operačním systému Ubuntu
GNU/Li\-nux 12.4 a zároveň na operačním systému Windows XP. 

Nativní PostgreSQL streaming replikace byla testována pouze na operačním
systému Linux. Server geohydro.upol.cz byl poskytnut jako testovaní server pro
tuto práci. Na server byl nainstalován 32bitový operační systém Debian
GNU/Li\-nux 7.3, který byl vybrán kvůli jeho stabilitě a jevil se tedy pro server
jako vhodný. Tato verze ovšem umožnila instalaci pouze programů verzí
PostgreSQL 9.1, PostGIS 1.5 a pgpool 3.1. Vzhledem k tomu, že se nejedná o
nejnovější verze zmíněných produktů, byla replikace testována také na osobním
počítači ve verzích PostgreSQL 9.3, PostGIS 2.1 a pgpool 3.3. To umožnilo
nastudování dalších možností, které nové verze přináší a které byly zohledněny
v návrhu replikačního řešení. 

Pro testování byla používána ukázková prostorová data vytvořená pro účel této
práce a dále byla na server uložena datová sada ArcČR ve verzi 3.0.



    
    %------------------------------------------------------------------------- TEORETICKÁ VÝCHODISKA
    \newpage
    \section{TEORETICKÁ VÝCHODISKA}
              Jak definuje \cite{Oppel2009}, databáze je soubor vzájemně propojených datových
        položek, které jsou spravovány jako jeden celek \citep{Oppel2009}. Databáze
        představuje entity, atributy a logické vztahy mezi entitami, často zvané
        relace. Jinými slovy, databáze obsahuje data, která logicky související
        \citep{Connolly2005}. Databáze umožňuje ukládání a editaci dat, rychlé
        vyhledávání a komplexní analýzu dat \citep{Momjian2001}. Systém řízení báze
        dat\footnote{V anglickém originále Database Management System (DBMS)} je
        počítačový software, který umožňuje uživatelům přistupovat k databázi,
        definovat, vytvářet a udržovat data \citep{Connolly2005}. Pro uložení dat
        malého projektu je samozřejmě možno použít i jiného formátu určeného pro
        ukládání dat, například tabulkového procesoru. Pro komplexní správu dat velkého
        projektu je však databáze více než vhodná. 

        Prostorová databáze, někdy také zvaná geodatabáze, není nic jiného než databáze
        přidaná o datový typ určený pro ukládání prostorové informace o prvku,
        prostorové indexy a sadu funkcí vhodných pro správu prostorových dat. Více
        informací o prostorových databázích v kapitole \odkazKapitola{PostgreSQL} PostgreSQL 9.x (PostGIS) a \odkazKapitola{MSSQL} MS SQL Server 2008.

        Z toho vyvstává otázka, co jsou prostorová data, také zvaná geodata. Z pohledu
        společnosti ESRI se jedná se prvky, které nesou informaci o geografické poloze,
        zakódovanou informaci o tvaru (bod, line, polygon) a popis geografického jevu.
        Tato geodata jsou uložená ve formátu, který je možno použít v geografickém
        informačním systému \citep{Esri2006}. Příkladem takového formátu může být
        vektorový Esri shapefile, Esri coverage, GML, KML, GeoJSON nebo rastrový Erdas
        Image a GeoTIFF. Dalším způsobem je již zmíněná databáze, do níž se vektorová
        data ukládají ve specifickém tvaru daném standardem OGC\footnote{OGC standardy
        jsou kontrolované konsorciem Open Geospatial Consortium, \newline{zdroj
        http://www.opengeospatial.org/ogc}} Simply Feature for SQL 1.2.1, který
        specifikuje způsob uložení dat v digitální podobě. Simple Features je založen
        na 2D geometrii s~možností lineární interpolace mezi lomovými body. To umožňuje
        vložení následujících prvků:

        \begin{itemize}
          \item bod - POINT(0 0)
          \item linie - LINESTRING(0 0, 1 1, 1 2)
          \item polygon - POLYGON ((0 0,4 0,4 4,0 4,0 0),(1 1, 2 1, 2 2, 1 2,1 1))
          \item série bodů - MULTIPOINT((0 0),(1 2))
          \item série linií - MULTILINESTRING((0 0,1 1,1 2),(2 3,3 2,5 4))
          \item geometrická kolekce, která může obsahovat různé geoprvky (body, linie i polygony) - GEOMETRYCOLLECTION(POINT(2 3),LINESTRING(2 3,3 4))\footnote{Zdroj http://postgis.net/docs/manual-2.1/using\_postgis\_dbmanagement.html\#RefObject}
        \end{itemize}

        První slovo specifikace určuje druh prvku (point, linestring, polygon, multipoint,~...), následují v závorce vypsané souřadnice lomových bodů. Za tím ještě může následovat volitelný parametr kód souřadnicového systému.

        Hodnoty lze dále vkládat přes Well-Known Binary (WKB) nebo Well-Known Text (WKT) reprezentaci. PostGIS funkce pro vkládání geometrie vypadá následovně:

        \begin{itemize}
          \item ST\_AsBinary(geometry) pro bitový zápis WKB
          \item ST\_AsText(geometry) pro WKT text
        \end{itemize}

        Příklad uložení linie do databáze s jedním lomovým bodem v souřadnicovém systému WGS84:
        \newline \newline
        \texttt{(LINESTRING(15.96 50.84, 17.29 49.64, 18.27 49.80), 4326)\hspace*{4.5em}(1)}

      \section{Vymezení pojmů}
\label{kVymezeniPojmu}

Databáze je strukturovaná kolekce dat, která slouží pro efektivní ukládání dat a~jejich zpětně čtení \cite{Oppel2009}. V relační databázi jsou data ukládána ve formě tabulek, tedy entit a atributů, které jsou vzájemně propojeny vazbami mezi entitami \cite{Connolly2005}. Toto logické uložení vazeb mezi tabulkami umožňuje efektivní manipulaci s daty, rychlé vyhledávání i komplexní analýzu \cite{Momjian2001}. 

Obvykle se rozlišují pojem databáze, který odkazuje na obecný koncept, a pojem databázový systém nebo přesněji {\it systém řízení báze dat} \footnote{angl. Database Management System (DBMS)}, což je konkrétním počítačovým program, který zajišťuje fyzické uložení dat. Moderní SŘBD jsou navrženy na principu klient/server, kdy databáze běží jako služba na pozadí a čeká na dotazy od klientů. Server uživatelům umožňuje skrze jazyk SQL přístupovat k databázi, vytvářet a aktualizovat data, stejně jak jako vyhledávat či analyzovat \cite{Connolly2005}.

{\it Prostorová databáze}, někdy také zvaná {\it geodatabáze}, není nic jiného než databáze obohacená o datový typ určený pro ukládání prostorové informace o~prvku, prostorové indexy a sadu funkcí vhodných pro správu prostorových dat. Dnes umožňují ukládat prostorová data například databázové systémy PostgreSQL 9.x, Microsoft SQL Server, Oracle Database, MySQL nebo SQLite. 

Pojmy replikace a synchronizace některé zdroje rozlišují, jiné je naopak po\-va\-žu\-jí za synonyma. Všechny zmíněné pojmy souvisí se zálohováním dat, tedy kopírovaním dat mezi dvěmi a více uložišti, a se liší konkrétním důvodem pro použití daného procesu. 

O synchronizaci souborů či datových složek je možno mluvit v případě, že existují dva datové zdroje, které je potřeba v daný okamžik sjednotit. Jde tedy o proces, který probíhá jednorázově a to většinou z důvodů potřeby porovnání dvou a více datových uložišť, které je potřeba dostat do totožného stavu. To může například přispět snazší spolupráci více uživatelů nad stejnými daty nebo pomoct uživateli, který pracuje na více počítačích. Proces může proběhnout jednou nebo opakovaně, ať už pravidelně či nepravidelně. U souborů se shodným názvem se porovnává čas posledního zápisu, velikost nebo obsah souboru, naopak soubory, u kterých není nalezena shoda, jsou jednoduše zkopírovány. 

Replikace je proces průběžný, který soustavně hlídá, zda ve zdrojových datech nedošlo ke změně, a pokud ano, dané změny zkopíruje na jiné datové uložiště. Často je tento proces používán právě ve spojitosti s databázemi, kdy jsou data kopírována z~důvodu snížení zátěže serveru, či zvýšení ochrany dat. Replikace je tedy často vyžadována z jiných důvodů než synchronizace, začíná s daty existujícími pouze na jednom uložišti a pro zajištění konzistence dat používá jiných technologií. Více se replikací zabývá kapitola \ref{kReplikace}.

Oba procesy je možno použít jednostranně, tedy kopírovat data pouze z~jednoho uložiště na druhé a nikoliv opačně, nebo oboustraně, kdy se data kopírují navzájem mezi sebou.


              \subsection{Replikace}
        \label{kReplikace}
Replikace je proces, u kterého jsou data a databázové objekty kopírované z jednoho databázového serveru na druhý a poté synchronizovány pro zachování souladu obou databází. Synchronizací v tomto případě myslíme kopírováním všech změn, které v databázi nastanou. Použitím databáze je možno data distribuovat na různě vzdálená místa nebo mezi mobilní uživatele v rámci počítačové sítě a internetu \citep{Microsoft2013}.

Mnohé moderní aplikace se musí zabývat velkým počtem přístupů do databáze, což může v některých případech způsobovat problémy. Buď je server přetížen počtem připojení a data tedy přicházejí k uživateli pomalu, nebo dokonce úplně vypadne. 

Mezi časté důvody použití databázové replikace tedy patří zajištění dostupnosti dat\footnote{angl. High Availability}, resp. snížení pravděpodobnosti, že data nebudou dostupná, což může být způsobeno již zmíněným výpadkem serveru nebo například fyzickou ztrátou dat \citep{ObeHsu2012}. Další důvodem je rozložení zátěže přístupů do databáze mezi více serverů, takže nebude docházet ke zpomalení výkonu hlavního serveru ani k situaci, že data nebudou dostupná kvůli jeho výpadku \citep{BellKindahlThalmann2010}. Databáze je často zálohovaná, například skriptem dump a i to může server zpomalit. Vhodným řešením je tedy nejdříve vytvořit kopii dat na jiný datový server a až poté proces zálohování spustit. 

Všechny databáze zapojené do procesu replikace jsou v odborné literatuře nazývané uzly, v angličtině node. Tyto uzly dohromady tvoří replikační cluster\footnote{Volně přeloženo skupina serveru zapojených do replikace}. Při správně nastavené replikaci, by v clusteru nikdy neměly být méně než 3 uzly. Může se totiž stát, že vypadne jeden ze dvou uzlů, čímž dojde, ikdyž jen na krátkou chvíli, k situaci, že data nebudou v daný okamžik zálohovaná. 

Uzly v replikačním clusteru mohou mít jednu ze dvou základních rolí, nejčastěji nazývaných Master a Slave. Master server nebo pouze Master je server, který poskytuje data k replikaci, má práva na čtení i zápis a probíhají tedy na něm veškeré aktualizace. Je možno se setkat také s pojmenováním Primary server, Provider, Sender, Parent nebo Source server. Naprosto jiný pojem zavádí MS SQL Server, který tento zdrojový server nazývá Publisher (česky Vydavatel). Druhý databázový server je nejčastěji nazýván Slave, Standby, Reciever, Child nebo Subsciber (česky Odběratel). Poslední pojem je také používán MS SQL Serverem. Na tento server, který je dostupný vždy jen pro čtení dat, se data a aktualizace kopírují, není však možné na něj změny zapisovat \citep{RiggsKrossing2010}.

        %parametr H říká že to bude přímo na tom místě kde je v textu...více http://en.wikibooks.org/wiki/LaTeX/Floats,_Figures_and_Captions
          \begin{figure}[H]
            \centering
            \includegraphics[scale=1]{../../../grafy/obr/schema_masterMasterSlave_maxiTence.png}
            \caption {Srovnání Master-Master a Master-Slave replikace}
            \label{srovnaniM-M-S}
          \end{figure}

Podle počtu Master a Slave serverů v replikačním clusteru, se rozlišuje zda se jedná o jednosměrnou nebo obousměrnou replikaci. Tzv. Master-Master replikace umožňuje zapisovat do všech uzlů v replikačním clusteru, což může být praktické například při použití databáze offline \odkazObrazek{srovnaniM-M-S}. Změny se tedy synchronizují mezi všemi databázovými uzly. Tento způsob však nese značné komplikace, je potřeba řešit konflikty změn ve stejných datech a je relativně náročný na údržbu. Tato práce se zabývá použitím druhé způsobu, tzv Master-Slave replikace. Tato replikace používá vždy jen jeden Master server v clusteru a dva a více Slave servery. Kopie dat tedy probíhá jednosměrně, vždy z Master na Slave servery. Podle Bella (2010) mají moderní aplikace často více čtenářů než zapisovatelů, proto je zbytečné, aby se všichni čtenáři připojovali na stejnou databázi jako zapisovatelé a zpomalovali tím jejich práci \citep{BellKindahlThalmann2010}. Z toho důvodu je tedy použití Master-Slave replikace více než vhodné.

Při návrhu replikace je potřeba zamyslet se také nad tím, zda bude synchronní či asynchronní. Synchronní replikace neumožní, aby na Master serveru proběhla nová transakce, dokud se poslední transakce úspěšně neprovede na Slave serveru \citep{Boszormenyi2013}. Tento přístup zajistí, že žádná data nebudou v průběhu transakce ztracena. V některých případech tento způsob může zbytečně zpomalit rychlost přístupu do databáze, protože je nutno čekat na každou nedokončenou transakci. Zároveň může způsobit snížení dostupnosti databáze, protože v případě, že se například přeruší spojení mezi servery, nemůže na masteru proběhnout žádný další dotaz. Ale jistě si najde své opodstatění například při bankovních transakcí, kde je potřeba, aby všechny operace proběhly na obou stranách. V tomto případě je užití tohoto způsobu zcela nezbytné. 

Druhým způsobem je asynchronní replikace, při které se nová data mohou zapisovat na Master server, přestože ještě nedošlo k replikaci stávajících dat na Slave server \citep{ObeHsu2012}. To je sice za běžného provozu rychlejší, v některý případech však může způsobit nekonzistenci dat, například když proběhne transakce na Master serveru, který však spadne dřív, než se změna zapíše na Slave. V takovém případě se Slave změní na Master server, ale zároveň se nikdy nedozví o transakci, o které má uživatel informace, že proběhla v pořádku. 

        \begin{figure}[H]
          \centering
          \includegraphics[scale=1]{../../../grafy/obr/schema_asyncSync_maxiTence.png}
          \caption {Rozdíl mezi synchronní a asanchronní replikací}
        \end{figure}

Replikace v PostgreSQL umožňuje plnou kopii dat z databáze i pouze výběr některých tabulek. Více o možnostech a způsobech nastavení replikace v kapitole \odkazKapitola{} a PRAKTICKÁ ČÁST :)
Dále je možno rozlišovat replikaci pole toho, zda je logická nebo fyzická. Výsledek obou typů má naprosto identický výsledek, přesto se mírně liší. 

Fyzická replikace kopíruje data na druhý server v binární podobě. Tím, že se kopírují celé složky dat, je na Slave serverech zajištěna identická replika. Protože se kopírují binární data, která mají jasně danou strukturu, je potřeba mít na obou serveru stejnou platformu a architekturu. Tento způsob je velice spolehlivý a často snazší na konfiguraci. Naopak logická přenáší SQL příkazy tak, jak byly použity na Master serveru a ty poté proběhnou na Slave serverech. Tím se nasimuluje průběh změn dat na hlavním serveru a zajistí se konzistence dat. Tento způsob je více flexibilní, umožňuje výběr jen několika databází nebo tabulek a není závislý na architektuře ani operačním systému \citep{Boszormenyi2013}. 

Posledním diskutovaným pojmem je kaskádová replikace, která umožňuje připojit repliku k jinému Slave serveru místo k hlavnímu Master serveru. Tento způsob může být výhodných předeším z těchto dvou důvodů. Řekněme, že se kaskádová replikace použivá při existenci většího počtu Slave serverů v clusteru, třeba sta. V případě, že by se všechny repliky připojovaly k hlavnímu serveru, došlo by u něj k razantnímu zpomalení jeho výkonu. Kaskádová replikace může být praktická také v okamžiku, kdy se data přenáší na velkou vzdálenost, třeba do Číny. V případě, že mají v Číně dvě repliky, je zcela zbytečné, aby se obě kopie přenášely na tak velkou vzdálenost, když druhá replika se může připojit k první a mít data s mnohem menším zpožděním.

          \begin{figure}[H]
            \centering
            \includegraphics[scale=1]{../../../grafy/obr/schema_kaskadova.png}
            \caption{Ukázka kaskádové replikace}
            \label{kaskadova}
          \end{figure}

Každý databázový server (myšleno SŘDB) si volí terminologii a konkrétní nastavení mírně odlišně. Tato kapitola se snaží popsat chápání replikace co v největší míře obecně s ohledem na použití tohoto pojmu v PostgreSQL. Zcela jinou terminologii, ikdyž založenou na stejných principech, zavádí MS SQL Server, který používá pojmy transakční replikace pro Master-Slave replikace a slučovací replikaci pro Master-Master replikaci. 



      \subsection{ArcGIS produkty}

V názvu práce se objevuje spojení Esri platforma, čímž jsou chápány produkty
ame\-ric\-ké společnosti Esri, založené v roce 1969 manželi Dangermondovými,
zabývájící se vývojem software zaměřeného na geografické informační
systémy\footnote{více informací na adrese
\url{http://www.esri.com/about-esri/history}}.

Z hlediska chápání Esri má GIS tři roviny. První je to GIS jako prostorová
databáze ukládající geografické informace, dále sada map zobrazující prvky na
zemském povrchu a vztahy mezi nimi a zároveň i software pro GIS jako sada
nástrojů pro odvozování nových informací ze stávajících. Esri tyto tři pohledy
na GIS propojuje v~software ArcGIS jakožto kompletní GIS, který se skládá z
katalogu (kolekce geografický datových sad), map a sady nástrojů pro
geografické analýzy.

Esri vytváří integrovanou sadu softwarových produktů ArcGIS, které poskytují
nástroje na kompletní správu geografických dat, a přizpůsobuje produkty různým
úrovním nasazení. Výběr produktu záleží na tom, zda zákazník požaduje jedno-
nebo víceuživatelský systém, zda se má jednat o stolní systém nebo server,
popř. zda má být dostupný prostřednictvím internetu. Nabízí také produkty
vhodné pro práci v~terénu \citep{Esri2006}.

%----------------------------------------------------------- Varianty a verze 
\subsubsection{Verze a varianty produktu}

Základními produkty\footnote{Názvy jednotlivých produktů použitých v tomto
odstavci jsou platné od verze ArcGIS 10.1. Starší verze ArcGIS používají jiné
názvy, jejichž přehled je možný na stránkách firmy ARCDATA Praha
\url{http://www.arcdata.cz/produkty-a-sluzby/software/arcgis/prejmenovani-arcgis/.}}
jsou stolní systémy ArcGIS for Desktop ve variantách Basic, Standard,
Advanced\footnote{zdroj
\url{http://www.esri.com/software/arcgis/about/gis-for-me}}, dále serverové
verze ArcGIS for Server (pro Linux a Windows) ve třech úrovních funkcionality
(Basic, Standard, Advanced) a dvou úrovních kapacity serveru (Workgroup a
Enterpise). Další produkt ArcGIS for Mobile, ve variantách ArcPad, ArcGIS for
Windows Mobile a ArcGIS for Smartphone and Tablet, je určený především pro
práci v terénu. A v neposlední řadě verze dostupná skrze internet ArcGIS
Online. K tomu všemu Esri přidává velké množství extenzí a dalších
verzí\footnote{kompletní seznam na oficiálních webových stránkách Esri
\url{http://www.esri.com/products} nebo
\url{http://www.arcdata.cz/produkty-a-sluzby/software/arcgis/}}.

  \begin{table}[H]
    \caption{Varianty programu ArcGIS platné od verze 10.1.}
    \label{verzeArcGIS}
    \begin{footnotesize}
      \begin{center}
        \rowcolors{1}{white}{lightgray}
        \begin{tabular}{|>{\centering} c |>{\centering}m{9.5em}  m{8.5em}  <{\centering} m{11em}  <{\centering}|}
          \hline
          {\bf \color{purpurova7}Produkt}	& \multicolumn{3}{c|}{\bf \color{purpurova7}Verze} \\
          \hline
          ArcGIS for Desktop & Basic & Standard & Advanced \\
            ArcGIS for Server &	Basic &	Standard &	Advanced \\
            ArcGIS for Mobile &	ArcGIS for Windows Mobile &	ArcPAD &	ArcGIS for Smartphone and Tablet \\
              ArcGIS Online   & & &	\\	
          \hline
        \end{tabular}
      \end{center}
    \end{footnotesize}
  \end{table}

Dle \cite{Law2008} je nativním formátem produktů ArcGIS geodatabáze a jsou
roz\-li\-šo\-vá\-ny tři druhy geodatabáze. Ani v jednom případě se však nejedná o
databázi v~pravém slova smyslu, tak jako ji chápame v kapitolách
\odkazKapitola{kPostgreSQL} a \odkazKapitola{MSSQL}. V každém případě však tyto
způsoby umožňují uložení a správu dat. U prvních dvou typů, personální
a~souborové geodatabáze, se data ukládají do jednoho binárního souboru, kde
jsou však ukládána ve stejné struktuře jako v plnohodnotném databázovém
serveru, tedy ve formě databáze s tabulkami. Do takového souboru můžeme uložit
více než jednu vrstvu, což je výrazný rozdíl oproti formátu Shapefile. Výhodou
je dále možnost uložení vztahů mezi datovými prvky, sofistikované dotazování a
v neposlední řadě i~snadná přenositelnost, protože se jedná vždy jen jeden
soubor obsahující všechny vrstvy. Oproti tomu Shapefile, který obsahuje jen
jednu vrstvu, je tvořen minimálně čtyřmi soubory. Oba tyto formáty podporují pouze
jednoho zapisujícího uživatele a mnoho uživetelů s právem čtení, nepodporují
dlouhé transakce ani verzování.

S touto prací nejvíce souvisí třetí typ nazývaný {\it geodatabáze
ArcSDE}. Nejedná se o geodatabázi, ale spíše o zprostředkovatele
komunikace mezi programem ArcGIS a databázovým server. Umožňuje
víceuživatelský přístup, verzování i replikaci \citep{Esri2006}. Tato
technologie využívá jako datové uložiště některý z již existujících
databázových serverů, např. níže popsané PostgreSQL nebo SQL server.
Touto technologií se dále zabývá samostatná podkapitola \odkazKapitola{kArcSDE}.

  \begin{table}[H]
    \caption{Přehled rozdílů personální a souborové geodatabáze}
    \label{verzeArcGIS}
    \begin{footnotesize}
      \centering
      \begin{center}
        \rowcolors{1}{white}{lightgray}
        \begin{tabular}{|>{\centering} m{10.2em} |>{\centering}m{10.2em}  m{10.2em}  <{\centering}|}
          \hline
          {\bf \color{purpurova7}databáze}	& {\bf \color{purpurova7}souborová .gdb\textsuperscript{1}} & {\bf \color{purpurova7}personální .mdb\textsuperscript{1}}\\
          \hline
          datové uložiště/ databázový server & lokální souborový systém &	MS Access \\
          licence & ArcGIS for Destop (všechny verze) & ArcGIS for Destop (všechny verze) \\
          operační systém & Windows (možná i jiné) & Windows \\
          požaduje ArcSDE & ne &	ne \\
          vlastní datový typ & ne &	ne \\
          víceuživatelská editace & ano, ale s limity &	ne \\
          počet editorů	&	1 pro každý dataset \newline nebo tabulku\textsuperscript{2} &	1\textsuperscript{2} \\
          počet čtenářů &	více než 1\textsuperscript{2} &	více než 1\textsuperscript{2} \\
    master server\textsuperscript{3} & ne\textsuperscript{1} &	ne\textsuperscript{1} \\
      slave server\textsuperscript{3} & ano &	ano \\
          \hline
          \multicolumn{3}{l}{\textsuperscript{1}\scriptsize{http://www.esri.com/software/arcgis/geodatabase/singlex-user-geodatabase}} \\
          \multicolumn{3}{l}{\textsuperscript{2}\scriptsize{http://help.arcgis.com/en/arcgisdesktop/10.0/help/index.html\#//003n00000007000000}} \\
          \multicolumn{3}{l}{\textsuperscript{3}\scriptsize{je možno použít jako master/slave server}} \\
        \end{tabular}
      \end{center}
    \end{footnotesize}
  \end{table}


%----------------------------------------------------------- ArcSDE
\subsubsection{ArcSDE Technology}
\label{kArcSDE}

ArcSDE\footnote{Spatial Database Engine} je technologie společnosti Esri pro
správu prostorových dat uložených v relačních databázových systémech. ArcSDE je
prostředník pro komunikaci mezi klientem a SQL databází. Je tedy součástí
třívrstvé architektury složené z databázového serveru, aplikačního serveru ArcSDE
a klientské části (ArcGIS for Desktop nebo Arc\-GIS for Server). ArcSDE převádí pořadavky klienta (čtení, zápis dat)
na posloupnost SQL příkazů, které pak směřuje na databázový systém. 

Jedná se o multiplatformní, otevřenou a interoperabilní technologii, která
umožňuje správu dat uložených v databázovém systému, současnou editaci jedné
databáze více uživateli, archivování dat, dlouhé transakce a zajišťuje
integritu dat \citep{Law2008}. ArcSDE zprostředkovává výhody databázového
systémy, který zajišťuje jednoduchý, formální model pro uložení a správu dat ve
formě tabulek, definici datových typů pro atributy prostorových dat, zpracování
dotazů, vyhledávání, analýzu dat, zabezpečení, zálohování nebo replikaci dat
\citep{Law2008}. ArcSDE podporuje databázové systémy Oracle, MS SQL Server,
PostgreSQL, IBM DB2 a IMB Informix \citep{Esri2013a}.


Poskytuje vlastní prostorový datový typ (St\_Geometry), který využívá jako své
nativní datové struktury standard konsorcia OGC Simple Features a ISO 19125
\citep{Law2008}. Je však možno použít i prostorovou knihovnu daného
databázového systému. Technologie ArcSDE poskytuje vysoký výkon a je přizpůsobena velkému počtu
uživatelů \citep{Esri2006}.

ArcSDE si vytváří vlastní databázové schéma, tedy databázi s jasně danou
strukturou, která obsahuje datové typy, prostrové funkce a indexy. Zároveň se
tabulky schématu užívají k ukládání dočasných změn v databázi v případě, že
databázi e\-di\-tu\-je více uživatelů najednou. Každý uživatel si vytvoří pracovní
verzi, kterou po dokočení úprav připojí ke stávajícím datům, viz verzování v
kapitole \ref{kVymezeniPojmu}.

Data uložená v databázi lze skrze ArcSDE připojit ke všem produktům ArcGIS for Desktop a ArcGIS for Server. Pro ArcGIS
for Mobile a ArcGIS Online je třeba vrstvy publikovat pomocí ArcGIS for Server. 

Existují tři úrovně ArcSDE databáze: desktop (ArcSDE Desktop), skupinová
(ArcSDE Workgroup) a podniková (ArcSDE Enterprise). Každá verze má jiné
parametry a umožňuje různou úroveň editace \odkazTabulka{tSde}.

  \begin{table}[H]
    \caption[Přehled verzí ArcSDE, jejich parametrů a možností]{Přehled verzí ArcSDE, jejich parametrů a možností}
      \label{tSde}
    \begin{footnotesize}
      \begin{center}
        \rowcolors{3}{lightgray}{white}
        \begin{tabular}{|>{\centering} m{9.5em} |>{\centering} m{9.5em} >{\centering} m{9.5em} m{9.5em}  <{\centering}|}
          \hline
          \multirow{2}{*}{{\bf \color{purpurova7}databáze}} & \multicolumn{3}{c|}{\bf \color{purpurova7}ArcSDE} \\
          & {\bf \color{purpurova7}Desktop\textsuperscript{1}} & {\bf \color{purpurova7}Workgroup\textsuperscript{1}} & {\bf \color{purpurova7}Enterprise\textsuperscript{1}}\\
          \hline
            databázový server & MS SQL Server Express & MS SQL Server Express &	PostgreSQL, Oracle, MS SQL Server a další \\
                        licence & ArcGIS for Destop &	ArcGIS for Server Workgroup	& ArcGIS for Server Enterprise \\
              operační systém & Windows & Windows & multiplatformní \\
                požaduje ArcSDE & ano & ano & ano \\
            vlastní datový typ & ne & ne & ano \\
      víceuživatelská editace & ne & ano & ano \\
                počet editorů	&	1 &	10 & bez limitu \\
              počet čtenářů & 3 & 10 &	bez limitu \\
                  master server\textsuperscript{2}  & ne & ne & ano \\
                    slave server\textsuperscript{2}  &	ano &	ano & ano \\
                    verzování & ano & ano & ano \\
          závislost na sítích & lokální síť & lokální síť, internet & lokální síť, internet \\
              velikostní limity & 10GB & 10GB & záleží na velikosti serveru \\
          \hline
          \multicolumn{4}{l}{\textsuperscript{1}\scriptsize{http://www.esri.com/software/arcgis/geodatabase/multi-user-geodatabase}}\\
          \multicolumn{4}{l}{\textsuperscript{2}\scriptsize{pozn. je-li možno použít jako master/slave server}} \\
        \end{tabular}
      \end{center}
    \end{footnotesize}
  \end{table}


Od verze ArcGIS 9.2 je ArcSDE Desktop spolu s databázovým systémem MS SQL Server
Express součástí licence produktů ArcGIS for Desktop Standard a Advanced.
Takovou databázi mohou současně používat 4 uživatelé, z toho jen jeden může
databázi editovat, jsou však omezeni velikostí databáze.

Součastí licence ArcGIS for Server Workgroup je ArcSDE Workgroup, která se liší
od verze Desktop především tím, že počet uživatelů, kteří mohou součastně
editovat nebo prohlížet databázi, je zvýšen na deset.

Nejvyšší úroveň, ArcSDE Enterprise, je možno získat s licencí ArcGIS for Server
Enterprise, která uživatelům přináší nejméně omezení. Mohou si vybrat z
několika komerčních i nekomerčních databázových systémů, počet uživatelů není
omezen, stejně jako velikost databáze.

Replikaci a synchronizaci dat umožňují pouze ArcSDE Enterprise a Workgroup
\citep{Esri2013b}. MS SQL Server Express je možno použít v replikačním clusteru
pouze jako slave server, viz kapitola \ref{MSSQL}. Vzhledem k tomu,
že proces replikace je implementován přímo do ArcObjects a ArcSDE, nezáleží na
konkrétním databázovém systému \citep{Law2008}.

      \subsection{Vybrané databázové systémy}
\label{kPouziteProstredky}

\subsubsection{PostgreSQL (+ PostGIS)}
        \label{kPostgreSQL}

PostgreSQL je objektově-relační databázový systém s otevřeným zdrojovým kódem dostupný na většině základních platforem. Je volně k dispozici pro použití, modifikaci a šíření způsobem, který si sami zvolíme. Jedná se o robustní, výkonný, bezpečný, kompatibilní a interoperabilní software s zákaznickou podporou. Vyhovuje standardům SQL od verze SQL 2008 a nabízí velké množství pokročilých funkcí. PostgreSQL je založen na architektuře klient-server, to znamená, že server pořád běží a čeká na dotazy klienta \citep{Momjian2001}. 

S vývojem databázového serveru PostgreSQL začala University of California v~Berkley před více než 20 lety. Nyní je vyvíjen a udržován velkou komunitou nezávislých vývojářů. Používá licenci TPL (The PostgreSQL Licence), která je mírně odlišná od open-source licence BSD (Berkeley Distribution Software), ze které vychází \citep{RiggsKrossing2010}.

Řadí se mezi nejpokročilejší databázové systémy. Díky schopnosti pracovat s velkými objemy dat, své rychlosti a bohaté funkcionalitě může soupeřit i s populárními komerčními systémy jako jsou Oracle Database, MySQL a MS SQL Server \citep{PostgreSQL2012}.

Samotné PostgreSQL neobsahuje datové typy ani funkce vhodné pro správu prostorových dat. K tomu je nutné přidat nadstavbu PostGIS, která implementuje specifikaci {\it Simple Features for SQL} konsorcia OGC a rozšiřuje tak databázový systém PostgreSQL o~podporu geografických dat. PostGIS umožňuje ukládání geometrických objektů (bod, linie, polygon), použití prostorových funkcí pro určení vzdálenosti, délky linií, výměru a obvodu ploch a výběr prostorových indexů.

PostGIS umožňuje práci s rozšířenými XML formáty GML, KML, GeoJSON~a SVG, jejichž funkce pro získání geometrie jsou:
\begin{itemize}
\item \texttt{ST\_AsGML(geometry)},
\item \texttt{ST\_AsKML(geometry)},
\item \texttt{ST\_AsGeoJSON(geometry)} a 
\item \texttt{ST\_AsSVG(geometry)}.
\end{itemize}

PostGIS používá dva základní prostorové datové typy {\it geography} a {\it geometry}. Typ geography ukládá polohu v kartézských rovinných souřadnicích, kterým odpovídá souřadnicový systém WGS84. Je vhodný zejména pro malá území, protože při výpočtu vzdálenosti dvou bodů uložených v tomto datovém typu, funkce vrátí jako výsledek nejkratší vzdálenost v kilometrech v rovině. Typ geometry data ukládá v polárním rovinném systému a umožňuje nastavit souřadnicový systém dle potřeb. Výsledkem dotazu na vzdálenost dvou bodů bude úhel ve stupních, který po přepočtu do metrické soustavy určí nejkratší vzdálenost na povrchu kouli. Při výběru datového typu může být rozhodující například velikost daného území, nebo počet funkcí, jichž pro typ geometry poskytuje PostGIS mnohem více než pro typ geography \citep{OpenGeo2012}.

Existuje také další nadstavba PostGIS Raster, která rozšiřuje PostgreSQL o možnost ukládání a manipulace s rastrovými daty, nadstavba PostGIS Topology pro topologickou správu vektorových dat a nadstavba pgRouting pro síťové analýzy. PostGIS je podporován velkou řadou softwarových produktů zabývajících se správou geografických dat, což také umožňuje snadnou přenositelnost a použitelnost jednotlivých nadstaveb (příklad software podporujících PostGIS: QGIS, GvSIG, GRASS, ArcGIS).

PostGIS využívá mnoho běžně používaných knihoven jako GEOS (Geometry Engine Open Source) pro implementaci jednoduchých prostorových prvků a metod pro topologii, PROJ4 pro převod mezi kartografickými projekcemi nebo GDAL/OGR (Geospatial Data Abstraction Library) pro převod mezi různými vektorovými i rastrovými formáty \citep{ObeHsu2011}. Nadstavba PostGIS 1.5. obsahovala přes 800 funkcí, typů a prostorových indexů \citep{ObeHsu2012}. Aktuální verze PostGIS\footnote{aktuálně na http://postgis.refractions.net/} je 2.1.

Od verze ArcGIS 9.3. je PostgreSQL oficiálně podporovaným databázovým systémem pro ukládání geodat v produktech ArcGIS. Při instalaci je potřeba zajistit kompatibilitu verzí jednotlivých nástrojů, \vizTabulka{tKompatibilita}. Pro verzi ArcGIS 10.1 jsou podporované verze PostgreSQL 9.0 a PostGIS 1.5., pro ArcGIS 10.1 SP1\footnote{Service Pack 1} lze použít novější PostgreSQL 9.1 a PostGIS 2.0 \citep{OSGEO2013}\footnote{zdroj a další informace na stránkách PostgreSQL \url{http://trac.osgeo.org/postgis/wiki/UsersWikiPostgisarcgis} nebo ArcGIS Resources \url{http://resources.arcgis.com/en/help/system-requirements/10.1/index.html\#//015100000075000000}}. Na stránkách ArcGIS Resources\footnote{\url{http://resources.arcgis.com/en/help/system-requirements/10.1/index.html\#//015100000075000000}} jsou dále popsána další doporučení, například že je podporovaná pouze 64-bitová verze PostgreSQL. 

Databázi PostgreSQL lze v ArcGIS produktech použít dvojím způsobem. Buď jen jako uložiště dat bez přidání geografického datového typu, nebo včetně datového typu, tedy včetně PostGIS knihovny. ArcSDE podporuje pouze datový typ PostGIS Geometry a přidává vlastní datový typ Esri St\_Geometry. Výhodou použivání Esri St\_Geometry je nezávislost na zvoleném databázovém systému, tedy snazší přenostitelnost celého řešení. 

        \begin{table}[H]
\caption{Možné kombinace verzí PostgreSQL (+ PostGIS) a ArcGIS }
          \label{tKompatibilita}
          \begin{footnotesize}
            \begin{center}
              \rowcolors{1}{white}{lightgray}
              \begin{tabular}{|cccc|}
                \hline 
                {\color{purpurova7}PostgreSQL} & {\color{purpurova7} PostGIS} & {\color{purpurova7}ArcGIS} & {\color{purpurova7}podporovaná architektura} \\ 
                \hline 
                9.3 & \multicolumn{3}{>{\cellcolor{lightgray}}c|}{PostgreSQL 9.3 není zatím podporováno produkty ArcGIS} \\ 
                    9.1 (64-bit) & 2.0 (64-bit) & 10.1 SP1\textsuperscript{1} & Linux 64-bit (x86\_64), Windows 64-bit \\ 
                    9.0 (64-bit) & 1.5\textsuperscript{2} (64-bit) & 10.1 SP1\textsuperscript{1} & Linux 64-bit (x86\_64), Windows 64-bit \\ 
                    9.0 (64-bit) & 1.5\textsuperscript{2} (64-bit) & 10.1 & Linux 64-bit (x86\_64), Windows 64-bit \\ 
                         8.3/8.4 & 1.4 & 10.0 & Linux 64-bit (x86\_64), Windows 64-bit \\ 
                \hline 
                \multicolumn{4}{l}{{\cellcolor{white}}\scriptsize{\textsuperscript{1}Service Pack 1}} \\ 
                \multicolumn{4}{l}{\scriptsize{\textsuperscript{2}není podporováno ve verzi Windows 64-bit}} \\ 
                \multicolumn{4}{l}{{\cellcolor{white}}\scriptsize{zdroj: http://support.esri.com/en/knowledgebase/techarticles/detail/40553}} \\ 
              \end{tabular}
            \end{center}
          \end{footnotesize}
        \end{table}

        \subsubsection{Microsoft SQL Server}
        \label{MSSQL}
Microsoft SQL Server (dále MS SQL Server) je relační databázový systém, vyvíjený společností Microsoft, dostupný pro různé verze operačního systému Windows. Dodává se v mnoha verzích, které lze nainstalovat na různé hadrwarové platformy na základě odlišných licenčních modelů \citep{Whalen2008}. Podle Leitera (2009) MS SQL Server nabízí 8~základních verzí: Enterprise, Standard, Workgroup, Web, Express, Express Advanced Edition, Developer Edition a Compact Edition. Enterprise edition podporuje naprosto vše, co MS SQL Server nabízí, naopak verze Express, která je dostupná zdarma, obsahuje omezení některých funkcí a proto je vhodná spíše pro malé nebo začínající projekty \citep{Leiter2009}.

Podpora prostorových dat je implementována jako CLR rozšíření a přidává databázovému serveru dva prostorové datové typy geometry a geography, jejichž rozdíl je podobný jako u PostgreSQL. První jmenovaný slouží k reprezentaci geografický prvků (bodů, linií, polygonů) v rovině, naproti tomu datový typ geography slouží ukládání těchto prvků na povrchu země. Oba typy pracují ve dvou dimenzích, nebere se tedy v potaz výška. Podporuje také indexování dat, které implementováno standardním B~stromem \citep{Cincura2009}. MS SQL Server podporuje OGC standardy pro prostorová data.

MS SQL Server je podporován a používán ArcGIS produkty od začátku jeho vývoje\footnote{pro přehled kompatibilních verzí ArcGIS a MS SQL Server viz \url{http://resources.arcgis.com/en/help/system-requirements/10.1/index.html\#/Microsoft\_SQL\_Server\_Database\_Requirements/015100000070000000/}}. Verze ArcGIS Enterprise může být propojena s jakoukoliv uživatelem zvolenou a zakoupenou licencí databázového systému. Verze ArcSDE Desktop a Workgroup používají verzi Express, která je dostupná zdarma a podporuje většinu základních funkcí. Replikaci plně podporuje jen verze Enterprise, ostatní verze ji podporují pouze s omezenými funkcemi. Avšak již zmiňovaná verze Express může být použita pouze jako slave server, tedy odběratel replikovaných dat, do takovéto databáze tedy není možné přímo zapisovat \citep{Whalen2008}. Stejně jako u PostgreSQL platí, že si uživatel může zvolit, zda použije datový typ, který je součastí ArcSDE, nebo ten, který je poskytován MS SQL Serverem.

%       \subsubsection{ArcSDE geodatabase}
%       \label{kArcSDE}
% ArcSDE je technologie firmy Esri pro správu geoprostorových dat uložených v relačních databázových systémem. Jedná se o otevřenou a interoperabilní technologii, která podporuje čtení a zápis mnoha standardů. Využívá jako své nativní datové struktury standard konsorcia OGC Simple Feature a prostorový typ ISO pro databázové systémy Oracle, IBM DB2 a Informix. Poskytuje vysoký výkon a je přizpůsobena velkému počtu uživatelů \citep{Esri2006}.

% ArcSDE je prostředník pro komunikaci mezi klientem (např. ArcView) a SQL databází (př. PostgreSQL). Umožňuje přístup a správu dat v databázi, současnou editaci jedné databáze více uživateli, archivování  dat, , dlouhé transakce, zajišťuje integritu a poskytuje vlastní prostorový datový typ (St\_Geometry) \citep{Law2008}. 

% ArcSDE je prostředník pro komunikaci mezi klientem (např. ArcView) a SQL databází (př. PostgreSQL). Umožňuje přístup a správu dat v databázi, současnou editaci jedné databáze více uživateli, archivování  dat, , dlouhé transakce, zajišťuje integritu a poskytuje vlastní prostorový datový typ (St\_Geometry) \citep{Law2008}. 
% Vytváří vlastní databázové schéma, tedy databázi s jasně danou strukturou, která obsahuje jednotlivé funkce, datové typy a indexy. Zároveň se schéma využívá na ukládání dočasných změn v databázi v případě, že databázi edituje více uživatelů najednou. Každý uživatel si vytvoří pracovní verzi, kterou po dokočení úprav připojí ke stávajícím datům.

% Technologie ArcSDE vyžaduje dvě úrovně: databázovou a aplikační, která se skládá z ArcObjects a ArcSDE. Databázová úroveň zajišťuje jednoduchý, formální model pro uložení a správu dat ve formě tabulek, definici typů atributů (datových typů), zpracování dotazů či víceuživatelské transakce \citep{Law2008}. ArcSDE podporuje databázové systémy IBM DB2, IMB Informix, Oracle, Microsoft SQL, PostgreSQL \citep{Esri2013a}.

% Existují tři úrovně ArcSDE databáze: desktop (ArcSDE Desktop), skupinová (ArcSDE Workgroup) a podniková (ArcSDE Enterprise). Každá verze má jiné parametry a umožňuje různou úroveň editace \odkazTabulka{tSde}.

%       \begin{table}[H]
%         \caption[Přehled verzí ArcSDE, jejich parametrů a možností]{Přehled verzí ArcSDE, jejich parametrů a možností}
%           \label{tSde}
%         \begin{footnotesize}
%           \begin{center}
%             \rowcolors{3}{lightgray}{white}
%             \begin{tabular}{|>{\centering} m{9.5em} |>{\centering} m{9.5em} >{\centering} m{9.5em} m{9.5em}  <{\centering}|}
%               \hline
%               \multirow{2}{*}{{\bf \color{purpurova7}databáze}} & \multicolumn{3}{c|}{\bf \color{purpurova7}ArcSDE} \\
%               & {\bf \color{purpurova7}Desktop\textsuperscript{1}} & {\bf \color{purpurova7}Workgroup\textsuperscript{1}} & {\bf \color{purpurova7}Enterprise\textsuperscript{1}}\\
%               \hline
%                 databázový server & SQL Server Express & SQL Server Express &	PostgreSQL, Oracle, SQL Server a další \\
%                             licence & ArcGIS for Destop &	ArcGIS for Server Workgroup	& ArcGIS for Server Enterprise \\
%                  operační systém & Windows & Windows & multiplatformní \\
%                    požaduje ArcSDE & ano & ano & ano \\
%                vlastní datový typ & ne & ne & ano \\
%          víceuživatelská editace & ne & ano & ano \\
%                     počet editorů	&	1 &	10 & bez limitu \\
%                  počet čtenářů & 3 & 10 &	bez limitu \\
%                       master server\textsuperscript{2}  & ne & ne & ano \\
%                        slave server\textsuperscript{2}  &	ano &	ano & ano \\
%                         verzování & ano & ano & ano \\
%              závislost na sítích & lokální síť & lokální síť, internet & lokální síť, internet \\
%                  velikostní limity & 10GB & 10GB & záleží na velikosti serveru \\
%              \hline
%              \multicolumn{4}{l}{\textsuperscript{1}\scriptsize{http://www.esri.com/software/arcgis/geodatabase/multi-user-geodatabase}}\\
%              \multicolumn{4}{l}{\textsuperscript{2}\scriptsize{pozn. je-li možno použít jako master/slave server}} \\
%             \end{tabular}
%           \end{center}
%         \end{footnotesize}
%       \end{table}


%       Od verze ArcGIS 9.2 je ArcSDE Desktop spolu s databázovým systémem SQL Server Express součástí licence produktů ArcGIS for Desktop Standard a Advanced. Takovou databázi mohou současně používat 4 uživatelé, z toho jen jeden může databázi editovat, jsou však omezeni velikostí databáze.

% Součastí licence ArcGIS for Server Workgroup je ArcSDE Workgroup, která se liší od verze Desktop především tím, že počet uživatelů, kteří mohou součastně editovat nebo prohlížet databázi, je zvýšen na deset.

% Nejvyšší úroveň, ArcSDE Enterprise, je možno získat s licencí ArcGIS for Server Enterprise, která uživatelům přináší nejméně omezení. Mohou si vybrat z několika komerčních i nekomerčních databázových systémů, počet uživatelů není omezen, stejně jako velikost databáze.

% Replikaci a synchronizaci dat umožňují pouze ArcSDE Enterprise a Workgroup \citep{Esri2013b}. Jak už bylo zmíněno v předchozí kapitole \odkazKapitola{MSSQL} Microsoft SQL Server Express 2008, SQL Server Express je možný použít v replikačním clusteru pouze jako slave server. Vzhledem k tomu, že proces replikace je implementován přímo do ArcObjects a ArcSDE, nezáleží na konkrétním databázovém systému \citep{Law2008}.

            \subsection{Nástroje pro replikaci v PostgreSQL}

      PostgreSQL nabízí hned několik nástrojů pro replikaci. Je možno použít zabudovanou streaming replikaci, která je dostupná od verze PostgreSQL 9.0, nebo některou z extenzí, například Slony-I, pgpool, Londiste, Bucardo nebo Postgres-XC, pro jejichž srovnání \vizTabulka{tSrovnaniReplikace}. Tato kapitola se dále bude zabývat nativní streaming replikací a extenzemi Slony-I a pgpool.

        \begin{table}[H]
          \caption{Srovnání různých typů dostupných replikačních řešení}
          \label{tSrovnaniReplikace}
          \begin{footnotesize}
            \begin{center}
              \rowcolors{1}{white}{lightgray}
              \begin{tabular}{|c|cccccc|}
                \hline
                {\bf \color{purpurova7}nástroje}	& {\bf \color{purpurova7}typ} & {\bf \color{purpurova7}technika} & {\bf \color{purpurova7}M/M} & {\bf \color{purpurova7}M/S} & {\bf \color{purpurova7}sync} & a{\bf \color{purpurova7}sync} \\
                \hline
                PostgreSQL 9.1* & fyzická & xlog & ne & ano & ano & ano \\
                     pgpool-II & logická & proxy & ano & ne & ano & ne \\
                       slony-I & logická & triggers & ne & ano & ne & ano \\
                      Londiste & logická & triggers & ne & ano & ne & ano \\
                       Bucardo & logická & triggers & ano & ano & ne & ano \\
                   Postgres-XC & cluster & - & ano & ne & ne & ano \\
                \hline
                \multicolumn{3}{l}{\scriptsize{*streaming replikace}} & \multicolumn{4}{r}{\scriptsize{zdroj: Tomáš Vondra, 2011}}\\
              \end{tabular}
            \end{center}
          \end{footnotesize}
        \end{table}

      \subsubsection{Slony-I}
      \label{kSlony}

      Jak píší \cite{Boszormenyi2013} je Slony-I jeden z nejrozšířenějších externích nástrojů pro replikaci PostgreSQL databází. Zároveň se také řadí mezi nejstarší, plně používán je v PostgreSQL již od verze 7.3. a je velmi dobře podporován i dalšími externími řešeními pro PostgreSQL, například programem PgAdmin3, který nabízí správu dat pomocí grafického rozhraní \citep{Boszormenyi2013}.

      Jedná se o trigger-based replikaci, což znamená, že je ke každé tabulce vybrané pro replikaci přidán trigger, který zajistí replikaci každé změny, která v tabulce nastane. Z toho také vyplývá, že se jedná o logickou replikaci, kdy je možné replikovat pouze změny v datech, tedy tzv. DML\footnote{Data Manipulation Language} příkazy (INSERT, UPDATE, DELETE), nikoli změny struktury databáze, tedy tzv. DDL\footnote{Data Definition Language} příkazy (CREATE, ALTER, DROP). Každá změna struktury se tedy musí provést ručně, což se může jevit jako nevýhodné. Nese to ale i své klady, například možnost výběru pouze některých tabulek. Uživatel vytváří tzv. {\it replikační set}, do kterého se zapíší pouze ty tabulky, které je potřeba replikovat. 

Další výhodou, a to zvlášť v porování se streaming replikací, je možnost replikace dat mezi různými verzemi PostgreSQL bez ohledu na platformu a architekturu. Naopak spíše za nevýhodu je považováno, že si vytváři ke každé tabulce vlastní schéma, do kterého se ukládají replikovaná data, což způsobuje redundanci dat. 

Slony-I replikace je z principu asynchronní, umožňuje kaskádovou replikaci i Hot Standby mód, což znamená, že v případě pádu master serveru je slave automaticky povýšen na master. Slony-I má vlastní konfigurační nástroj a samotná replikace funguje díky vlastnímu replikačnímu démonu, který běží stále, registruje změny a kopíruje je na slave servery.

      \subsubsection{Streaming replikace}
      \label{kStreamingTeorie}

      Streaming replikace je nativní řešení PostgreSQL implementované od verze 9.0. Jedná se o fyzickou replikaci, proto je nutné použití stejné verze PostgreSQL, stejné platformy i architektury na všech uzlech replikačního clusteru. Jedná se o log-shipping replikaci, což znamená, že jsou na slave servery posílány záznamy transakčního logu, v PostgreSQL nazývané WAL (Write Ahead Log). Do něj jsou změny nejdříve zaznamenávány přímým zápisem na disk a až poté potvzeny jako úspěšné. Tento způsob zajišťuje datům naprosté bezpečí, protože kdyby došlo k chybě a změny se nezapisovaly na disk, ale pouze do cache, mohlo by dojít k jejich ztrátě. Zároveň to zajišťuje kopii jak dat, tak i struktury databáze. Existuje pouze jeden transakční log pro jednu instalaci PostgreSQL, proto se replikují vždy všechny databáze a není možné výběru jen několika tabulek, tak jako u Slony-I \citep{Boszormenyi2013}. . 
 
      {\color{purpurova7}Výhodou tohoto řešení je která jakožto nativní řešení PostgreSQL nabízí větší efektivitu a stabilitu, než jiná diskutovaná řešení. Tento výběr má i tu nevýhodu, že servery je nutno aktualizovat 'všechny najednou, co se týče verze software, operačního systému i architektury - výhodou je, nevýhodou }

Streaming replikace umožňuje jak synchronní, tak asynchronní replikaci, dále Hot standby mode i kaskádovou replikaci.

      \subsubsection{pgpool}
      \label{kpgpool}

      Nástroj pgpool, který je stejně jako Slony-I extenzí pro PostgreSQL, je dalším z nástrojů, který je možno použít pro replikaci dat, umožňuje však i další pokročilé funkce jakými jsou sdílení spojení klienta s databází mezi servery (angl. connection pooling), paralelní uložení dat (angl. parallel query) a rozložení zátěže mezi více servery (angl. load balancing). 
Nástroj pgpool umožňuje sdílení spojení klienta s databází, což v praxi znamená, že se vytvoří několik spojení se serverem, která i po skončení dotazu zůstanou otevřená a připravená pro další použití. Nemusí se tedy navazovat spojení při každém požadavku ze strany klienta, což velice zrychlí provoz a zajistí plynulost užívání databáze. Je vhodným nástrojem pro správu velkých tabulek díky distribuovanému způsobu ukládání dat \citep{pgpool2014}. 

Zároveň pgpool umožňuje rozložení zátěže mezi více serverů v replikačním clusteru, aby nedocházelo k přetížení jednotlivých uzlů a celkově se zvýšila rychlost a~efektivita práce s databází. V tomto se pgpool stává prostředníkem pro komunikaci mezi klientem a serverem. Aby nebylo potřeba dát každému uživateli přístup k jinému slave serveru, nebo přístupy do databáze manuálně rozkládat skrze složité programové řešení, nabízí se možnost použití pgpool, který se navenek jeví jako jakákoliv jiný databázový server, do kterému se uživatelé připojí. pgpool pak sám rozdělí dotazy mezi uzly v replikačním clusteru dle aktuální zátěže \citep{Boszormenyi2013}. Zároveň, pokud má uživatel práva ke čtení i k zápisu, umí na základě aktuálního SQL příkazu rozhodnout, zda jej přepošle master nebo slave serveru \odkazObrazek{opgpool}. 

Na základě vybraných funkcí je možno použít jeden ze čtyř základních módů, které pgpool poskytuje\footnote{kompletní přehled na \url{http://www.pgpool.net/docs/latest/pgpool-en.html\#config}}: základní, replikační, master/slave a paralelní. V návrh databázového řešení byl použit mód master/slave, který je dále popisován v kapitole \odkazKapitola{kpgpool}.

      \begin{figure}[H]
        \centering
        \includegraphics[scale=1]{../../../grafy/obr/schema_pgpool.png}
        \caption{Zjednodušené schéma pgpool v módu master/slave}
        \label{opgpool}
      \end{figure}


    
    %------------------------------------------------------------------------- VÝSLEDKY
    \newpage
    \section{NÁVRH A KONFIGURACE REPLIKACE}
      \subsection{Návrh replikačního řešení}
\label{kNavrh}

Po provedení rešeršní části a zohlednění všech podmínek, požadavků a možností katedry, byl sestaven následující návrh pro kompletní databázové řešení založené na procesu replikaci. Z databázových serverů, diskutovaných v kapitole \odkazKapitola{kPouziteProstredky}, byl vybrán server PostgreSQL hned z několika důvodů. Jedná se o plnohodnotný databázový systém dostupný zdarma se všemi nástroji, je multiplatfomní a od verze ArcGIS 9.3 je plně podporován produkty ArcGIS. 

Byl navržen replikační cluster s nejméně třemi servery z důvodů, které již byly diskutovány v kapitole \odkazKapitola{kReplikace}. Celý cluster poběží na stejné platformě a proto bude možno použít streaming replikaci, která jakožto nativní řešení PostgreSQL, nabízí větší stabilitu a bezpečnost, díky přenosu transakčních logů, než jiná diskutovaná řešení. Byla zvolena jednosměrná master-slave replikace, cluster tedy bude obsahovat jeden master a dva (popř. více) slave serverů. Aby nedošlo ke ztrátě dat v případě, že by master server spadl dřív, než se data zkopírují na slave server, pro první slave (slave1) byla zvolena varianta synchronní replikace. Je vhodné, aby servery běžely v~lokální síti, protože se tím snižuje pravděpodobnost, že by došlo k~výpadku spojení mezi master a slave1 server a nebylo by tak možno zapisovat na master. 

Druhý server (slave2) bude replikovat asynchronně a zároveň, aby nedocházelo k~přetížení master serveru, bude replikace probíhat ze slave1 na slave2, tedy kaskádově. Ze slave2 lze dále tvořit pravidelnou, například denní nebo týdenní, zálohu pomocí ulitily pg\_dump, která je více popsána v kapitole \odkazKapitola{kPriprava}. Záloha přes pg\_dump tak nebude zatěžovat master server a sama o sobě bude probíhat rychleji, než by tomu bylo na master serveru, který je již tak velmi vytížen dalšími procesy.


Vzhledem k tomu, že existují klienti, kteří mají právo číst i zapisovat, budou přístupy do datábáze řešeny nástrojem pgpool. Uživatelům tedy nebude potřeba dávat přihlašovací údaje dvakrát, jednou pro zápis na master a druhý pro čtení na slave. To jim usnadní práci i z toho pohledu, že si nebudou muset hlídat, ke kterému ze serverů se připojit na základě jejich aktuálního dotazu. pgpool se bude tvářit jako jakákoli jiná databáze, ke které se klienti přihlásí bez ohledu na typ jejich dotazu a pgpool pak sám rozhodne, ke kterému ze serverů klienta přihlásí. Tím bude mít možnost také rozložit zátěž na dostupné uzly v clusteru dle počtu konkrétních dotazů. pgpool bude zároveň uchovávat databázová spojení a při novém dotazu využije stávajícího spojení, místo aby vytvářel spojení nové. Tímto se zajistí plynulost a zvýší rychlost provozu databáze.


        \begin{figure}[H]
          \label{oNavrhKatedra}
          \centering
          \includegraphics[scale=1]{../../../grafy/obr/schema_navrhKatedra2.png}
          \caption {Návrh replikačního řešení}
        \end{figure}

        Vzhledem k tomu, že se k databázi bude přistupovat skrze pgpool, není potřeba aby jednotlivé uzly v clusteru měly veřejnou IP adresu. Plně dostačuje, že servery poběží na lokální síti a pouze pgpool na serveru s veřejnou IP, čímž se zajistí, že data budou přístupná i skrze internet. 

Návrh počítá také s externími pracovišti, která budou často přistupovat do databáze s právem čtení, a budou mít zájem o zrychlení přístupu k datům tím, že se slave server přesune na jejich pracoviště, tedy na hardware, který bude připojen do jejich lokální sítě. Typ replikace se zvolí podle jejich operačního systému a jeho architektury. Pokud se bude jednat o shodný systém, jaký bude použit ve výše popsaném clusteru, pak bude možno použít asynchronní streaming replikaci, naopak pokud se bude bude jednat o systém jiný, bude použita Slony-I replikace.

            \subsection{Příprava serverů před konfigurací replikace}
      \label{kPriprava}
      Na začátku je potřeba připravit servery a nainstalovat na ně PostgreSQL s extenzemi PostGIS, Slony-I a pgpool. Informace o~instalacích jednotlivých komponent jsou dostupné na jejich webových stránkách. Ve Windows si stačí stáhnout pouze instalační balík pro PostgreSQL, který umožňuje instalaci databázového systému včetně všech výše zmíněných extenzí. Pro grafickou administraci databáze je doporučený, ale nepovinný, program \mbox{PgAdmin3}\footnote{více na \url{http://www.pgadmin.org/}}, který je taktéž multiplatformní. Většina příkazů je zde popisována skrze příkazový řádek, grafické rozhraní však poskytuje odpovídající volby.

Všechny technologie byly testovány na operačním systému Debian Linux. Některé příklady použité v této kapitole, především pak ukázky absolutních cest k souborům, tedy odpovídají struktuře tohoto systému. Slony-I bylo navíc vyzkoušeno také na systému Windows XP. Bylo použito databázového systému PostgreSQL ve verzích PostgreSQL 9.1 a 9.3, Postgis verze 1.5 a 2.1, Slony-I verze 2.1 a pgpool verze 3.1 a~3.3.

Pro databázové servery byla zvolena tři datová uložiště, pro jejichž přehled \vizTabulka{tServery}. IP adresy byly pro větší názornost upraveny na rozsah běžné lokální sítě. Vzhledem k tomu, že se do databáze bude přistupovat skrze pgpoool, není třeba, aby kterýkoli z níže vypsaných serverů, měl veřejnou IP adresu. Všechny servery běží na defaultním portu 5432, který je standardem pro PostgreSQL.

        \begin{table}[H]
          \caption{Přehled databázových serverů v navrhovaném clusteru}
            \label{tServery}
            \begin{center}
              \rowcolors{1}{white}{lightgray}
            \begin{tabular}{|ccc|}
              \hline
              \color{purpurova7}{název} & \color{purpurova7}{IP adresa} & \color{purpurova7}{port}\\
                                 master& 192.168.0.100 & 5432\\
                                 slave1& 192.168.0.101 & 5432\\
                                 slave2& 192.168.0.102 & 5432\\
              \hline
              \end{tabular}
            \end{center}
        \end{table}

Aby bylo možné pracovat s databází, je nejdříve nutné chápat význam jednotlivých konfiguračních souborů a mít přehled o souborové struktuře PostgreSQL. Vzhledem k tomu, že si ji každý systém uzpůsobuje podle sebe, nezbývá než po instalaci PostgreSQL nastudovat, kde se jaký soubor nachází. Existuje tabulka \texttt{pg\_settings}, která uchovává veškeré informace o nastavení databáze. SQL příkazem, který čte z této tabulky, je možno vypsat absolutní cestu k datům (\texttt{data\_directory}) a cestu ke třem hlavním konfiguračním souborům:

\begin{itemize}
  \item \texttt{postgres.conf}, který definuje obecné nastavení databáze,
    \item \texttt{pg\_hba.conf}, který povoluje konkrétním uživatelům přístup z určitých IP adres,
    \item \texttt{pg\_ident.conf}, který slouží k mapování uživatelů operačního systému na uživatele PostgreSQL \citep{ObeHsu2012}.
\end{itemize}

Příklad SQL příkazu, spuštěného na master serveru, který vypíše umístění jednotlivých souborů a složek:

\begin{lstlisting}
SELECT name, setting FROM pg_settings WHERE category =  'File Locations';
\end{lstlisting}
      \begin{table}[H]
        \label{fileLocation}
          \begin{center}
            \begin{tabular}{lll}
              \texttt{name} & &\texttt{ settings}\\
                    \texttt{--------------------------------------}&\texttt{+}&\texttt{---------------------------------------------------------------------------------}\\
                      \texttt{data\_directory} & \texttt{|}&\texttt{ /var/lib/postgresql/9.1/main} \\
                  \texttt{external\_pid\_file} & \texttt{|}&\texttt{ /var/run/postgresql/9.1-main.pid} \\
                            \texttt{hba\_file} & \texttt{|}&\texttt{ /etc/postgresql/9.1/main/pg\_hba.conf}\\ 
                         \texttt{config\_file} & \texttt{|}&\texttt{ /etc/postgresql/9.1/main/postgresql.conf} \\
                          \texttt{ident\_file} & \texttt{|}&\texttt{ /etc/postgresql/9.1/main/pg\_ident.conf} \\
            \end{tabular}
          \end{center}
      \end{table}

      U všech typů replikace je potřeba mít vytvořeného databázového uživatele s právem pro replikaci, pod kterým bude daný proces probíhat. Je možné vytvořit nového uživatele a nastavit mu tato práva nebo použít již existující účet \texttt{postgres}, který jako \texttt{SUPERUSER} obsahuje také práva pro replikaci. Je však potřeba mu hned na začátku změnit heslo.

Příklad změny hesla uživatele \texttt{postgres} na master serveru:
\begin{lstlisting}
ALTER ROLE postgres PASSWORD 'kgigis';
\end{lstlisting}

Příklad vytvoření nového uživatele \texttt{replikator} s přidáním práv pro replikaci na master serveru: 
\begin{lstlisting}[]
CREATE ROLE replikator REPLICATION ENCRYPTED PASSWORD   'kgigis';
\end{lstlisting}

Pokud se začíná databázovým systémem, který ještě neobsahuje žádná data, je vhodné replikaci spustit ještě před přidáváním dat. V případě, že již databáze naplněná daty je, není problém replikaci spustit, jen je třeba počítat s delším časem kopírovaní dat a větší opatrností při konfiguraci. 

Každý typ replikace vyžaduje lehce odlišnou přípravu dat před spuštěním samotné replikace. 
Slony-I replikace vyžaduje mít předem vytvořenou strukturu databáze včetně tabulek a poté zajistit existenci totožné kopie na všech serverech v clusteru. Je možné toho dosáhnout použitím utility \texttt{pg\_dump}, která data exportuje na master serveru a \texttt{pg\_restore}, která data importuje na slave serverech. Tímto způsobem lze převádět jak strukturu databáze, tak data, a zároveň to umožňuje přenášet pouze vybrané části databáze. 

Příklad exportu a importu dat z databáze do databáze:
\begin{lstlisting}[keywordstyle=\color{purpurova7},identifierstyle=\color{black},stringstyle=\color{black}]
> pg_dump > /tmp/dump.sql
> pg_restore /tmp/dump.sql
\end{lstlisting}

Streaming replikace vyžaduje kopii celé složky \texttt{data\_directory}. Je mnoho způsobů, jak toho dosáhnout, například klasickým kopírováním skrze utilitu \texttt{cp}, resp. \texttt{scp} u vzdálených složek, nebo utilitou \texttt{rsync}. Kopírování dat za běhu databáze navíc vyžaduje použití příkazu \texttt{SELECT pg\_start\_backup}, který zajistí, že po dobu kopírování budou změny zapisovány do transakčního logu nikoli do databáze, přímý zápis do databáze lze znovu povolit příkazem \texttt{SELECT pg\_stop\_backup}. Tím je zajištěna konzistence kopírovaných dat. 

Při kopírování za běhu databázového systému se zkopíruje také soubor \texttt{postmaster.pid}, který se vytváří po spuštění databázového systému a nese informaci o jeho proces ID. Pomocí tohoto ID je možno s procesem komunikovat nebo jej násilně ukončit. Na slave serveru však tento soubor nenese význam, protože proces tohoto ID neexistuje, a navíc při jeho existenci se služba nespustí, protože se domnívá, že již služba běží. Proto je třeba tento soubor smazat, například ulititou \texttt{rm}.

Příklad zkopírování dat z master serveru na slave1 příkazem spuštěným ze slave1:
\begin{lstlisting} 
psql -U postgres -h 192.168.1.100 -c "SELECT pg_start_backup('x',true);"
\end{lstlisting}
\begin{lstlisting}[keywordstyle=\bfseries\color{purpurova7},identifierstyle=\color{black},stringstyle=\color{black}]
> scp -rv root@192.168.0.100:/etc/postgresql/9.1/main /etc/postgresql/9.1/main
> scp -rv root@192.168.0.100:/var/lib/postgresql/9.1/main /var/lib/postgresql/9.1/main
> rm /var/lib/postgresql/9.1/main/postmaster.pid
\end{lstlisting}
\begin{lstlisting}
psql -U postgres -h 192.168.1.100 -c "SELECT pg_stop_backup();"
\end{lstlisting}

Alternativou výše zmíněního postupu je použití utility přímo určené pro zálohování dat v PostgreSQL nazvané \texttt{pg\_basebackup}. Tento příkaz mimo jiné umožňuje kopírování dat za běhu replikace bez nutnosti použití \texttt{pg\_start/stop\_backup}.

Použití \texttt{pg\_basebackup} pro vytvoření repliky:
\begin{lstlisting}[keywordstyle=\bfseries\color{purpurova7},identifierstyle=\color{black},stringstyle=\color{black}]
> pg_basebackup -D /var/lib/postgresql/9.1/main/ -U replikator -h 192.168.0.100
\end{lstlisting}

Kopírování dat je velice důležitý krok pro správný chod replikace. V případě, že se data nesprávně zkopírují, není možné replikaci zprovoznit. 

Pří kopírování celé datové struktury je vhodné nastavit jednotlivým souborům a~složkám správa. Vzhledem k tomu, že databázový systém zapisuje do složky s daty (\texttt{data\_directory}), musí mít PostgreSQL, i po zkopírování celé datové struktury na jiný server, práva pro zápis.

V neposlední řadě je potřeba zajistit vzájemnou konektivitu všech serverů v replikačním clusteru. S tím souvisí i nutnost povolení přístupů z IP adres slave serverů. K tomu slouží konfigurační soubor \texttt{pg\_hba.conf}. Následující příklad ukazuje možné nastavení souboru \texttt{pg\_hba.conf} na master serveru. Povoluje uživetelům market a replication, přihlášených z dané IP adresy, přistupovat na master server a číst, resp. replikovat data.
{\color{purpurova7}JINÝ PŘÍKLAD}
      \begin{table}[H]
        \label{pgHba}
          \begin{center}
            \begin{tabular}{lllll}
              \texttt{\#host} & \texttt{DATABASE} & \texttt{USER} & \texttt{ADDRESS} & \texttt{METHOD} \\
                \texttt{host} & \texttt{all} & \texttt{market} & \texttt{80.188.74.1/32} & \texttt{md5} \\
       \texttt{host} & \texttt{replication} & \texttt{replication} & \texttt{80.188.74.1/32} &	\texttt{md5} \\
            \end{tabular}
          \end{center}
      \end{table}





      \subsection{Konfigurace replikace}
\subsubsection{Streaming replikace}

Jak bylo nastíněno v kapitole \odkazKapitola{kNavrh}, databázové řešení staví na streaming replikaci a~skládá se ze tří uzlů v~clusteru, jednoho master serveru a dvou slave serverů. Pokud je správně provedena příprava dle kapitoly \odkazKapitola{kPriprava}, samotné nastavení replikace není nijak náročné. V první fázi je potřeba konfigurace souboru \texttt{postgresql.conf} na master serveru. Pro asynchonní replikaci stačí editace parametrů: 
\begin{itemize}
\item \texttt{wal\_level}, který určuje, kolik informací má být zapsáno do transakčního logu (WAL) a
\item \texttt{max\_wal\_senders}, který odpovídá maximálnímu počtu připojených slave serverů. 
\end{itemize}
Hodnota \texttt{wal\_level}, stanovena na \texttt{hot\_stadby}, zajistí, že na slave serveru bude umožněno dotazování. Vzhledem k tomu, že se bude na master server připojovat pouze \texttt{slave1} a všechny další slave servery se poté budou připojovat k němu, pak hodnota \texttt{1} zcela dostačuje. Je však možné hodnotu zvýšit, aby se soubor v budoucnu nemusel znovu editovat z důvodu připojení dalšího serveru.

Pro nastavení synchronní replikace stačí přidat jeden další parametr a to \texttt{synchronous\_standby\_names}, jehož hodnota může být libovolné slovo. 

Konfigurace \texttt{postgres.conf} na master serveru:
  \begin{lstlisting}
  wal_level = hot_standby
  max_wal_senders = 1
  synchronous_standby_names = 'gis'
  \end{lstlisting}

Stejně tak je potřeba konfigurovat \texttt{postgresql.conf} na slave serverech. Hodnoty \texttt{wal\_level} a \texttt{max\_level\_sender} můžou a nemusí zůstat stejné jako na masteru. Pokud však má být slave připraven zastoupit master server v případě jeho výpadku, pak je vhodné, aby hodnoty byly nastaveny shodně. Na slave serveru je dále potřeba editovat:
\begin{itemize}
\item \texttt{hot\_standby}, který určuje, zda je možno dotazovat v průběhu replikace a 
\item \texttt{hot\_stadby\_feedback}, který udává, zda bude replika informovat master server o příkazech na ní provedených.
\end{itemize}

  \begin{lstlisting}
  wal_level = hot_standby
  max_wal_senders = 5
  hot_standby = on	
  hot_standby_feedback = on
  \end{lstlisting}

Posledním krokem je vytvoření souboru \texttt{recovery.conf} na slave serveru ve složce s daty, který definuje parametry:
\begin{itemize}
\item \texttt{standby\_mode}, který povoluje či zakazuje použití serveru jako slave a
\item \texttt{primary\_conninfo}, který nastavuje informace o serveru, ze kterého budou data replikována. Parametr nastavuje IP adresu serveru, ze kterého se data budou replikovat, název replikačního uživatele a jeho heslo a v případě synchronní replikace ještě klíčové slovo, které musí být shodné s hodnotou, která byla nastavená na master serveru v souboru \texttt{postgresql.conf} v parametru \texttt{synchronous\_standby\_name}.
\end{itemize}

Konfigurace \texttt{recovery.conf} uloženého ve složce s daty na \texttt{slave1}, který je připojován na master server a běží jako synchronní:

\begin{lstlisting}
standby_mode='on'
   primary_conninfo='host=192.168.1.100 user=replikator password=kgigis application_name=gis'  
\end{lstlisting}

V návrhu je počítáno s kaskádovou replikací, tedy s tím, že se slave1 bude připojovat na slave2 místo na master server. To lze nastavit úpravou souboru \texttt{recovery.conf}, kde IP adresa parametru \texttt{host} bude odpovídat IP adrese serveru slave1. 

Konfigurace \texttt{recovery.conf} uloženého ve složce s daty na \texttt{slave2}, který běží jako asynchronní a je kaskádově připojován ke slave1:

  \begin{lstlisting}
standby_mode='on'
   primary_conninfo='host=192.168.1.101 user=replikator password=kgigis'
  \end{lstlisting}

To, že replikace běží lze zkontrolovat několika způsoby. Na slave server nesmí být možné zapsat žádná data:

\begin{lstlisting}
  INSERT INTO student (jmeno) VALUES('Jan Vlasovec');
\end{lstlisting}
\begin{lstlisting}[keywordstyle=\color{black},identifierstyle=\color{black},stringstyle=\color{black}]
  ERROR:  cannot execute INSERT in a read-only transaction
\end{lstlisting}

Připojené repliky lze vypsat pomocí SQL příkazu \texttt{pg\_stat\_replication}, kde poslední parametr udává, zda se jedná o synchronní nebo asychronní replikaci:

\begin{lstlisting}
  SELECT usename, application_name, client_addr, state, sync_state FROM pg_stat_replication ;
\end{lstlisting}

      \begin{table}[H]
        \label{pgHba}
          \begin{center}
            \begin{tabular}{lllllllll}
              \texttt{usename} & \texttt{|} & \texttt{application\_name}  &\texttt{|}  &\texttt{client\_addr}  &\texttt{|}    &\texttt{state}    &\texttt{|}  &\texttt{sync\_state} \\
           \texttt{---------}  &\texttt{+}  &\texttt{----------------}  &\texttt{+}  &\texttt{-------------}  &\texttt{+}  &\texttt{-----------}  &\texttt{+}  &\texttt{------------}\\
          \texttt{replikator}  &\texttt{|}  &\texttt{gis}               &\texttt{|}  &\texttt{192.168.1.101}  &\texttt{|}  &\texttt{streaming}  &\texttt{|}  &\texttt{sync}\\
              \texttt{(1 row)} & & & & & & & & \\

            \end{tabular}
          \end{center}
      \end{table}

Stejně tak lze na slave serveru zjistit, zda běží jako replika či nikoli pomocí SQL příkazu \texttt{pg\_is\_in\_recovery()}:

\begin{lstlisting}
  postgres=# select pg_is_in_recovery();
\end{lstlisting}
\begin{lstlisting}[basicstyle=\footnotesize\ttfamily,keywordstyle=\color{black},identifierstyle=\color{black},stringstyle=\color{black}]
  pg_is_in_recovery 
  -------------------
  t
  (1 row)
\end{lstlisting}

V případě, že master server spadne, je možné během pár minut vyměnit role, určit jako master jeden ze slave serverů. Lze to udělat několik způsoby, jedním z nich je sledování existence souboru, který je definován v souboru \texttt{recovery.conf} na kterémkoli slave serveru:

\begin{lstlisting}
  trigger_file = '/tmp/trigger.txt'
\end{lstlisting}

Název souboru může být zvolen libovolně a může být zcela prázdný. Slave server pouze hlídá jeho existenci a jen to, že se soubor objeví v dané složce, způsobí, že se ze slave serveru stane master. Obsah souboru můžou tvořit další instrukce, které se můžou ovlivnit další chod databáze. 

\subsubsection{Slony-I replikace}

  Ukázka konfiguračního souboru na master serveru pro inicializaci clusteru (init cluster) s názvem init\_master.txt uloženého ve složce z daty:
  \begin{lstlisting}[basicstyle=\footnotesize\ttfamily,identifierstyle=\color{black},stringstyle=\color{black},keywordstyle=\color{black},
  ]

  cluster name = second_cluster;
  # definice uzlu
  node 1 admin conninfo='dbname= host=192.168.1.1 user=replikator password=kgigis';
  node 2 admin conninfo='dbname= host=192.168.1.2 user=replikator password=kgigis';

  # inicializace clusteru
  init cluster (id=1, comment = 'master');
  store node (id=2, comment = 'slave1', event node=1);

  # vytvoreni replikacniho setu a pridani tabulek do setu
  create set (id=1, origin=1, comment='Tabulky k replikaci');

  set add table (set id=1, origin=1, id=1, fully qualified name = 'public.student', comment='prehled studentu');
  set add table (set id=2, origin=1, id=1, fully qualified name = 'public.student2', comment='prehled studentu');

  store path (server=1, client=2, conninfo='dbname= host=localhost user=postgres  password=tfgt');
  store path (server=2, client=1, conninfo='dbname=repl2 host=localhost user=postgres  password=tfgt');

  \end{lstlisting}


    %------------------------------------------------------------------------- DISKUZE
    \newpage
    \section{DISKUZE}

    %------------------------------------------------------------------------- ZÁVĚR
    \newpage
    \section{ZÁVĚR}

    %------------------------------------------------------------------------- LITERATURA
    \makeBibliography{literatura}
    %\bibliography{literatura}


    %------------------------------------------------------------------------- SUMMARY
    \begin{summary}
      There is summary of all aims, methods and results in this chapter.
      Summary is not only translation of chapter Závěr. There is more
      information from chapters Cíle, Výsledky and Diskuze. Number of
      pages of Summary chapter is two at least. The style is Normalni
      Summary. Language is set to Angličina(Velká Británie) for automatic
      spell check. Do not use language Angličtina(USA). 
    \end{summary}

    %------------------------------------------------------------------------- PŘÍLOHY
    \newpage
    \addcontentsline{toc}{section}{PŘÍLOHY}
    \vspace*{180pt}
    \begin{center}
    \section*{PŘÍLOHY}
    \end{center}
    \vspace*{\fill}

    \newpage
    \begin{prilohy}
      \textbf{Volné přílohy}

      Příloha 1 CD \newline
      \newline
      \textbf{Popis sktruktury CD}

        Adresáře a soubory:

        - \texttt{skripty\slash} - složka se skripty

        - \texttt{web\slash} - webové stránky jako doplněk k diplomové práci

        - \texttt{Solanska\_DP.pdf} - text diplomové práce

      \vspace*{\fill}
  \end{prilohy}
      
  \end{document}
