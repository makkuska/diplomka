Tato kapitola se zabývá hodnocením současného stavu správy dat na katedře geoinformatiky, návrhem databázového řešení dle požadavků a možností katedry a podrobně popisuje vytvoření testovacího prostředí na serverech katedry dle vytvořeného návrhu. Do hloubky popisuje konfiguraci vybraných nástrojů, včetně jejich praktického spuštění. 
Katedra aktuálně provozuje tři servery, konkrétně virtus.upol.cz, gislib.upol.cz a geohydro.upol.cz. Poslední z jmenovaných byl poskytnut jako testovací server pro tutu práci a v budoucnu s ním počítá jako s master serverem pro zde popisované databázové řešení. První dva zmíněné servery jsou aktivně používané a kromě jiných služeb, je zde spuštěn také MS SQL Server. Na tyto databáze je připojený například geoportál běžící skrze ArcGIS Server nebo metadatový systém Micka. Každý ze serverů momentálně obsahuje jiná data, která nejsou pravidelně zálohovaná, protože aktualizace dat není příliš častá. Aktuální stav nepoužívá replikaci dat, což hrozí nedostupností dat výpadkem serveru. 
Databáze aktuálně obsahují data z projektů BotanGIS\footnote{http://botangis.upol.cz/botangis/mapa}, Virtuální studovny CHKO Litovelské Pomoraví\footnote{http://virtus.upol.cz/}, dále data metadatového systému Micka\footnote{gislib.upol.cz/metadata}, data ze senzorové sítě KGI, data ke studentským prací a ukázková data určená pro výuku. Je založeno přibližně 10 účtů, které mají přístup pro zápis, a řádově v desítkách účtů s právem čtení. V současné situaci není příliš často do databází zapisováno. 

Současný stav, kdy se přenášejí data přes různá hardwarová zařízení nebo kopírují po síti, není plně vyhovující mimo jiné z důvodu, že se často jedná o velké objemy dat, jejichž kopie může trvat řádově až desítky minut. 
Katedra má zájem využít potenciál databázového řešení a plánuje využít tento návrh k uložení dalších datových sad, které má k dispozici a které jsou momentálně dostupné ve formátech shapefile nebo geodatabase, ale zatím neukládá do databáze a nesdílí skrze ní. Jedná se například o datové sady ArcČR500, Data200 (ČUZK) nebo CEDA ČR 150.
Takto připravená data poté mnohem snáze využitelná jak pracovníky, tak i studenty katedry nejen pro výuku, ale také pro jejich odborné práce. 
