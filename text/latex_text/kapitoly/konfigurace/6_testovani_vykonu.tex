\subsection{Testování výkonu}
Před spuštěním databázového clusteru v reálném provozu je vhodné provést
optimalizaci nastavení nejen replikace, ale celého databázového řešení. Je třeba
zohlednit atributy jako výkonnost serverů, prostupnost sítě, způsob užití
databáze, počet uživatelů, atd.

V rámci pozorování chování databázového řešení bylo zjištěno, že data i většího
rozsahu jsou přenesena v řádech jednotek vteřin. Je však nutno zohlednit, že
všechny tři uzly komunikují po vnitřní síti, což rychlost přenosu zvyšuje.
Zároveň nebyl zaznamenán případ, kdy by se přenesla pouze část dat nebo byla
některá přenesená data chybná.

Při použití pgpool je možno testovat rozdíl počtu transakcí za sekundu v případě
připojení pouze na master, resp. slave server a na pgpool. V ideálním případě by
měl pgpool znásobit počet transakcí třikrát. To, že se zvýší počet transakcí za
sekundu, dokazuje, že pgpool efektivně rozkládá dotazy mezi jednotlivé uzly. Je
třeba zohlednit, že pgpool nějakou dobu vyhodnocuje příkaz, než jej přepošle
dál. To může proces mírně zpomalit, pak záleží na tom, jak velké dotazy jsou do
něj posílány.

Při použití asynchronní replikace nedochází ke zpomalení vykonání transakcí při
čtení ani zápisu. U synchronní replikace však dochází ke zpomalení zápisu, neboť
se čeká, až je dotaz zapsán na slave, viz kapitola \ref{kpgpool}.

Testovat je vždy potřeba až hotové řešení s ohledem na konkrétní nastavení.
Zohlednit je potřeba i typy operací, které jsou v databázi vykonávány, např.
jaké příkazy provádí ArcGIS server.  Pro testování výkonu existuje několik
nástrojů, například pgbench.
