
Jak už bylo zmíněno v kapitole \odkazKapitola{kSlony}, u Slony-I není možné replikovat strukturu databáze. Výhodou však je, že lze vybrat pouze některé tabulky k replikaci a zároveň, že se nemusí shodovat názvy databází. V tomto ohledu tedy nabízí velkou variabilitu propojení. Jak bylo nastíněno v kapitole \odkazKapitola{kPriprava}, nastavení replikace začíná přípravou uživatele, pod kterým bude proces probíhat, vytvořením datové struktury a zajištění shodné kopie dat na všech uzlech v clusteru.

Pro názornost byla vytvořena databáze {\it studenti} pomocí příkazu \texttt{CREATE DATABASE} a  tabulky \texttt{student} a \texttt{rodne\_mesto}, u kterých byl nastaven \texttt{primary key} jakožto podmínka Slony-I replikace, pomocí SQL příkazu \texttt{CREATE TABLE}:

\begin{lstlisting}
CREATE DATABASE studenti;
studenti=# CREATE TABLE student (id int, jmeno varchar, id_rodne_mesto int, primary key(id));
\end{lstlisting}
\begin{lstlisting}[keywordstyle=\color{black},identifierstyle=\color{black},stringstyle=\color{black}]
NOTICE:  CREATE TABLE / PRIMARY KEY will create implicit index "student_pkey" for table "student"
CREATE TABLE
\end{lstlisting}
\begin{lstlisting}
studenti=# CREATE TABLE rodne_mesto (id int, jmeno varchar, umisteni geometry, primary key(id));
\end{lstlisting}
\begin{lstlisting}[keywordstyle=\color{black},identifierstyle=\color{black},stringstyle=\color{black}]
NOTICE:  CREATE TABLE / PRIMARY KEY will create implicit index "rodne_mesto_pkey" for table "rodne_mesto"
CREATE TABLE
\end{lstlisting}


Dále tento způsob obnáší vytvoření konfiguračních skriptů, které zajistí inicializaci replikace. Slony-I pro to používá vlastní konfigurační jazyk, pomocí kterého se nastavují konkrétní požadavky na replikaci. Na začátku se díky němu sestaví konfigurační soubor k inicializaci replikace, později se používá pro jakýkoli zásah do replikace, například přidání další tabulky do replikačního setu nebo změny struktury databáze. Tento vzniklý konfigurační skript se provede pomocí utility \texttt{slonik}, která se vždy spouští jednorázově a vykonává požadavky definované v konfiguračním souboru. 

Ukázka konfiguračního skriptu nazvaného \texttt{init\_master.txt} uloženého na master serveru pro inicializaci replikačního clusteru (init cluster):
%\begin{lstlisting}[identifierstyle=\color{black},stringstyle=\color{black},keywordstyle=\color{black}]
\begin{lstlisting}[language=ruby]
$master = 'dbname=studenti host=192.168.1.100 user=replikator password=kgigis'
$slave1 = 'dbname=studenti host=192.168.1.101 user=replikator password=kgigis'
$slave2 = 'dbname=studenti host=192.168.1.102 user=replikator password=kgigis'

# nazev clusteru
cluster name = gis_cluster;

# definice jednotlivych uzlu v clusteru
node 1 admin conninfo=$master;
node 2 admin conninfo=$slave1;
node 3 admin conninfo=$slave2;

# inicializace clusteru
init cluster (id=1, commnet = 'master');
store node   (id=2, comment = 'slave1', event node=1);
store node   (id=3, comment = 'slave2', event node=1);

# vytvoreni replikacniho setu
create set (id=1, origin=1, comment='Tabulky k replikaci');
# pridani tabulek do setu
# prvni id odpovida id setu
# druhe id odpovida id uzlu masteru
# treti id je id nove pridane tabulky
set add table (set id=1, origin=1, id=1, fully qualified name = 'public.student', comment='seznam studentu');
set add table (set id=1, origin=1, id=2, fully qualified name = 'public.rodne_mesto', comment='seznam mest');

store path (server=1, client=2, conninfo=$slave1);
store path (server=1, client=3, conninfo=$slave2);
store path (server=2, client=1, conninfo=$master);
store path (server=2, client=3, conninfo=$slave2);
store path (server=3, client=1, conninfo=$master);
store path (server=3, client=2, conninfo=$slave1);

store listen (origin=1, provider=2, receiver=1);
store listen (origin=1, provider=1, receiver=2);
store listen (origin=1, provider=2, receiver=3);
\end{lstlisting}

Skript {\texttt Slonik} volá příkazy:
\begin{itemize}
\item \texttt{cluster name} představující jedinečný název pro daný cluster,
\item \texttt{node ival/číslo admin conninfo} definující všechny uzly v clusteru a parametry jejich připojení k databázi,
\item \texttt{init cluster}, který inicializuje cluster a nastavuje master server jako uzel s id 1, 
\item \texttt{store node}, který vytváří další uzly,
\item \texttt{create set} vytvářející soubor tabulek určených k replikaci, 
\item \texttt{set add table} přidávájící vždy jednu tabulku do replikačního setu s identickým id,
\item \texttt{store path}, který nastavuje cesty mezi jednotlivými uzly a 
\item \texttt{store listen} nastavující naslouchání jednotlivých uzlů. 
\end{itemize}

Slony-I rozlišuje tři druhy serverů:
\begin{itemize}
\item \texttt{origin} odpovídá master serveru, tedy jedinému uzlu, kterému je povoleno zapisování,
\item \texttt{subscriber} je ekvivalentem slave serveru s právy čtení a 
\item \texttt{provider} je poskytovatel dat, může to být master server, ale při kaskádové replikaci, kterýkoliv ze slave serverů. 
\end{itemize}

Konfigurační skript pro inicializaci clusteru se spustí ze stejné složky, ve které je uložen daný soubor, příkazem \texttt{slonik} a názvem souboru, viz příklad:

\begin{lstlisting}
# slonik init_master.txt
\end{lstlisting}

Pomocí dalších skriptu se mohou slave servery přihlásit odběru replikačního setu. Příklad vytvoření skriptu \texttt{subscribe\_slave1.txt} pro přidání serveru do existujícího clusteru (subscribe set):

%\begin{lstlisting}[identifierstyle=\color{black},stringstyle=\color{black},keywordstyle=\color{black}]
\begin{lstlisting}[language=ruby]
# nazev clusteru
cluster name = gis_cluster;

# definice jednotlivych uzlu v clusteru
node 1 admin conninfo=$master;
node 2 admin conninfo=$slave1;
node 3 admin conninfo=$slave2;

subscribe set (id=1, provider=1, receiver=2, forward=yes);
\end{lstlisting}

Spuštění skriptu \texttt{slonik} pro přidání slave1 do clusteru:

\begin{lstlisting}
slonik subscribe_slave1.txt
\end{lstlisting}

Stejný způsobem se připojít k odběru i slave2.

To, že se vytvořit cluster a tabulka do něj byla přidána, lze zkontrolovat ve výpisu, kde nově přibyly triggery, které sledují změny, které v tabulce nastanou:

    \begin{table}[H]
      \label{pgHba}
        \begin{center}
          \begin{tabular}{lllllll}
            \multicolumn{7}{l}{\texttt{studenti=\# \slash d student}} \\
            \multicolumn{5}{c}{\texttt{Tabulka "public.student"}} & & \\
                      \texttt{Column} & \texttt{|} & \texttt{Type} & \texttt{|} & \texttt{Modifiers} & & \\ 
            \texttt{----------------} & \texttt{+} & \texttt{-------------------} & \texttt{+} & \texttt{--------------} & & \\
                          \texttt{id} & \texttt{|} & \texttt{integer} & \texttt{|} & \texttt{not null} & & \\
                        \texttt{jmeno} & \texttt{|} & \texttt{character varying} & \texttt{|} & & & \\
                                \texttt{id\_rodne\_mesto} & \texttt{|} & \texttt{integer} & \texttt{|} & & \\
            \multicolumn{7}{l}{\texttt{Indexes: "student\_pkey" PRIMARY KEY, btree (id)}} \\
            \multicolumn{7}{l}{\texttt{Triggers: \_gis\_cluster\_logtrigger AFTER INSERT OR DELETE}} \\
            \multicolumn{7}{l}{\texttt{OR UPDATE ON repl\_test FOR EACH ROW EXECUTE PROCEDURE}} \\
            \multicolumn{7}{l}{\texttt{\_is\_cluster.logtrigger('\_gis\_cluster', '1', 'k')}} \\
            \multicolumn{7}{l}{\texttt{Disabled triggers: \_gis\_cluster\_denyaccess BEFORE INSERT}} \\
            \multicolumn{7}{l}{\texttt{OR DELETE OR UPDATE ON repl\_test FOR EACH ROW EXECUTE PROCEDURE}} \\
            \multicolumn{7}{l}{\texttt{\_gis\_cluster.denyaccess('\_gis\_cluster')}} \\
          \end{tabular}
        \end{center}
    \end{table}

Běh replikace je zajištěn vlastním démonem, který je možnost spustit v okamžiku, kdy je vytvořen cluster a všechny repliky jsou do něj přidány. Démon \texttt{slon}, který je potřeba spustit na všech uzlech, v parametrech přebírá název clusteru a hodnoty připojení daného uzlu k databázi. Je důležité, aby log po spuštění nevypisoval žádné chyby, jinak je potřeba zkontrolovat všechny příkazy konfiguračních souborů. 

Příklad spuštění démona na master serveru s názvem clusteru a parametry připojení k serveru:
\begin{lstlisting}
slon gis_cluster 'host=192.168.1.100 dbname=student  user=replikator'
\end{lstlisting}
Obdobně je démona spuštěn i na obou slave serverech:
\begin{lstlisting}
slon gis_cluster 'host=192.168.1.101 dbname=student  user=replikator'
slon gis_cluster 'host=192.168.1.102 dbname=student  user=replikator'
\end{lstlisting}

Podoboně jako u streaming replikace, lze zkontrolovat, že replikace běží správně, přidáním nového záznamu na slave server. Pokud nepovolí přidání a vypíše následující chybu, znamená, že je replikace správně nastavená a funkční.

\begin{lstlisting}
studenti=# INSERT INTO student (jmeno) VALUES ('Josef Kraus');
\end{lstlisting}
\begin{lstlisting}[identifierstyle=\color{black},stringstyle=\color{black},keywordstyle=\color{black}]
ERROR:  Slony-I: Table repl_names is replicated and cannot be modified on a subscriber node - role=0
\end{lstlisting}

Přidání další tabulky či jakákoli jiná změna struktury databáze probíhá v několika krocích. Nejdříve je potřeba vytvořit soubor, který bude obsahovat SQL příkaz provádějící zvolenou změnu databáze. Poté se spustí program slonik, který zavolá soubor s SQL příkazem a vykoná jej na všech uzlech clusteru. 

Vytvoření souboru \texttt{createTable.sql} ve slozce \texttt{/tmp/} s SQL příkazem \texttt{CREATE TABLE}:

\begin{lstlisting}
CREATE TABLE predmet (id int, jmeno varchar, primary  key(id));
\end{lstlisting}

Vytvoření skriptu \texttt{ddlZmena.txt}, který umožní přidání tabulky za chodu replikace:

%\begin{lstlisting}[identifierstyle=\color{black},stringstyle=\color{black},keywordstyle=\color{black}]
\begin{lstlisting}[language=ruby]
# nazev clusteru
cluster name = gis_cluster;

# definice jednotlivych uzlu v clusteru
node 1 admin conninfo=$master;
node 2 admin conninfo=$slave1;
node 3 admin conninfo=$slave2;

execute script (
  SET ID = 1,
  filename = '/tmp/createTable.sql',
  event node = 1
);

\end{lstlisting}

Spuštění programu \texttt{slonik}, který spustí skript a vykoná daný SQL příkaz na všech uzlech: 

\begin{lstlisting}
slonik ddlZmena.txt
\end{lstlisting}
\begin{lstlisting}[identifierstyle=\color{black},stringstyle=\color{black},keywordstyle=\color{black}]
DDL script consisting of 2 SQL statements
DDL Statement 0: (0,67) [ CREATE TABLE predmety (id int, nazev varchar, primary key(id));]
slony_ddl.txt:6: NOTICE:  CREATE TABLE / PRIMARY KEY will create implicit index "predmety_pkey" for table "predmety"
DDL Statement 1: (67,69) [ ]
Submit DDL Event to subscribers...
\end{lstlisting}

Podrobný výpis informuje, že se změnilo DDL\footnote{Data Definition Language} schéma, které provádí příkazy CREATE, ALTER, DROP, tedy příkazy měnící strukturu databáze. Posledním řádkem potvrzuje, že se schéma zapsalo také na slave servery. Takto se tabulka přidá do databáze, nikoliv však do replikačního clusteru. K tomu je potřeba vytvořit další slonik skript. 

Vytvoření skriptu \texttt{add\_to\_set.txt} pro přidání tabulky do replikačního clusteru:

%\begin{lstlisting}[identifierstyle=\color{black},stringstyle=\color{black},keywordstyle=\color{black}]
\begin{lstlisting}[language=ruby]
# nazev clusteru
cluster name = gis_cluster;

# definice jednotlivych uzlu v clusteru
node 1 admin conninfo=$master;
node 2 admin conninfo=$slave1;
node 3 admin conninfo=$slave2;

# definice noveho setu (id 2)
create set (id=2, origin=1, comment='Dalsi tabulky k replikaci');

# pridani nove vytvorene tabulky
set add table (set id=2, origin=1, id=3, fully qualified name = 'public.predmet', comment='seznam predmetu');

# pridani noveho setu id=2
subscribe set (id=2, provider=1, receiver=2);

# spojeni setu id2 se setem id1
merge set(id=2, add id=1, origin=1);
\end{lstlisting}

Spuštění skriptu \texttt{slonik}, které příkazy vykoná

\begin{lstlisting}
slonik add_to_set.txt
\end{lstlisting}

Podobně je možné provést také smazání tabulky pomocí parametru \texttt{DROP SET}. 

