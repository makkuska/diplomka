
Jak bylo nastíněno v kapitole \odkazKapitola{kNavrh}, databázové řešení staví na
streaming replikaci a~skládá se ze tří uzlů v~clusteru - jednoho master serveru a
dvou slave serverů. Pokud je správně provedena příprava dle kapitoly
\odkazKapitola{kPriprava}, samotné nastavení replikace není nijak náročné. V
první fázi je potřeba zeditovat soubor \texttt{postgresql.conf} na master
serveru. Pro asynchonní replikaci stačí nastavit parametry:
\begin{itemize}
  \item \texttt{wal\_level}, který určuje, jaké informace mají být do transakčního
    logu (WAL) zapisovány a
  \item \texttt{max\_wal\_senders}, který nastavuje maximální počet připojených
    slave serverů.
\end{itemize}

Hodnota \texttt{wal\_level} nastavená na \texttt{hot\_stadby} zajistí, že do
transakčního logu bude zapisován takový typ informací, který poté na slave
serveru umožní dotazování. Vzhledem k tomu, že se bude na master
server připojovat pouze \texttt{slave1} a všechny další slave servery se poté
budou připojovat k němu, hodnota \texttt{1} parametru \texttt{max\_wal\_senders}
zcela dostačuje. Je však možné hodnotu rovnou navýšit, aby se soubor v budoucnu
nemusel znovu editovat z důvodu připojení dalšího serveru.

Tyto dva parametry stačí pro nastavnení achynchronní replikace, pro nastavení
synchronní je ještě potřeba přidat parametr
\texttt{synchronous\_standby\_names}, jehož hodnota je použita při výběru slave
serverů pro synchronní replikaci (viz parametr \texttt{application\_name} u
editace souboru \texttt{recovery.conf}).

Konfigurace \texttt{postgres.conf} na master serveru:
\begin{lstlisting}
wal_level = hot_standby
max_wal_senders = 1
synchronous_standby_names = 'gis'
\end{lstlisting}

Stejně tak je potřeba konfigurovat \texttt{postgresql.conf} na slave serverech.
Hodnoty parametrů \texttt{wal\_level} a \texttt{max\_level\_sender} mohou a
nemusí zůstat stejné jako na masteru. Pokud však má být slave připraven
zastoupit master server v případě jeho výpadku, je nutné, aby hodnoty byly
nastaveny shodně.  Na slave serveru je dále potřeba nastavit:
\begin{itemize}
  \item\texttt{hot\_standby}, který určuje, zda je na slave serveru umožněno
    dotazování a
  \item\texttt{hot\_stadby\_feedback}, který omezuje riziko přerušení
    dotazu na slave v případě kolidujících změn na master databázi.
\end{itemize}

Konfigurace \texttt{postgresql.conf} shodně na obou slave serverech:
\begin{lstlisting}
wal_level = hot_standby
max_wal_senders = 5
hot_standby = on
hot_standby_feedback = on
\end{lstlisting}

Posledním krokem je vytvoření souboru \texttt{recovery.conf} na slave serveru ve
složce s daty, který definuje parametry:
\begin{itemize}
  \item \texttt{standby\_mode}, který povoluje či zakazuje použití serveru jako
    slave a
  \item \texttt{primary\_conninfo}, který nastavuje informace o serveru, ze
    kterého budou data replikována - IP adresu serveru, port, název replikačního
    uživatele a jeho heslo a v případě synchronní replikace ještě hodnotu,
    která musí být shodná s hodnotou, která byla nastavená na master serveru v
    souboru \texttt{postgresql.conf} v parametru
    \texttt{synchronous\_standby\_name}.
\end{itemize}

Ukázka konfigurace \texttt{recovery.conf} uloženého ve složce s daty na
\texttt{slave1}, který je připojen k master serveru a replikuje synchronně:
\begin{lstlisting}
standby_mode='on'
primary_conninfo='host=192.168.1.100 user=replikator password=kgigis application_name=gis'
\end{lstlisting}

V návrhu je počítáno s kaskádovou replikací, tedy s tím, že se \texttt{slave1}
bude připojovat k \texttt{slave2}, nikoli k master serveru. To lze nastavit
úpravou souboru \texttt{recovery.conf}, kde IP adresa parametru \texttt{host}
bude odpovídat IP adrese serveru \texttt{slave1}.

Ukázka konfigurace \texttt{recovery.conf} uloženého ve složce s daty na
\texttt{slave2}, který replikuje asynchronně a připojuje se k
\texttt{slave1}:
\begin{lstlisting}
standby_mode='on'
primary_conninfo='host=192.168.1.101 user=replikator password=kgigis'
\end{lstlisting}

To, že je replikace správně nastavená, lze zkontrolovat několika způsoby.
Připojené slave servery lze vypsat pomocí dotazu nad tabulkou
\texttt{pg\_stat\_replication}, kde poslední atribut udává, zda se jedná o
synchronní, nebo asychronní replikaci.

Výsledek dotazu spuštěného na master serveru:

\begin{lstlisting}
SELECT usename, application_name, client_addr, state, sync_state FROM pg_stat_replication;
\end{lstlisting}
\begin{table}[H]
  \label{pgHba}
  \begin{center}
    \begin{tabular}{lll}
      \texttt{-[ RECORD 1 ]----} & \texttt{+} & \texttt{--------------}\\
      \texttt{usename}           & \texttt{|} & \texttt{replikator}\\
      \texttt{application\_name} & \texttt{|} & \texttt{gis}\\
      \texttt{client\_addr}      & \texttt{|} & \texttt{192.168.1.101}\\
      \texttt{state}             & \texttt{|} & \texttt{streaming}\\
      \texttt{sync\_state}       & \texttt{|} & \texttt{sync}\\
    \end{tabular}
  \end{center}
\end{table}
Výsledek téhož dotazu spuštěného na \texttt{slave1}:
\begin{table}[H]
  \label{pgHba}
  \begin{center}
    \begin{tabular}{lll}
      \texttt{-[ RECORD 1 ]----} & \texttt{+} & \texttt{--------------}\\
      \texttt{usename}           & \texttt{|} & \texttt{replikator}\\
      \texttt{application\_name} & \texttt{|} & \texttt{walreceiver}\\
      \texttt{client\_addr}      & \texttt{|} & \texttt{192.168.1.102}\\
      \texttt{state}             & \texttt{|} & \texttt{streaming}\\
      \texttt{sync\_state}       & \texttt{|} & \texttt{async}\\
    \end{tabular}
  \end{center}
\end{table}

Stejně tak lze na slave serveru zjistit, zda skutečně replikuje z master serveru
voláním funkce \texttt{pg\_is\_in\_recovery()}:
\begin{lstlisting}
postgres=# select pg_is_in_recovery();
\end{lstlisting}
\begin{table}[H]
  \begin{center}
    \label{pgHba}
    \begin{tabular}{l}
      \texttt{pg\_is\_in\_recovery}\\
      \texttt{---------------------------------}\\
      \texttt{t}\\
      \texttt{(1 row)}\\
    \end{tabular}
  \end{center}
\end{table}

Zároveň na slave server nesmí být možné zapsat žádná data:
\begin{lstlisting}
INSERT INTO student (jmeno) VALUES('Jan Vlasovec');
\end{lstlisting}
\begin{lstlisting}[keywordstyle=\color{black},identifierstyle=\color{black},stringstyle=\color{black}]
ERROR: cannot execute INSERT in a read-only transaction
\end{lstlisting}

V případě, že master server spadne, je možné během pár minut vyměnit role a~určit
jako master jeden ze slave serverů. Lze to udělat několik způsoby, jedním z~nich
je sledování existence souboru definovaného v souboru \texttt{recovery.conf} na
kterémkoli slave serveru:
\begin{lstlisting}
trigger_file = '/tmp/trigger.txt'
\end{lstlisting}

Název souboru může být zvolen libovolně a nemusí obsahovat žádná data. Slave
server pouze hlídá jeho existenci a jen to, že se soubor objeví v dané složce,
způsobí, že slave server povýší na master. Obsah souboru mohou dále tvořit další
instrukce, které mohou ovlivnit další chod databáze.

