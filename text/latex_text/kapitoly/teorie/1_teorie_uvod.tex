        Jak definuje \cite{Oppel2009}, databáze je soubor vzájemně propojených datových
        položek, které jsou spravovány jako jeden celek \citep{Oppel2009}. Databáze
        představuje entity, atributy a logické vztahy mezi entitami, často zvané
        relace. Jinými slovy, databáze obsahuje data, která logicky související
        \citep{Connolly2005}. Databáze umožňuje ukládání a editaci dat, rychlé
        vyhledávání a komplexní analýzu dat \citep{Momjian2001}. Systém řízení báze
        dat\footnote{V anglickém originále Database Management System (DBMS)} je
        počítačový software, který umožňuje uživatelům přistupovat k databázi,
        definovat, vytvářet a udržovat data \citep{Connolly2005}. Pro uložení dat
        malého projektu je samozřejmě možno použít i jiného formátu určeného pro
        ukládání dat, například tabulkového procesoru. Pro komplexní správu dat velkého
        projektu je však databáze více než vhodná. 

        Prostorová databáze, někdy také zvaná geodatabáze, není nic jiného než databáze
        přidaná o datový typ určený pro ukládání prostorové informace o prvku,
        prostorové indexy a sadu funkcí vhodných pro správu prostorových dat. Více
        informací o prostorových databázích v kapitole \odkazKapitola{PostgreSQL} PostgreSQL 9.x (PostGIS) a \odkazKapitola{MSSQL} MS SQL Server 2008.

        Z toho vyvstává otázka, co jsou prostorová data, také zvaná geodata. Z pohledu
        společnosti ESRI se jedná se prvky, které nesou informaci o geografické poloze,
        zakódovanou informaci o tvaru (bod, line, polygon) a popis geografického jevu.
        Tato geodata jsou uložená ve formátu, který je možno použít v geografickém
        informačním systému \citep{Esri2006}. Příkladem takového formátu může být
        vektorový Esri shapefile, Esri coverage, GML, KML, GeoJSON nebo rastrový Erdas
        Image a GeoTIFF. Dalším způsobem je již zmíněná databáze, do níž se vektorová
        data ukládají ve specifickém tvaru daném standardem OGC\footnote{OGC standardy
        jsou kontrolované konsorciem Open Geospatial Consortium, \newline{zdroj
        http://www.opengeospatial.org/ogc}} Simply Feature for SQL 1.2.1, který
        specifikuje způsob uložení dat v digitální podobě. Simple Features je založen
        na 2D geometrii s~možností lineární interpolace mezi lomovými body. To umožňuje
        vložení následujících prvků:

        \begin{itemize}
          \item bod - POINT(0 0)
          \item linie - LINESTRING(0 0, 1 1, 1 2)
          \item polygon - POLYGON ((0 0,4 0,4 4,0 4,0 0),(1 1, 2 1, 2 2, 1 2,1 1))
          \item série bodů - MULTIPOINT((0 0),(1 2))
          \item série linií - MULTILINESTRING((0 0,1 1,1 2),(2 3,3 2,5 4))
          \item geometrická kolekce, která může obsahovat různé geoprvky (body, linie i polygony) - GEOMETRYCOLLECTION(POINT(2 3),LINESTRING(2 3,3 4))\footnote{Zdroj http://postgis.net/docs/manual-2.1/using\_postgis\_dbmanagement.html\#RefObject}
        \end{itemize}

        První slovo specifikace určuje druh prvku (point, linestring, polygon, multipoint,~...), následují v závorce vypsané souřadnice lomových bodů. Za tím ještě může následovat volitelný parametr kód souřadnicového systému.

        Hodnoty lze dále vkládat přes Well-Known Binary (WKB) nebo Well-Known Text (WKT) reprezentaci. PostGIS funkce pro vkládání geometrie vypadá následovně:

        \begin{itemize}
          \item ST\_AsBinary(geometry) pro bitový zápis WKB
          \item ST\_AsText(geometry) pro WKT text
        \end{itemize}

        Příklad uložení linie do databáze s jedním lomovým bodem v souřadnicovém systému WGS84:
        \newline \newline
        \texttt{(LINESTRING(15.96 50.84, 17.29 49.64, 18.27 49.80), 4326)\hspace*{4.5em}(1)}
