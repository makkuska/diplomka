Databáze je soubor vzájemně propojených datových složek, který slouží pro efektivní ukládání dat \citep{Oppel2009}. Ta jsou ukládána ve formě tabulek, entit a atributů, a jsou vzájemně propojeny logickými vazbami, které se nazývájí {\it relace} \citep{Connolly2005}. Toto logické uložení vazeb mezi tabulkami umožňuje efektivní manipulaci s daty, rychlé vyhledávání i komplexní analýzu \citep{Momjian2001}. 

Základy {\it relační databáze} položit v roce 1970 matematik E. F. Codd, který relačnímu modelu přidal i srozumitelné příkazy vycházejících z běžné angličtiny, které jsou dnes známy jako jazyk {\it SQL} (Structured Query Language)\citep{Zak2001}. V dnešní době je možné setkat se také s pojmy objektová a objektově-relační databáze, které přebírají řadu vlastností z oblasti objektového programování.

Obvykle se rozlišují pojmy databáze, který odkazuje na obecný koncept, a pojem databázový systém nebo přesněji {\it systém řízení báze dat} \footnote{angl. Database Management System (DBMS)}, což je konkrétním počítačovým program, který zajišťuje fyzické uložení dat. Moderní SŘBD jsou navrženy na principu klient/server, kdy databáze běží jako služba na pozadí a čeká na dotazy od klientů. Server umožňuje uživatelům přístup k databázi, vytváření a aktualizování dat, stejně jak jako vyhledávání či analyzování \citep{Connolly2005}. Uživatel s databázi komunikuje skrze jazyk SQL většinou v kombinaci s vlastním jazykem daného databázového systému.

Pro uložení dat malého projektu je samozřejmě možno použít i jiného formátu určeného pro ukládání dat, například tabulkového procesoru, formátu XML, či moderního JSON. Pro komplexní správu dat velkého projektu je však databáze pro své relace více než vhodná. 

Prostorová databáze, někdy také zvaná geodatabáze, není nic jiného než databáze přidaná o datový typ určený pro ukládání prostorové informace o prvku, prostorové indexy a sadu funkcí vhodných pro správu prostorových dat. Více informací o prostorových databázích v kapitole \odkazKapitola{PostgreSQL} PostgreSQL 9.x (PostGIS) a \odkazKapitola{MSSQL} MS SQL Server 2008. 

Z toho vyvstává otázka, co jsou prostorová data, také zvaná geodata. Z pohledu společnosti Esri se jedná se prvky, které nesou informaci o geografické poloze, zakódovanou informaci o tvaru (bod, line, polygon) a popis geografického jevu. Tato geodata jsou uložená ve formátu, který je možno použít v geografickém informačním systému \citep{Esri2006}. Příkladem takového formátu může být vektorový Esri shapefile, Esri coverage, GML, KML, GeoJSON nebo rastrový Erdas Image a GeoTIFF. Dalším způsobem je již zmíněná databáze, do níž se vektorová data ukládají ve specifickém tvaru daném standardem OGC\footnote{OGC standardy jsou kontrolované konsorciem Open Geospatial Consortium, zdroj \url{http://www.opengeospatial.org/ogc}} Simply Feature for SQL 1.2.1, který specifikuje způsob uložení dat v digitální podobě. Simple Features je založen na 2D geometrii s~možností lineární interpolace mezi lomovými body. To umožňuje vložení následujících prvků:

        \begin{itemize}
          \item bod - POINT(0 0),
          \item linie - LINESTRING(0 0, 1 1, 1 2),
          \item polygon - POLYGON ((0 0,4 0,4 4,0 4,0 0),(1 1, 2 1, 2 2, 1 2,1 1)),
          \item série bodů - MULTIPOINT((0 0),(1 2)),
          \item série linií - MULTILINESTRING((0 0,1 1,1 2),(2 3,3 2,5 4)),
          \item geometrická kolekce, která může obsahovat různé geoprvky (body, linie i polygony) - GEOMETRYCOLLECTION(POINT(2 3),LINESTRING(2 3,3 4))\footnote{zdroj http://postgis.net/docs/manual-2.1/using\_postgis\_dbmanagement.html\#RefObject}.
        \end{itemize}

První slovo specifikace určuje druh prvku (point, linestring, polygon, multipoint,~...), následují v závorce vypsané souřadnice lomových bodů. Za tím ještě může následovat volitelný parametr kód souřadnicového systému.

Hodnoty lze dále vkládat přes Well-Known Binary (WKB) nebo Well-Known Text (WKT) reprezentaci. PostGIS funkce pro vkládání geometrie vypadá následovně:

        \begin{itemize}
          \item ST\_AsBinary(geometry) pro bitový zápis WKB
          \item ST\_AsText(geometry) pro WKT text
        \end{itemize}

        Příklad uložení linie do databáze s jedním lomovým bodem v souřadnicovém systému WGS84:

        \texttt{(LINESTRING(15.91 50.84, 17.20 49.64, 18.92 49.82), 4326)}
