Databáze je strukturovaná kolekce dat, která slouží pro efektivní ukládání dat a jejich zpětně čtení \citep{Oppel2009}. V relační databázi jsou data ukládána ve formě tabulek, tedy entit a atributů, které jsou vzájemně propojeny vazbami mezi entitami \citep{Connolly2005}. Toto logické uložení vazeb mezi tabulkami umožňuje efektivní manipulaci s daty, rychlé vyhledávání i komplexní analýzu \citep{Momjian2001}. 

Základy {\it relační databáze} položil v roce 1970 matematik E. F. Codd, který relačnímu modelu přidal i srozumitelné příkazy vycházející z běžné angličtiny, které jsou dnes známy jako jazyk {\it SQL} \footnote{Structured Query Language}\citep{Zak2001}. V dnešní době je možné setkat se také s pojmy objektová a objektově-relační databáze, které přebírají řadu vlastností z oblasti objektového programování.

Obvykle se rozlišují pojem databáze, který odkazuje na obecný koncept, a pojem databázový systém nebo přesněji {\it systém řízení báze dat} \footnote{angl. Database Management System (DBMS)}, což je konkrétním počítačovým program, který zajišťuje fyzické uložení dat. Moderní SŘBD jsou navrženy na principu klient/server, kdy databáze běží jako služba na pozadí a čeká na dotazy od klientů. Server uživatelům umožňuje skrze jazyk SQL přístupovat k databázi, vytvářet a aktualizovat data, stejně jak jako vyhledávat či analyzovat \citep{Connolly2005}.

Pro uložení dat malého projektu je samozřejmě možno použít i jiného formátu určeného pro ukládání dat, například soubory formátu XLS, XML, CSV či moderního JSON. Pro komplexní správu dat velkého projektu je však databáze pro své relace více než vhodná. 

{\it Prostorová databáze}, někdy také zvaná {\it geodatabáze}, není nic jiného než databáze obohacená o datový typ určený pro ukládání prostorové informace o prvku, prostorové indexy a sadu funkcí vhodných pro správu prostorových dat. Více informací o prostorových databázích viz kapitola \odkazKapitola{kPostgreSQL} PostgreSQL 9.x (+ PostGIS) a \odkazKapitola{MSSQL} Microsoft SQL Server. 

{\it Prostorová data}, také zvaná {\it geodata}, jsou z pohledu společnosti Esri prvky, které nesou informaci o geografické poloze, zakódovanou informaci o tvaru (bod, line, polygon) a popis geografického jevu. Tato geodata jsou uložená ve formátu, který je možno použít v geografickém informačním systému \citep{Esri2006}. Příkladem takového formátu může být rastrový Erdas Image, Esri grid, GeoTIFF, PNG,  JPEG2000 nebo vektorový Esri shapefile, Esri coverage, GML, KML, DFX, DGN, GeoJSON nebo GeoHash. 

Dalšími formáty, používanými právě pro ukládání dat do databáze, jsou Well-Known Binary (WKB) a Well-Known Text (WKT) pro reprezentaci vektorových dat. Jejich tvar je dán standardem OGC\footnote{OGC standardy jsou kontrolované konsorciem Open Geospatial Consortium,\newline zdroj \url{http://www.opengeospatial.org/ogc}} {\it Simply Features for SQL 1.2.1}, který specifikuje model pro uložení prostorových dat v digitální podobě. 

Simply Features je založen na 2D geometrii s~možností lineární interpolace mezi lomovými body. Základní prvky\footnote{zdroj \url {http://postgis.net/docs/manual-2.1/using\_postgis\_dbmanagement.html\#RefObject}}, které je možno vkládat ve formátu druh prvku a~souřadnice lomových bodů v~závorce, jsou:
\begin{itemize}
\item bod - \texttt{POINT(0 0)},
\item linie - \texttt{LINESTRING(0 0, 1 1, 1 2)},
\item polygon - \texttt{POLYGON ((0 0,4 0,4 4,0 4,0 0),(1 1, 2 1, 2 2, 1 2,1 1))},
\item série bodů - \texttt{MULTIPOINT((0 0),(1 2))},
\item série linií - \texttt{MULTILINESTRING((0 0,1 1,1 2),(2 3,3 2,5 4))},
\item série polygonů - \texttt{MULTIPOLYGON(((0 0,4 0,4 4,0 4,0 0),(1 1,2 1,2 2,1 2,1 1)), ((-1 -1,-1 -2,-2 -2,-2 -1,-1 -1)))} a
\item geometrická kolekce, která může obsahovat geoprvky různých typů (body, linie, polygony) - \texttt{GEOMETRYCOLLECTION(POINT(2 3),LINESTRING(2 3,3 4))}.
\end{itemize}

Standard OGC definuje i sadu prostorových funkcí, založených na jazyku SQL, určených pro získávání prostorových vlastností prvků a vztahů mezi nimi, pro ma\-ni\-pu\-la\-ci s geometrií a pro analýzu prostorových vztahů. Základními typy funkcí jsou vytváření prostorových dat a jejich správa, konverze mezi formáty, porovnání na základě geometrie a vyhledávání na základě prostorového dotazu. Dle standardu jsou funkce pro získání geometrie z databáze:
\begin{itemize}
\item \texttt{ST\_AsBinary(geometry)} pro bitový zápis WKB a
\item \texttt{ST\_AsText(geometry)} pro textovou podobu WKT.
     \end{itemize}

