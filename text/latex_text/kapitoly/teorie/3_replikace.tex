        \subsection{Replikace}

        Replikace je proces, u kterého jsou data a databázové objekty kopírované z
        jednoho databázového serveru na druhý a poté synchronizovány pro zachování
        souladu obou databází. Synchronizací v tomto případě myslíme kopírováním všech
        změn, které v databázi nastanou. Použitím databáze je možno data distribuovat
        na různě vzdálená místa nebo mezi mobilní uživatele v rámci počítačové sítě a
        internetu \citep{Microsoft2013}.

        Mnohé moderní aplikace se musí zabývat velkým počtem přístupů do databáze, což
        může v některých případech způsobovat problémy. Buď je server přetížen počtem
        připojení a data tedy přicházejí k uživateli pomalu, nebo dokonce úplně
        vypadne. 

        Mezi časté důvody použití databázové replikace tedy patří zajištění dostupnosti
        dat\footnote{Anglicky High Availability}, resp. snížení pravděpodobnosti, že
        data nebudou dostupná, což může být způsobeno již zmíněným výpadkem serveru
        nebo například fyzickou ztrátou dat \citep{ObeHsu2012}. Další důvodem je
        rozložení zátěže přístupů do databáze mezi více serverů, takže nebude docházet
        ke zpomalení výkonu hlavního serveru ani k situaci, že data nebudou dostupná
        kvůli jeho výpadku \citep{BellKindahlThalmann2010}. Databáze je často
        zálohovaná, například skriptem dump a i to může server zpomalit. Vhodným
        řešením je tedy nejdříve vytvořit kopii dat na jiný datový server a až poté
        proces zálohování spustit. 

        Všechny databáze zapojené do procesu replikace jsou v odborné literatuře
        nazývané uzly, v angličtině node. Tyto uzly dohromady tvoří replikační
        cluster\footnote{Volně přeloženo skupina serveru zapojených do replikace}. Při
        správně nastavené replikaci, by v clusteru nikdy neměly být méně než 3 uzly.
        Může se totiž stát, že vypadne jeden ze dvou uzlů, čímž dojde, ikdyž jen na
        krátkou chvíli, k situaci, že data nebudou v daný okamžik zálohovaná. 

        Uzly v replikačním clusteru mohou mít jednu ze dvou základních rolí, nejčastěji
        nazývaných Master a Slave. Master server nebo pouze Master je server, který
        poskytuje data k replikaci, má práva na čtení i zápis a probíhají tedy na něm
        veškeré aktualizace. Je možno se setkat také s pojmenováním Primary server,
        Provider, Sender, Parent nebo Source server. Naprosto jiný pojem zavádí MS SQL
        Server, který tento zdrojový server nazývá Publisher (česky Vydavatel). Druhý
        databázový server je nejčastěji nazýván Slave, Standby, Reciever, Child nebo
        Subsciber (česky Odběratel). Poslední pojem je také používán MS SQL Serverem.
        Na tento server, který je dostupný vždy jen pro čtení dat, se data a
        aktualizace kopírují, není však možné na něj změny zapisovat
        \citep{RiggsKrossing2010}.

        %parametr H říká že to bude přímo na tom místě kde je v textu...více http://en.wikibooks.org/wiki/LaTeX/Floats,_Figures_and_Captions
          \begin{figure}[H]
            \centering
            \includegraphics[scale=1]{../../../grafy/obr/schema_masterMasterSlave_pokus.png}
            \caption {Srovnání Master-Master a Master-Slave replikace}
            \label{srovnaniM-M-S}
          \end{figure}

        Podle počtu Master a Slave serverů v replikačním clusteru, se rozlišuje zda se
        jedná o jednosměrnou nebo obousměrnou replikaci. Tzv. Master-Master replikace
        umožňuje zapisovat do všech uzlů v replikačním clusteru, což může být praktické
        například při použití databáze offline \odkazObrazek{srovnaniM-M-S}. Změny
        se tedy synchronizují mezi všemi databázovými uzly. Tento způsob však nese
        značné komplikace, je potřeba řešit konflikty změn ve stejných datech a je
        relativně náročný na údržbu. Tato práce se zabývá použitím druhé způsobu, tzv
        Master-Slave replikace. Tato replikace používá vždy jen jeden Master server v
        clusteru a dva a více Slave servery. Kopie dat tedy probíhá jednosměrně, vždy z
        Master na Slave servery. Podle Bella (2010) mají moderní aplikace často více
        čtenářů než zapisovatelů, proto je zbytečné, aby se všichni čtenáři připojovali
        na stejnou databázi jako zapisovatelé a zpomalovali tím jejich práci
        \citep{BellKindahlThalmann2010}. Z toho důvodu je tedy použití Master-Slave
        replikace více než vhodné.

        Při návhu je potřeba se zamyslet také nad způsobem replikace, zda bude
        synchronní či asynchronní. Synchronní replikace zajišťuje, že na Master
        serveru nikdy neproběhne nová transakce, dokud se poslední transakce
        neprovede na Slave serveru \citep{Boszormenyi2013}. Tento
        přístup zajistí, že žádná data nebudou v průběhu transakce ztracena. V
        některých případech tento způsob může zbytečně zpomalit rychlost
        přístupu do databáze, protože je nutno čekat na každou nedokončenou
        transakci. Naopak při bankovních transakcích, kde je potřeba, aby
        všechny operace proběhly na všech stranách, je tento způsob nezbytný.
        Druhým způsobem je asynchronní replikace, při které se nová data mohou
        zapisovat na Master server, přestože ještě nedošlo k replikaci
        stávajících dat na Slave server \citep{ObeHsu2012}. 
        
        \begin{figure}[H]
          \centering
          \includegraphics[scale=1]{../../../grafy/obr/schema_asyncSync.png}
          \caption {Rozdíl mezi synchronní a asanchronní replikací}
        \end{figure}

        Replikace v PostgreSQL umožňuje plnou kopii dat z databáze i pouze
        výběr některých tabulek. Více o možnostech a způsobech nastavení
        replikace v kapitole PRAKTICKÁ ČÁST :)

        Každý databázový server (myšleno SŘDB) si volí terminologii a konkrétní
        nastavení mírně odlišně. Tato kapitola se snaží popsat chápání
        replikace co v největší míře obecně s ohledem na použití tohoto pojmu v
        PostgreSQL. Zcela jinou terminologii, ikdyž založenou na stejných
        principech, zavádí MS SQL Server, který používá pojmy transakční
        replikace pro Master-Slave replikace a slučovací replikaci pro
        Master-Master replikaci. 

          \begin{figure}[H]
            \centering
            \includegraphics[scale=1]{../../../grafy/obr/schema_kaskadova.png}
            \caption{Ukázka kaskádové replikace}
            \label{srovnaniM-M-S}
          \end{figure}

