\label{kDiskuze}
Jedním z požadavků pro výběr databázového systému bylo rozšířené používání v~oblasti
geoinformatiky a~zároveň podpora ArcGIS produkty. Z tohoto důvodu nemohl být použit
databázový server MySQL, který je sice v oblasti GIS používaný, není však
podporovaný produkty Arc\-GIS. Opačný problém je řešen u MS SQL Serveru, který je
doposud na katedře používán a který je podporován ArcGIS produkty, není však
tak široce používaný v oblasti geoinformatiky. Jeho rozšíření o prostorová data
totiž například nepoužívá nástroje pro transformaci dat, které jsou v~PostgreSQL řešeny knihovnou OGR. 

U výběru replikačního nástroje se vycházelo z nativního řešení pro PostgreSQL, který
podporuje pouze master-slave replikaci. Při velkém počtu editací by přicházelo v~úvahu použití multimaster replikace, pro kterou je však třeba použít
některého z~externích nástrojů a jejíž konfigurace je složitější. 

Replikační řešení je navrženo tak, že je zcela nezávislé na technologii ArcSDE.
ArcSDE je v tom řešení pouze prostředníkem pro připojení databáze k ArcGIS
produktům. Replikaci je tedy možno nastavit bez ohledu na ArcSDE, přičemž po přidání schématu
\texttt{sde} do databáze, lze replikovat také tabulky toho schématu. 

Jak už bylo několikrát zmíněno, pro fungování celého řešení, tedy včetně
připojení databáze k ArcGIS produktům, je potřeba zajistit kompatibilitu jednotlivých verzí.
Tento fakt byl zjištěn příliš pozdě, proto se jej při vytváření testovacího prostředí nepodařilo dodržet.
Přesto byly všechny programy nastudovány, otestovány a podrobně v této práci zdokumentovány,
při příštím vytváření databázového clusteru by tedy nemělo dojít k výše zmíněným problémům.

K databázi přes ArcSDE lze připojit jak verze ArcGIS for
Desktop, tak ArcGIS for Server. Verzi ArcGIS Online nelze připojit k databázi
přímo, ale pouze přes vrstvu publikovanou ArcSDE. 

Pro efektivní fungování databáze by ještě předtím než bude
datázové řešení spuštěno do plného provozu, měla být provedena optimalizace nastavení
databázového systému a testování zátěže. Optimalizace by měla zohlednit počty uživatelů
přistupujících k databázi stejně tak jako náročnost jejich dotazů. V rámci
optimalizace je možno testovat zátěž pomocí pgbench, důvody pro jeho použití
byly nastíněny v kapitole \odkazKapitola{kpgbench}, nebo jiného dostupného
nástoje. Jedná se však o složitý problém, který vyžaduje bohaté zkušenosti s
provozem databázového systému a široké znalosti nejen v oblasti databází, ale
také počítačové techniky.  

Nástroj pgpool, který v návrhu zajišťuje rozkládání dotazů mezi servery v
clusteru, nesleduje, jaké je aktuální vytížení serveru jinými procesy. Pro řízení
load ba\-la\-ncing používá náhodného rozdělování dotazů mezi jednotlivé uzly. 
Pokud už na serveru běží jiné služby (např. geoportál), které nesmí být ve svém
provozu omezeny, je možno zátěž, kterou pgpool směřuje na jednotlivé slave
servery, ovlivnit parametrem \texttt{backend\_weight}. Jeho hodnotou lze proporčně
nastavit zátěž jednotlivých uzlů, jak již bylo ukázáno v kapitole
\ref{kRozlozeni}. Takto lze například snížit počet dotazů, které jsou
směřovány na server, zatížený jinými službami.

