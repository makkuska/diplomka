Cílem diplomové práce je provést rešerši v oblasti dostupných replikačních řešení a na jejím základě prakticky otestovat proces synchronizace a replikace geodat, které je možnost v kombinaci s ArcGIS produkty.

V teoretické části práce budou podrobně definovány pojmy týkající se zálohování dat, především však synchronizace a replikace, dále deteilně rozebrána replikace ve všech možných variantách nastavení, tedy jednosměrná, dvousměrná, synchronní, asynchronní, kaskádová, logická a fyzická. Dále rozbor zahrne celé portfólio produktů od desktop řešení, přes možnosti ArcGIS serveru až po cloudový ArcGIS online.

V rešerší části budou diskutovány dva databázové server, SQL Server a PostgreSQL, oba podporované ArcGIS produkty a jejím základě pak vybrát jeden, na kterém pak proces replikace bude prakticky testován.

Praktická část se bude zabývát návrhem replikačního řešení, které zahrne požadavky a možnosti katedry a bude brát v úvahu její způsoby využívání databáze. Na základě rešerše pak bude vybráno replikační řešení, připraveno testovací prostředí na základě všech výše zmíněných kritérií a na konec i praktickému testování výše zmíněných procesů.

Postupnými opakovanými procesy budou sledovány dílčí parametry procesu (rychlost procesu, úplnost, chybovost, podporované formáty). 
