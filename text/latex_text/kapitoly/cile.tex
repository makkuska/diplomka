
      Cílem diplomové práce je provést rešerši a na jejím základě
      prakticky otestovat proces synchronizace a replikace geodat, které
      se dnes objevují napříč platformou \gls{Esri}. V teoretické části práce
      bude detailně analyzován proces synchronizace a replikace ve všech
      možných variantách (jednosměrná, dvousměrná, synchronní,
      asynchronní, ...) a popsány prostředky, které se na platformě Esri k
      těmto procesům využívají. Rozbor zahrne celé portfólio produktů od
      desktop řešení, přes možnosti ArcGIS serveru až po cloudový ArcGIS
      online. Budou popsány možnosti, požadavky a předpoklady pro úspěšnou
      realizaci.

      V praktické části, nad existujícími katedrálními daty, dojde k
      praktickému testování těchto procesů na předem připraveném
      testovacím prostředí. Postupnými opakovanými procesy budou sledovány
      dílčí parametry procesu (rychlost procesu, úplnost, chybovost,
      podporované formáty). Vyjde se z primárně podporovaného databázového
      stroje SQL Server, který bude konfrontován s možnosti dalšího
      podporovaného systému PostgeSQL.

      Můj jeden odstaveček - něco jako - jak vidím vlastní přínos do tématu. 
