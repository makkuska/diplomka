
      Úvod je v rozsahu jedné strany. Je důležitou součástí práce. Uvádí
      do problematiky řešené v bakalářské / magisterské práci. Student v
      něm vyjadřuje potřebu řešení zadaného tématu, případně návaznost na
      jeho předcházející práce. Doporučuje se, aby byl do své konečné
      podoby dopsán až jako poslední, tj. až po napsání celého textu.

      Dnešní trend je stále více dat ukládat a ponechávat pouze v
      digitální podobě. Mnoho dokumentů už se vůbec netiskne do papírové
      podoby. Tento trend dnes podporují i elektronické podpisy, díky
      kterým je tištěná verze naprosto zbytečná. S přibývajícím počtem
      dat je však třeba řešit komplikace, které počítačová data přináší.
      Počítačoví experti řeší například kam ukládat tak velké množství
      dat, jak data aktualizovat, jak zabránit poškození dat ať už
      způsobených lidským faktorem či poškozením hardwaru. V připadě, že
      se poškodí disk, můžeme často během okamžiku přijít o všechna
      data, někdy však pro ztrátu dat stačí stikntou pouhé jedno
      tlačítko na klávesnici. Určitě už se vám nejednou stalo, že jste
      se nemohli přihlásit do svého účtu na internetu z důvodu přetížení
      serveru. Jak zabránit těmto komplikacím, které mohou poškodit či
      zcela zničit celou dosavadní práci nebo zrychlit celý proces práce
      s tady? Řešením velkého počtu výše uvedených problémů je replikace
      dat. Jedná se pokročilou funkci, kterou nabízí dnešní databázové
      servery, zajišťující robustnost databáze a vysokou dostupnost dat
      tím, že data zkopíruje na více serverů.
      
      Replikaci lze využít ve všech odvětvích, které pracují s daty.
      Výjimkou tedy není ani geoiformatika, která pracuje s velkým
      počtem dat, které navíc nesou informaci o geografické poloze. Z
      mého pohledu data středně velkého nebo velkého projektu je
      nejvhodnější ukládata do databáze. Nabízí nám to sofistikované
      uložení dat, snadné propojení jednotlivých vrstev, snadnou
      přenostitelnost dat a další. Replikace se dá využít pro kopii 
      dat a následnou aktualizaci změn. Výhodou databáze je, že se při
      změně jednoho prvku, aktualizuje v databázi pouze jeden řádek a
      nekopíruje se znovu celá databáze, což je jednoznačná výhoda
      oproti binárnímu uložení dat napřiklad ve formátu shapefile.
      
      Replikaci ocení určitě i uživatelé, které pracují na jednom
      projektu. Z hlediska rychlosti práce s databází je výhodnější mít
      databázi přimo na počítači, na kterém pracují, než data in-real
      time stahovat ze serveru. Po dokončení editace se data replikují
      prostřednictvím počítačové sítě nebo internetu. Dobrým příkladem
      využitelnosti replikace je také nový trend využívání offline
      mobilních aplikací v mobilních telefonech. Databáze se vždy
      replikuje do mobilního telefonu, vždy když se klient připojít na
      internetovou síť, aplikace kontroluje zda není na serveru novější
      verze databáze a pokud je, zkopíruje pouze změny, které proběhly
      od posledního stahování. (Jako příklad z geoinformatického
      prostředí bych uvedla diplomovou práci Dalibora Janáka, který řeší
      replikaci databáze lezeckých cest do mobilní aplikace.) Databázové
      systémy nabízí širokou škálu nastavitelnosti, která umožňuje
      přizpůsobit replikaci danému řešení. 

