\label{kZaver}
Tato práce hodnotí možnosti dostupných replikačních řešení a na základě toho navrhuje databázové řešení s ohledem na možnosti a požadavky katedry. V rešerší části byly vymezeny pojmy synchronizace, replikace a související pojem verzování a~popsána replikace včetně variant synchronní, asynchronní, jednosměrné, obou\-smě\-rné, kaskádové, logické i fyzické. Byly rozebrány požadavky na databázové ukládání dat jednotlivých produktů ArcGIS a byla podrobně popsána technologie ArcSDE, která se v ArcGIS produktech používá pro připojení k databázi.

Na základě rešerše byl vybrán databázový systém PostgreSQL, který je možno použít v kombinaci s produkty ArcGIS, což bylo jedním z hlavních požadavků pro výběr databázového systému. Byl sestaven návrh databázového řešení, který zohledňuje všechny požadavky katedry a možnosti daných technologií. Bylo vytvořeno testovací prostředí na serveru poskytnutém katedrou, na němž byly dané procesy otestovány. Na základě toho byl pak sepsán podrobný popis toho, jak nastavit replikaci ve variantě streaming a Slony-I. Návrh zahrnuje také možnost použití nástroje pgpool pro rozložení zátěže mezi servery v databázovém clusteru.

Návrh databázového řešení slibuje zvyšení interoperability, usnadnění sdílení dat a~dodržování licenčních podmínek, zajištění vysoké dostupnosti a aktuálnosti dat. Studenti navíc budou mít možnost vyzkoušet si pokročilou práci s databází, která je může lépe připravit na budoucí zaměstání.
