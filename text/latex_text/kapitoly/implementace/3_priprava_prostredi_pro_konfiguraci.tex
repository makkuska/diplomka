\subsection{Příprava prostředí pro konfiguraci replikace}
\label{kPriprava}
Na začátku je potřeba připravit testovací prostředí hned s několika závislostmi. Hlavním používaným software je PostgreSQL s extenzemi PostGIS, Slony-I a pgpool. Informace o instalacích jednotlivých komponent jsou dostupné na jejich webových stránkách, ve Windows si stačí stáhnout pouze instalační balík pro PostgreSQL, který umožňuje instalaci databázového systému včetně všech výše zmíněných extenzí. Pro grafickou administraci databáze je doporučený, ale nepovinný, program PgAdminIII, který je taktéž multiplatformní. Většina konfigurace je zde popisována skrze příkazový řádek, neznamená to však, že nemá ekvivalentní použití skrze grafické rozhraní.

Všechny technologie byly testovány na Debian-based Linux, Slony-I také na Windows XP. 

U všech typů replikace je potřeba mít vytvořeného databázového uživatele s právem pro replikaci, pod kterým bude daný proces probíhat. Je možné vytvořit nového uživatele a nastavit mu tato práva nebo použít již existující účet \texttt{postgres}, který jako SUPERUSER obsahuje také práva pro replikaci. Je však potřeba mu hned na začátku po instalaci změnit heslo.

Změna hesla:

\begin{lstlisting}
ALTER ROLE postgres ENCRYPTED PASSWORD 'kgigis';
\end{lstlisting}

Vytvoření nového uživatele:

\begin{lstlisting}
ADD ROLE replikator REPLICATION;
ALTER ROLE replikator ENCRYPTED PASSWORD 'kgigis';
\end{lstlisting}

Aby bylo možné pracovat s databází, je nejdříve nutné chápat význam jednotlivých konfiguračních souborů a mít přehled o souborové struktuře PostgreSQL. Vzhledem k tomu, že si ji každý systém uzpůsobuje podle sebe, nezbývá než po instalaci PostgreSQL nastudovat, kde se jaký soubor nachází. Naštěstí existuje tabulka \texttt{pg\_settings}, která uchovává veškeré informace o nastavení databáze. 

Příklad SQL příkazu, spuštěného na serveru geohydro, který vypíše umístění jednotlivých souborů a složek:

\begin{lstlisting}
  SELECT name, setting FROM pg_settings WHERE category = 'File Locations';
\end{lstlisting}

      \begin{table}[H]
        \label{fileLocation}
          \begin{center}
            \begin{tabular}{lll}
              \texttt{name} & &\texttt{ settings}\\
              \texttt{--------------------------------------}&\texttt{+}&\texttt{---------------------------------------------------------------------------------}\\
                                    \texttt{data\_directory} & \texttt{|}&\texttt{ /var/lib/postgresql/9.1/main} \\
              \texttt{external\_pid\_file} & \texttt{|}&\texttt{ /var/run/postgresql/9.1-main.pid} \\
                      \texttt{hba\_file} & \texttt{|}&\texttt{ /etc/postgresql/9.1/main/pg\_hba.conf}\\ 
                     \texttt{config\_file} & \texttt{|}&\texttt{ /etc/postgresql/9.1/main/postgresql.conf} \\
                    \texttt{ident\_file} & \texttt{|}&\texttt{ /etc/postgresql/9.1/main/pg\_ident.conf} \\
            \end{tabular}
          \end{center}
      \end{table}

Jak je možno vidět, existují tři hlavní konfigurační soubory:
\begin{itemize}
  \item postgres.conf, který ovládá obecné nastavení jako výchozí uložiště, kterým IP databáze naslouchá, velikost alokované paměti a další,
  \item pg\_hba.conf, který povoluje konkrétním uživatelům přístup z určitých IP adres,
  \item pg\_ident.conf, který slouží k mapování uživatel operačního systému na uživatele PostgreSQL \citep{ObeHsu2012}.
\end{itemize}

Předtím, než k dojde samotné konfiguraci replikace, je potřeba vždy dosáhnout totožného stavu všech databázových serverů v clusteru. Narozdíl od streaming replikace, kde je možné začít s prázdným databázovým serverem bez databází a tabulek, Slony-I replikace vyžaduje již vytvořenou strukturu databáze včetně tabulek. 

Pro názornost byla vytvořena databáze univerzita pomocí příkazu CREATE DATABASE a tabulka studenti pomocí CREATE TABLE. Slony-I navíc vyžaduje existenci \texttt{primary key}.

 U systémů, které nemají stejnou architekturu, není možné dosáhnout totožné kopie na úrovni binárních data, protože adresářové struktury obou systémů se mohou lišit. V takovém případě je možné zajistit kopii dat například skriptem \texttt{pg\_dump} na master a \texttt{pg\_restore} na slave serveru. Tímto způsobem, který oceníme především u Slony-I replikaci, lze převádět jak strukturu databáze, tak data, zároveň je zde možnost přenášet pouze vybrané části databáze. 

\begin{lstlisting}
  CREATE DATABASE student;
  student=# CREATE TABLE student (id int, jmeno varchar, rodne_mesto geometry, primary key(id));
\end{lstlisting}
\begin{lstlisting}[keywordstyle=\color{black},identifierstyle=\color{black},stringstyle=\color{black}]
  NOTICE:  CREATE TABLE / PRIMARY KEY will create implicit index "student_pkey" for table "student"
  CREATE TABLE
\end{lstlisting}

Naopak, pokud se má jednat o kopii dat a konfiguračních souborů mezi dvěmi databázemi stejného systému a architektury, je možné kopírovat celou složku, tedy přímo binární data. Je mnoho způsobů, jak toho dosáhnout, například klasickým kopírováním skrze skript \texttt{cp}, resp. \texttt{scp} u vzdálených složek, nebo skriptem \texttt{rsync}. Kopírování dat za běhu databáze navíc vyžaduje použití příkazu \texttt{SELECT pg\_start\_backup}, který zajistí, že bude archivovat transakční log do té doby než proběhne příkaz \texttt{SELECT pg\_stop\_backup}. Tím nepřijdeme o žádné změny.

Příklady možných způsobu zkopírování dat na repliku:
\begin{lstlisting}
  SELECT pg_start_backup('backup', true)
\end{lstlisting}

\begin{lstlisting}[keywordstyle=\bfseries\color{purpurova7},identifierstyle=\color{black},stringstyle=\color{black}]
  scp -r root@158.194.94.93:/var/lib/postgresql/9.1/main /var/lib/postgresql/9.1/main
\end{lstlisting}
\begin{center}
nebo 
\end{center}
\begin{lstlisting}[keywordstyle=\bfseries\color{purpurova7},identifierstyle=\color{black},stringstyle=\color{black}]
  rsync -ave ssh root@158.194.94.93:/var/lib/postgresql/9.1/main /var/lib/postgresql/9.1/main
\end{lstlisting}

\begin{lstlisting}
  SELECT pg_stop_backup()
\end{lstlisting}

Nabízí se ještě možnost použití skriptu přímo určeného pro zálohování dat v PostgreSQL \texttt{pg\_basebackup}. Tento příkaz mimo jiné umožňuje kopírování dat za běhu replikace bez nutnosti použití \texttt{pg\_start/stop\_backup}.

Použití \texttt{pg\_basebackup} pro vytvoření repliky:

\begin{lstlisting}[keywordstyle=\bfseries\color{purpurova7},identifierstyle=\color{black},stringstyle=\color{black}]
  pg_basebackup -D /var/lib/postgresql/9.1/main/ -U replikator -h 158.194.94.93
\end{lstlisting}

Pří kopírování celé datové struktury je vhodné zajistit jednotlivým souborům a složkám správa. Vzhledem k tomu, že databázový systém zapisuje do složky s daty \texttt{(data\_directory)}, musí mít postgres, i po zkopírování celé datové struktury na jiný server, práva pro zápis.

Kopie souborové struktury dat je potřeba především pro streaming replikaci. Tento krok je velice důležitý pro správný chod replikace. V případě, že se data nespávně zkopírují, není možné replikaci zprovoznit. 

Obecně platí, že kopírování databázového systému bez dat rychlejší než z daty. Pokud se tedy začíná s prázdným databázovým systémem, je vhodné nastavit replikaci dřív, než se data začnou do databáze přidávat. V případě, že už databáze naplněná daty je, není problém replikaci nastavit, jen je třeba počítat s delším časem kopírovaní dat a větší opatrností při konfiguraci. 

 A v neposlední řadě je potřeba zajistit konektivitu obou, resp. všech serverů v replikačním clusteru. S tím souvisí i nutnost nastavení povolení přístupů z IP adres slave serverů, kterou je možno zajistit skrze konfigurační soubor \texttt{pg\_hba.conf}. Následující příklad ukazuje možné nastavení souboru \texttt{pg\_hba.conf} na master serveru. Povoluje uživetelům market a replication, přihlášených z dané IP adresy, přistupovat na master server a číst, resp. replikovat data.

      \begin{table}[H]
        \label{pgHba}
          \begin{center}
            \begin{tabular}{lllll}
              \texttt{\#host} & \texttt{DATABASE} & \texttt{USER} & \texttt{ADDRESS} & \texttt{METHOD} \\
                \texttt{host} & \texttt{all} & \texttt{market} & \texttt{80.188.74.1/32} & \texttt{md5} \\
       \texttt{host} & \texttt{replication} & \texttt{replication} & \texttt{80.188.74.1/32} &	\texttt{md5} \\
            \end{tabular}
          \end{center}
      \end{table}

