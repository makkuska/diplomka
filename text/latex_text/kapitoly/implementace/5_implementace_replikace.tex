\subsection{Konfigurace replikace}
\subsubsection{Slony-I replikace}
Ukázka konfiguračního souboru na master serveru pro inicializaci clusteru (init cluster) s názvem init\_master.txt uloženého ve složce z daty:
\begin{lstlisting}[basicstyle=\footnotesize\ttfamily,identifierstyle=\color{black},stringstyle=\color{black},keywordstyle=\color{black},
]

cluster name = second_cluster;
# definice uzlu
node 1 admin conninfo='dbname= host=192.168.1.1 user=replikator password=kgigis';
node 2 admin conninfo='dbname= host=192.168.1.2 user=replikator password=kgigis';

# inicializace clusteru
init cluster (id=1, comment = 'master');
store node (id=2, comment = 'slave1', event node=1);

# vytvoreni replikacniho setu a pridani tabulek do setu
create set (id=1, origin=1, comment='Tabulky k replikaci');

set add table (set id=1, origin=1, id=1, fully qualified name = 'public.student', comment='prehled studentu');
set add table (set id=2, origin=1, id=1, fully qualified name = 'public.student2', comment='prehled studentu');

store path (server=1, client=2, conninfo='dbname= host=localhost user=postgres  password=tfgt');
store path (server=2, client=1, conninfo='dbname=repl2 host=localhost user=postgres  password=tfgt');

\end{lstlisting}
\subsubsection{Streaming replikace}
Inicializace asynchronní streaming replikace

Samotné nastavení replikace, není nijak náročné. V první fázi je potřeba konfigurace postgresql.conf na master servere. Stačí editace pouhých dvou parametrů, aby byla replikace v provozu. 

\begin{lstlisting}
wal_level = hot_standby
max_wal_senders = 1
\end{lstlisting}

Stejně tak je potřeba konfigurovat \texttt{postgresql.conf} na slave serveru. \texttt{Wal\_level} a \texttt{max\_level\_sender} může a nemusí zůstat stejný jako na masteru. Pokud však má slave být slave připraven, že by zastoupit master server v případě, že by master vypadl, pak je v hodné, aby hodnoty byly nastaveny stejně. Co je potřeba nastavit je \texttt{hot\_standby} a \texttt{hot\_stadby\_feedback}.

\begin{lstlisting}
wal_level = hot_standby
max_wal_senders = 1
hot_standby = on	
hot_standby_feedback = on
\end{lstlisting}

Tím však nastavení replikace nekončí. Ve složce s daty je potřeba vytvoři \texttt{recovery.conf}, který ponese informace o master serveru, viz příklad. 

\begin{lstlisting}
standby_mode = 'on'
primary_conninfo = 'host=158.194.94.93 user=replication password=kgigis'
\end{lstlisting}

Inicializace synchronní streaming replikace

Synchronní replikace vyžaduje v \texttt{postgresql.conf} na master serveru přidání hodnoty parametru \texttt{synchronous\_standby\_names}:

\begin{lstlisting}
synchronous_standby_names = 'synchronous_gis'
\end{lstlisting}

a dále v \texttt{recovery.conf} na slave serveru přidat parametr \texttt{application\_name}, který nese stejnou hodnotu jako výše změněný parametr:

\begin{lstlisting}
standby_mode = 'on'
primary_conninfo = 'host=158.194.94.93 application_name=synchrounous_gis user=replication password=kgigis'
\end{lstlisting}

Inicializace kaskádové streaming replikace

Kaskádová replikace vyžaduje nejméně tři servery v clusteru. Ale nejedná se o nic složitějšího než o změnu konfigurace souboru \texttt{recovery.conf} na replice, kde se hodnoty masteru vymění za hodnoty slavu, ze kterého chceme na danou repliku kopírovat. 

\begin{lstlisting}
standby_mode = 'on'
primary_conninfo = 'host=158.194.94.18 user=replication password=kgigis'
\end{lstlisting}

Přehled všech parametrů týkající se replikace včetně vystvětlení jejich smyslu je možné najít v oficiální dokumentaci\footnote{http://www.postgresql.org/docs/9.1/static/runtime-config-replication.html}.
